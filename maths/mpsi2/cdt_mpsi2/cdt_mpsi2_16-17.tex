\documentclass[12pt,a4paper]{article}

\textheight=25cm
\topmargin=-50pt
\textwidth= 17cm
\oddsidemargin=-.5cm


\usepackage{enumerate}
\usepackage{stmaryrd} %\sslash chapitre 19 au moins (parallèle)
\usepackage{amsfonts}
\usepackage{fancybox}
\usepackage{color}
\usepackage{eurosym}
\usepackage{amssymb}
\usepackage[T1]{fontenc}
\usepackage{amsmath}
\usepackage{theorem}
\usepackage[french]{babel}
\usepackage[utf8]{inputenc}
\usepackage{latexsym}
\usepackage{amscd}
\usepackage{indentfirst}
\usepackage[dvips]{graphicx}
\usepackage{textcomp}
\usepackage{mathrsfs}

\newcommand{\G}{\ensuremath{\mcal{G}}}
\newcommand{\cG}{\ensuremath{\mcal{G}}}
\newcommand{\cL}{\ensuremath{\mcal{L}}}
\newcommand{\lin}{\ensuremath{\mcal{L}}}
\newcommand{\rmdim}{\dim}
\newcommand{\dprod}{\displaystyle\prod}
\newcommand{\dsum}{\displaystyle\sum}
\newcommand{\intt}{\displaystyle\int}
\newcommand{\ml}{\ensuremath{\mcal{L}}}
\newcommand{\E}{\ensuremath{\mathbb{E}}}
\newcommand{\I}{\ensuremath{\mathcal{I}}}
\newcommand{\N}{\ensuremath{\field{N}}}
\newcommand{\Z}{\ensuremath{\field{Z}}}
\newcommand{\D}{\ensuremath{\field{D}}}
\newcommand{\M}{\ensuremath{\mcal{M}}}
\newcommand{\Q}{\ensuremath{\field{Q}}}
\newcommand{\R}{\ensuremath{\field{R}}}
\newcommand{\Rp}{\ensuremath{\field{R}_+}}
\newcommand{\Rpe}{\ensuremath{\field{R}_+^{\ast}}}
\newcommand{\Ret}{\ensuremath{\field{R}^{\ast}}}
\newcommand{\Rd}{\ensuremath{\field{R}^2}}
\newcommand{\Rt}{\ensuremath{\field{R}^3}}
\newcommand{\C}{\ensuremath{\field{C}}}
\newcommand{\F}{\ensuremath{\field{F}}}
\newcommand{\U}{\ensuremath{\field{U}}}
\newcommand{\K}{\ensuremath{\field{K}}}
\newcommand{\limcur}{\ensuremath{\text{\cursive{l}}}}
\newcommand{\sfrak}{\ensuremath{\mathfrak{S}}}
\newcommand{\sfrakn}{\ensuremath{\mathfrak{S}_n}}
\newcommand{\bu}{\noindent\ensuremath{\bullet}}
\newcommand{\llbr}{\ensuremath{\llbracket}}
\newcommand{\rrbr}{\ensuremath{\rrbracket}}
\newcommand{\minus}{\ensuremath{\backslash}}
\newcommand{\eps}{\ensuremath{\varepsilon}}
\newcommand{\ssi}{si et seulement si}
\newcommand{\implique}{\Rightarrow}
\newcommand{\pgcd}{\text{pgcd\,}}
\newcommand{\ppcm}{\text{ppcm\,}}
\newcommand{\id}{\text{Id}}
\newcommand{\norm}[1]{\ensuremath{\Vert #1\Vert}}
\newcommand{\ch}{\mathop{\mathrm{ch}}\nolimits}
\newcommand{\sh}{\mathop{\mathrm{sh}}\nolimits}
\newcommand{\Vect}{\mathop{\mathrm{Vect}}\nolimits}
\renewcommand{\tanh}{\mathop{\mathrm{th}}\nolimits}
\renewcommand{\geq}{\geqslant}
\renewcommand{\leq}{\leqslant}
\newcommand{\tq}{\text{ tq }}
\newcommand{\chbox}{\LARGE\Checkedbox\normalsize\ }
\newcommand{\crbox}{\LARGE\Crossedbox\normalsize\ }
\newcommand{\Arcsin}{\mathop{\mathrm{Arcsin}}\nolimits}
\newcommand{\Arccos}{\mathop{\mathrm{Arccos}}\nolimits}
\newcommand{\Arctan}{\mathop{\mathrm{Arctan}}\nolimits}
\newcommand{\Argsh}{\mathop{\mathrm{Argsh}}\nolimits}
\newcommand{\Argch}{\mathop{\mathrm{Argch}}\nolimits}
\newcommand{\Argth}{\mathop{\mathrm{Argth}}\nolimits}
\newcommand{\abs}[1]{\left| #1 \right|}
\newcommand{\tend}{\ensuremath{\underset{n\to +\infty}{\longrightarrow}}}
\newcommand{\inv}{\ensuremath{^{-1}}}
\newcommand{\enx}[1]{\ensuremath{\underset{x\to #1}{=}}}
\newcommand{\simm}{\ensuremath{\underset{n\to +\infty}{\sim}}}
\newcommand{\simx}[1]{\ensuremath{\underset{x\to #1}{\sim}}}
\newcommand{\conj}[1]{\ensuremath{\overline{#1}}}






\begin{document}

\begin{center}
\Large\bf Cahier de textes de mathématiques en MPSI 2 :
\end{center}
\vspace{1cm}
\vspace{.4cm}\\

\noindent\textbf{Jeudi 22 septembre 2016}\\
\bu\ Cours sur les calculs algébriques : partie 3 terminée, partie 4 jusqu'à la définition 4.2.3.\\
\bu\ Feuille n°2 terminée. \vspace{.4cm}\\

\noindent\textbf{Mercredi 21 septembre 2016}\\
\bu\ DM n°3 distribué (à rendre le 29 septembre). \\
\bu\ Cours sur les calculs algébriques : partie 1 et 2 terminées, partie 3 jusqu'au corollaire 3.0.6.\\
\bu\ Pour la séance suivante : réfléchir à l'hypothèse de récurrence à écrire pour montrer le corollaire 3.0.6. et comprendre pourquoi les récurrences proposées dans la partie 4.9 du chapitre 2 sont fausses. \vspace{.4cm}\\

\noindent\textbf{Lundi 19 septembre 2016}\\
\bu\ Interrogation écrite n°3.\\
\bu\ Cours de logique (II) : terminé.\\
\bu\ Cours sur les calculs algébriques : partie 1 jusqu'à l'exemple 1.0.6.\vspace{.4cm}\\

\noindent\textbf{Vendredi 16 septembre 2016}\\
\bu\ Cours de logique (II) : parties 2 et 3 terminées.\\
\bu\ Pour la séance suivante : exercices 3.2.5. et 3.4.1.\\
\bu\ DS n°1 : nombres complexes.\vspace{.4cm}\\

\noindent\textbf{Jeudi 15 septembre 2016}\\
\bu\ Cours de logique (II) : partie 2.3 terminée.\\
\bu\ Feuille n°1 : terminée.\vspace{.4cm}\\

\noindent\textbf{Mercredi 14 septembre 2016}\\
\bu\ Cours de logique (II) : parties 1, 2.1 et 2.2 traitées.\\
\bu\ Pour la séance suivante : travailler l'exercice n°2.2.5.\\
\bu\ Feuille n°1 : exercices 15, 18, 19 (question 1) corrigés.\vspace{.4cm}\\

\noindent\textbf{Lundi 12 septembre 2016}\\
\bu\ Interrogation écrite n°2.\\
\bu\ Cours sur les nombres complexes (I) terminé.\\
\bu\ Pour la séance suivante : travailler l'exemple n° 5.3.8.\\
\bu\ Feuille n°1 : exercices 7 à 14, exercice 15 débuté, exercice 16 (question 1).\\
\bu\ La feuille d'exercice sera corrigée jeudi, il convient donc de chercher ces exercices.\vspace{.4cm}\\

\noindent\textbf{Vendredi 09 septembre 2016}\\
\bu\ Cours sur les nombres complexes (I) : partie 4 terminée, partie 5 jusqu'à la définition 5.3.1 (similitudes et isométries).\\
\bu\ Feuille n°1 : exercice 6.\vspace{.4cm}\\

\noindent\textbf{Jeudi 08 septembre 2016}\\
\bu\ Cours sur les nombres complexes (I) : partie 3 terminée, partie 4 jusqu'à la partie 4.2.c (factorisation de sommes).\\
\bu\ DM n°2 (à rendre le 15 septembre). \\
\bu\ Feuille n°1 : exercices 3 à 5. Pour la séance suivante : préparer l'exercice 6.\\
\bu\ Pour la séance suivante : écrire la sixième ligne du triangle de Pascal et développer $(a+b)^6$.\vspace{.4cm}\\

\noindent\textbf{Mercredi 07 septembre 2016}\\
\bu\ Cours sur les nombres complexes (I) : partie 2 terminé, partie 3 jusqu'au début du théorème 3.2.1 (forme canonique et racines).\\
\bu\ Pour la séance suivante : Factoriser $z^4-1$.\vspace{.4cm}\\

\noindent\textbf{Lundi 05 septembre 2016}\\
\bu\ Interrogation écrite n°1.\\
\bu\ Cours sur les nombres complexes (I) : partie 2 jusqu'à la remarque 2.3.4.\\
\bu\ Feuille n°1 : exercices 1 et 2.\\
\bu\ Pour la séance suivante : expliciter $\mathbf{U}_7$ et déterminer les racines 5-ièmes de $42 e^{i\pi/3}$\vspace{.4cm}\\

\noindent\textbf{Vendredi 02 septembre 2016}\\
\bu\ Cours sur les nombres complexes (I) : partie 1 terminée, partie 2 jusqu'à la définition 2.1.3.\\
\bu\ Pour la séance de TD du lundi : préparer les exercices n° 1 et 2 de la feuille 1. \vspace{.4cm}\\

\noindent\textbf{Jeudi 01 septembre 2016}\\
\bu\ Journée de rentrée ; distribution des feuilles de TD, des formulaires, des
chapitres I et II.  \\
\bu\ DM n° 1 (à
rendre le 9 septembre). \\
\bu\ Cours sur les nombres complexes (I) : jusqu'au théorème 1.4.3. 
\label{end}
\end{document}


