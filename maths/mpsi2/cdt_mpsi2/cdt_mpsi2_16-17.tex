\documentclass[12pt,a4paper]{article}

\textheight=25cm
\topmargin=-50pt
\textwidth= 17cm
\oddsidemargin=-.5cm


\usepackage{enumerate}
\usepackage{stmaryrd} %\sslash chapitre 19 au moins (parallèle)
\usepackage{amsfonts}
\usepackage{fancybox}
\usepackage{color}
\usepackage{eurosym}
\usepackage{amssymb}
\usepackage[T1]{fontenc}
\usepackage{amsmath}
\usepackage{theorem}
\usepackage[french]{babel}
\usepackage[utf8]{inputenc}
\usepackage{latexsym}
\usepackage{amscd}
\usepackage{indentfirst}
\usepackage[dvips]{graphicx}
\usepackage{textcomp}
\usepackage{mathrsfs}

\newcommand{\G}{\ensuremath{\mcal{G}}}
\newcommand{\cG}{\ensuremath{\mcal{G}}}
\newcommand{\cL}{\ensuremath{\mcal{L}}}
\newcommand{\lin}{\ensuremath{\mcal{L}}}
\newcommand{\rmdim}{\dim}
\newcommand{\dprod}{\displaystyle\prod}
\newcommand{\dsum}{\displaystyle\sum}
\newcommand{\intt}{\displaystyle\int}
\newcommand{\ml}{\ensuremath{\mcal{L}}}
\newcommand{\E}{\ensuremath{\mathbb{E}}}
\newcommand{\I}{\ensuremath{\mathcal{I}}}
\newcommand{\N}{\ensuremath{\field{N}}}
\newcommand{\Z}{\ensuremath{\field{Z}}}
\newcommand{\D}{\ensuremath{\field{D}}}
\newcommand{\M}{\ensuremath{\mcal{M}}}
\newcommand{\Q}{\ensuremath{\field{Q}}}
\newcommand{\R}{\ensuremath{\field{R}}}
\newcommand{\Rp}{\ensuremath{\field{R}_+}}
\newcommand{\Rpe}{\ensuremath{\field{R}_+^{\ast}}}
\newcommand{\Ret}{\ensuremath{\field{R}^{\ast}}}
\newcommand{\Rd}{\ensuremath{\field{R}^2}}
\newcommand{\Rt}{\ensuremath{\field{R}^3}}
\newcommand{\C}{\ensuremath{\field{C}}}
\newcommand{\F}{\ensuremath{\field{F}}}
\newcommand{\U}{\ensuremath{\field{U}}}
\newcommand{\K}{\ensuremath{\field{K}}}
\newcommand{\limcur}{\ensuremath{\text{\cursive{l}}}}
\newcommand{\sfrak}{\ensuremath{\mathfrak{S}}}
\newcommand{\sfrakn}{\ensuremath{\mathfrak{S}_n}}
\newcommand{\bu}{\noindent\ensuremath{\bullet}}
\newcommand{\llbr}{\ensuremath{\llbracket}}
\newcommand{\rrbr}{\ensuremath{\rrbracket}}
\newcommand{\minus}{\ensuremath{\backslash}}
\newcommand{\eps}{\ensuremath{\varepsilon}}
\newcommand{\ssi}{si et seulement si}
\newcommand{\implique}{\Rightarrow}
\newcommand{\pgcd}{\text{pgcd\,}}
\newcommand{\ppcm}{\text{ppcm\,}}
\newcommand{\id}{\text{Id}}
\newcommand{\norm}[1]{\ensuremath{\Vert #1\Vert}}
\newcommand{\ch}{\mathop{\mathrm{ch}}\nolimits}
\newcommand{\sh}{\mathop{\mathrm{sh}}\nolimits}
\newcommand{\Vect}{\mathop{\mathrm{Vect}}\nolimits}
\renewcommand{\tanh}{\mathop{\mathrm{th}}\nolimits}
\renewcommand{\geq}{\geqslant}
\renewcommand{\leq}{\leqslant}
\newcommand{\tq}{\text{ tq }}
\newcommand{\chbox}{\LARGE\Checkedbox\normalsize\ }
\newcommand{\crbox}{\LARGE\Crossedbox\normalsize\ }
\newcommand{\Arcsin}{\mathop{\mathrm{Arcsin}}\nolimits}
\newcommand{\Arccos}{\mathop{\mathrm{Arccos}}\nolimits}
\newcommand{\Arctan}{\mathop{\mathrm{Arctan}}\nolimits}
\newcommand{\Argsh}{\mathop{\mathrm{Argsh}}\nolimits}
\newcommand{\Argch}{\mathop{\mathrm{Argch}}\nolimits}
\newcommand{\Argth}{\mathop{\mathrm{Argth}}\nolimits}
\newcommand{\abs}[1]{\left| #1 \right|}
\newcommand{\tend}{\ensuremath{\underset{n\to +\infty}{\longrightarrow}}}
\newcommand{\inv}{\ensuremath{^{-1}}}
\newcommand{\enx}[1]{\ensuremath{\underset{x\to #1}{=}}}
\newcommand{\simm}{\ensuremath{\underset{n\to +\infty}{\sim}}}
\newcommand{\simx}[1]{\ensuremath{\underset{x\to #1}{\sim}}}
\newcommand{\conj}[1]{\ensuremath{\overline{#1}}}






\begin{document}

\begin{center}
\Large\bf CAHIER DE TEXTES DE MATHÉMATIQUES\\
MPSI 2 La Martinière Monplaisir\\ 2016-2017
\end{center}
\vspace{1cm}
\vspace{.4cm}\\

% \noindent\textbf{\bf Jeudi 23 mai 2015 \rm} (2 heures)\\
% \bu\ Remise de l'interrogation surprise n° 13.\\
% \bu\ Cours : 3 - Automorphismes orthogonaux (dans le plan).\\
% \bu\ Exercices : feuille n° 28, ex. 15 et 14 et feuille n° 292, ex.
%18.\vspace{.4cm}\\
% 
% \noindent\textbf{\bf Mercredi 22 mai 2015 \rm} (3 heures)\\
% \bu\ Distribution du DM n° 20 à rendre le 29 mai.\\
% \bu\ Remise des copies du DS n° 09, remarques au tableau.\\
% \bu\ Interrogation surprise n° 13.\\
% \bu\ Cours : 3 - Automorphismes orthogonaux (début).\\
% \bu\ Exercices : feuille n° 28, ex. 13.\vspace{.4cm}\\
% 
% \noindent\textbf{ \bf Vendredi 17 mai 2015 \rm}(2 heures)\\
% \bu\ Cours : 2 - Orthogonalité (fin).\\
% 
% \noindent\textbf{\bf Jeudi 16 mai 2015 \rm} (2 heures)\\
% \bu\ Cours : 2 - Orthogonalité (début).\\
% \bu\ Exercices : feuille n° 28, ex. 10 et 17.\vspace{.4cm}\\
% 
% \noindent\textbf{\bf Mercredi 15 mai 2015 \rm} (3 heures)\\
% \bu\ Distribution du DM n° 19 à rendre le 22 mai.\\
% \bu\ Cours : \bf Chapitre XXIX \rm : Espaces vectoriels euclidiens : 1 -
%Produits scalaires, normes et distances.\\
% \bu\ Exercices : feuille n° 28, ex. 8, 9, 11 et 12.\vspace{.4cm}\\
% 
% \noindent\textbf{ Lundi 13 mai 2015 \rm} (2$\times$2 heures + 3 heures)\\
% \bu\ Remise des copies du DM n° 18, remarques au tableau et distribution du
%corrigé.\\
% \bu\ Cours : 5 - Déterminant d'une matrice carrée ; 6 - Applications.\\
% \bu\ Exercices : feuille n° 28, ex. 1 à 7 et ex. 18.\vspace{.4cm}\\
% 
% \noindent\textbf{ Vendredi 10 mai 2015 \rm} (4 heures)\\
% $\bullet$\ DS n° 09.\vspace{.4cm}\\
% 
% \noindent\textbf{ \bf Vendredi 10 mai 2015 \rm}(2 heures)\\
% \bu\ Remise de l'interrogation surprise n° 12.\\
% \bu\ Cours : 3 - Déterminant d'une famille de vecteurs (fin) ; 4 - Déterminant
%d'un endomorphisme.\\
% \bu\ Exercices : feuille n° 27, ex. 9.\vspace{.4cm}\\
% 
% \noindent\textbf{ Lundi 06 mai 2015 \rm} (2$\times$2 heures + 3 heures)\\
% \bu\ Interrogation surprise n° 12.\\
% \bu\ Remise des copies du DM n° 17, remarques au tableau et distribution du
%corrigé.\\
% \bu\ Cours : 2 - Applications multinéaires ; 3 - Déterminant d'une famille de
%vecteurs (début).\\
% \bu\ Exercices : feuille n° 27, ex. 3, 4, 6, 7 et 8.\vspace{.4cm}\\
% 
% \noindent\textbf{ Vacances de printemps \rm}\\
% 
% \noindent\textbf{ \bf Vendredi 19 avril 2015 \rm}(2 heures)\\
% \bu\ Cours : 1 - Groupe symétrique (fin).\\
% \bu\ Exercices : feuille n° 27, ex. 1 et 2.\vspace{.4cm}\\
% 
% \noindent\textbf{\bf Jeudi 18 avril 2015 \rm} (2 heures)\\
% \bu\ Cours : \bf Chapitre XXVIII \rm : Déterminants : 1 - Groupe symétrique
%(début).\vspace{.4cm}\\
% 
% \noindent\textbf{\bf Mercredi 17 avril 2015 \rm} (3 heures)\\
% \bu\ Remise de l'interrogation n° 11.\\
% \bu\ Distribution du DM n° 18 à rendre le 10 mai, et ramassage du DM n° 17.\\
% \bu\ Cours : 2 - Étude locale d'une fraction rationnelle (fin) ; 3 -
%Application au calcul intégral.\\
% \bu\ Exercices : feuille n° 26, ex. 22 à 24.\vspace{.4cm}\\
% 
% \noindent\textbf{ Lundi 15 avril 2015 \rm} (5 heures)\\
% \bu\ Distribution du cours du chapitre XXIX.\\
% \bu\ Remise de l'interrogation n° 10.\\
% \bu\ Interrogation surprise n° 11.\\
% \bu\ Cours : 6 - Rang d'une matrice (fin) ; 7 - Systèmes linéaires.\\ : 1 - Le corps des fractions
%rationnelles ;\\
% 2 - Étude locale d'une fraction rationnelle (début).\\
% \bu\ Exercices : feuille n° 26, ex. 18 à 21 (partiellement).\vspace{.4cm}\\
% 
% \noindent\textbf{ \bf Vendredi 12 avril 2015 \rm}(2 heures)\\
% \bu\ Cours : 6 - Rang d'une matrice (fin).\\
% \bu\ Exercices : feuille n° 26, ex. 17.\vspace{.4cm}\\
% 
% \noindent\textbf{\bf Jeudi 11 avril 2015 \rm} (2 heures)\\
% \bu\ Cours : 6 - Rang d'une matrice (début).\\
% \bu\ Exercices : feuille n° 26, ex. 12 à 16.\vspace{.4cm}\\
% 
% \noindent\textbf{\bf Mercredi 10 avril 2015 \rm} (3 heures)\\
% \bu\ Distribution du DM n° 17 à rendre le 10 avril.\\
% \bu\ Interrogation surprise n° 10.\\
% \bu\ Cours : 5 - Opérations élémentaires sur les matrtices.\\
% \bu\ Exercices : feuille n° 26, ex. 9 et 11.\vspace{.4cm}\\
% 
% \noindent\textbf{ Lundi 08 avril 2015 \rm} (2$\times$2 heures + 3 heures)\\
% \bu\ Cours : 4 - Matrices remarquables.\\
% \bu\ Exercices : feuille n° 26, ex. 2 à 8.\vspace{.4cm}\\
% 
% \noindent\textbf{ \bf Vendredi 05 avril 2015 \rm}(4 heures)\\
% \bu\ Devoir surveillé n° 8.\\
% \bu\ Remise des copies du DM n° 16 et de l'interro n° 9 et distribution du
%corrigé.\vspace{.4cm}\\
% 
% \noindent\textbf{ \bf Vendredi 05 avril 2015 \rm}(2 heures)\\
% \bu\ Cours : 3 - Matrices, familles de vecteurs et applications linéaires
%(fin).\\
% \bu\ Exercices : feuille n° 25, ex. 15.\vspace{.4cm}\\
% 
% \noindent\textbf{\bf Jeudi 04 avril 2015 \rm} (2 heures)\\
% \bu\ Exercices : feuille n° 25, ex. 9 à 11 et 13 et 14.\vspace{.4cm}\\
% 
% \noindent\textbf{\bf Mercredi 03 avril 2015 \rm} (3 heures)\\
% \bu\ Interrogation n° 9.\\
% \bu\ Remise des copies du DM n° 15, remarques au tableau et distribution du
%corrigé, et ramassage du DM n° 16.\\
% \bu\ Distribution du cours des chapitres XXVII et XXVIII.\\
% \bu\ Cours : 3 - Matrices, familles de vecteurs et applications linéaires
%(suite).\\
% \bu\ Exercices : feuille n° 25, ex. 1 et 5.\\
% $\bullet$\ À faire pour jeudi 04/04 : feuille n° 25, ex. 14.\vspace{.4cm}\\
% 
% \noindent\textbf{ \bf Vendredi 29 mars 2015 \rm}(2 heures)\\
% $\bullet$\ Cours :  2 - Le produit matriciel (fin) ; 3 - Matrices, familles de
%vecteurs et applications linéaires
% (début).\\
% \bu\ Exercices : feuille n° 25, ex. 2.\\
% $\bullet$\ À faire pour mercredi 03/04 : feuille n° 25, ex. 5.\vspace{.4cm}\\
% 
% \noindent\textbf{\bf Jeudi 28 mars 2015 \rm} (2 heures)\\
% \bf Chapitre XXVI \rm : Matrices ; 1 - L'espace vectoriel $\mcal M_{n,p}(\K)$
%; 2 - Le produit matriciel (début).\\
% \bu\ Exercices : feuille n° 24, ex. 14 à 18.\vspace{.4cm}\\
% 
% \noindent\textbf{\bf Mercredi 27 mars 2015 \rm} (3 heures)\\
% \bu\ Remise des copies du DM n° 14, remarques au tableau et distribution du
%corrigé.\\
% \bu\ Distribution du DM n° 16 à rendre le 03 avril, et ramassage du DM n°
%15.\\
% \bu\ Cours : 4 - Exemples et applications (fin).\\
% \bu\ Exercices : feuille n° 24, ex. 8, 12 et 13.\\
% $\bullet$\ À faire pour jeudi 28/03 : feuille n° 24, ex. 14.\vspace{.4cm}\\
% 
% \noindent\textbf{ Lundi 25 mars 2015 \rm} (2$\times$2 heures + 3 heures)\\
% $\bullet$\ Cours : 3 - Formules de Taylor-Young ; 4 - Exemples et applications
%(début).\\
% \bu\ Exercices : feuille n° 24, ex. 1 à 7 et 9 à 11.\\
% $\bullet$\ À faire pour mercredi 27/03 : feuille n° 24, ex. 8.\vspace{.4cm}\\
% 
% \noindent\textbf{ \bf Vendredi 22 mars 2015 \rm}(2 heures)\\
% \bu\ Cours : 2 - Opérations.\vspace{.4cm}\\
% 
% \noindent\textbf{\bf Jeudi 21 mars 2015 \rm} (2 heures)\\
% $\bullet$\ Cours : \bf Chapitre XXV \rm : Développements limités. 1 -
%Définitions.\\
% \bu\ Exercices : feuille n° 23, ex. 21 à 28.\vspace{.4cm}\\
% 
% \noindent\textbf{\bf Mercredi 20 mars 2015 \rm} (3 heures)\\
% \bu\ Distribution du cours des chapitres XXV et XXVI et du DM n° 15 à rendre
%le 27 mars, et ramassage du DM n° 14.\\
% \bu\ Remise de l'interrogation n° 7.\\
% \bu\ Cours : 4 - Formes linéaires et hyperplans.\\
% \bu\ Exercices : feuille n° 23, ex. 17 (fin), 18, 19 et 20.\vspace{.4cm}\\
% 
% \noindent\textbf{ Lundi 18 mars 2015 \rm} (2$\times$2 heures + 3 heures)\\
% \bu\ Interrogation n° 7.\\
% \bu\ Cours : 2 - Sev en dimension finie (fin) ; 3 - Applications linéaires en
%dimension finie.\\
% \bu\ Exercices : feuille n° 23, ex. 9, 14 à 16 et 17 (début).\vspace{.4cm}\\
% 
% \noindent\textbf{ \bf Vendredi 15 mars 2015 \rm}(2 heures)\\
% $\bullet$\ Cours : 2 - Sev en dimension finie (fin).\\
% \bu\ Exercices : feuille n° 23, ex. 5 à 8.\vspace{.4cm}\\
% 
% \noindent\textbf{\bf Jeudi 14 mars 2015 \rm} (2 heures)\\
% $\bullet$\ Cours : 1 - Notion de dimension (fin) ; 2 - Sev en dimension finie
%(début).\\
% \bu\ Exercices : feuille n° 23, ex. 3.\\
% $\bullet$\ À faire pour vendredi 15/03 : feuille n° 23, ex. 5.\vspace{.4cm}\\
% 
% \noindent\textbf{\bf Mercredi 13 mars 2015 \rm} (3 heures)\\
% \bu\ Distribution du DM n° 14 (à rendre le 20 mars).\\
% $\bullet$\ Cours : \bf Chapitre XXIV \rm : Espaces vectoriels de dimension
%finie : 1 - Notion de dimension
% (début).\\
% \bu\ Exercices : feuille n° 23, ex. 4.\\
% $\bullet$\ À faire pour jeudi 14/03 : feuille n° 23, ex. 3.\vspace{.4cm}\\
% 
% \noindent\textbf{ Lundi 11 mars 2015 \rm} (2$\times$2 heures + 3 heures)\\
% \bu\ Remise du DM n° 13 et du DS n° 7, remarques au tableau.\\
% $\bullet$\ Cours : 4 - Formules de Taylor ; 5 - Fonctions à valeurs complexes
%; 6 - Méthodes d'approximation ; 7 -
% Annexes.\\
% \bu\ Exercices : feuille n° 22, ex. 13, 14, 15, 17, 18 et 19 et feuille n° 23,
%ex. 1.\\
% $\bullet$\ À faire pour mercredi 13/03 : feuille n° 23, ex. 4.\vspace{.4cm}\\
% 
% \noindent\textbf{ Vacances d'hiver \rm}\\
% 
% \noindent\textbf{ \bf Vendredi 22 février 2015 \rm}(4 heures)\\
% Devoir surveillé n° 7.\vspace{.4cm}\\
% 
% \noindent\textbf{ \bf Vendredi 22 février 2015 \rm}(2 heures)\\
% $\bullet$\ Cours : 3 - Méthodes de calcul.\\
% \bu\ Exercices : feuille n° 22, ex. 7, 12, 20 et 21.\vspace{.4cm}\\
% 
% \noindent\textbf{\bf Jeudi 21 février 2015 \rm} (2 heures)\\
% $\bullet$\ Cours : \bf Chapitre XXIII \rm : Intégration : 1 - Construction de
%l'intégrale ; 2 - le théorème
% fondamental
% de l'analyse.\\
% \bu\ Exercices : feuille n° 22, ex. 6 et 10.\vspace{.4cm}\\
% 
% \noindent\textbf{\bf Mercredi 20 février 2015 \rm} (3 heures)\\
% \bu\ Ramassage du DM n° 13 et remise du DM n° 12.\\
% $\bullet$\ Cours : 4 - PGCD, PPCM et polynômes irréductibles (fin).\\
% \bu\ Exercices : feuille n° 22, ex. 2, 8, 16 et 5.\vspace{.4cm}\\
% 
% \noindent\textbf{ Lundi 18 février 2015 \rm} (2$\times$2 heures + 3 heures)\\
% \bu\ Remise de l'interrogation n° 6.\\
% $\bullet$\ Cours : 3 - Dérivation des polynômes ; 4 - PGCD, PPCM et polynômes
%irréductibles (début).\\
% \bu\ Exercices : feuille n° 21, ex. 17 à 19 et feuille n° 22, ex. 1, 3 et
%4.\vspace{.4cm}\\
% 
% \noindent\textbf{ \bf Vendredi 15 février 2015 \rm}(4 heures)\\
% $\bullet$\ Cours : 2 - Décomposition (fin).\\
% \bu\ Exercices : feuille n° 21, ex. 6 (fin), 11 à 14 et 16.\\
% $\bullet$\ À faire pour lundi 18/02 : feuille n° 21, ex. 15.\vspace{.4cm}\\
% 
% \noindent\textbf{\bf Jeudi 14 février 2015 \rm} (2 heures)\\
% $\bullet$\ Cours : 2 - Décomposition (début).\\
% \bu\ Interrogation n° 6.\\
% \bu\ Exercices : feuille n° 21, ex. 9, 10 et 6 (début).\\
% $\bullet$\ À faire pour vendredi 15/02 : feuille n° 21, ex. 6 à
%finir.\vspace{.4cm}\\
% 
% \noindent\textbf{\bf Mercredi 13 février 2015 \rm} (3 heures)\\
% \bu\ Distribution du DM n° 13 (à rendre le 20 février) et ramassage du DM n°
%12.\\
% $\bullet$\ Cours : 1 - \KX\, définitions et résultats algébriques (fin).\\
% \bu\ Exercices : feuille n° 21, ex. 4, 5, 7 et 8.\vspace{.4cm}\\
% 
% \noindent\textbf{ Lundi 11 février 2015 \rm} (2$\times$2 heures + 3 heures)\\
% $\bullet$\ Cours : 6 - Méthode de Newton - Raphson.\\
% $\bullet$\ \bf Chapitre XXII \rm : Polynômes : 1 - \KX\, définitions et
%résultats algébriques (début).\\
% \bu\ Exercices : feuille n° 20, ex. 6 (fin) et feuille n° 21, ex. 1 à 3.\\
% $\bullet$\ À faire pour mercredi 13/02 : feuille n° 21, ex. 4.\vspace{.4cm}\\
% 
% \noindent\textbf{\bf Jeudi 07 février 2015 \rm} (2 heures)\\
% $\bullet$\ Cours : 5 - Fonctions convexes.\\
% \bu\ Exercices : feuille n° 20, ex. 5 (fin) et 6 (début).\\
% $\bullet$\ À faire pour lundi 11/02 : feuille n° 20, ex. 6.\vspace{.4cm}\\
% 
% \noindent\textbf{\bf Mercredi 06 février 2015 \rm} (3 heures)\\
% \bu\ Distribution du cours du chapitre XXIII et du DM n° 12 (à rendre le 13
%février).\\
% $\bullet$\ Cours : 2 - Les grands théorèmes ; 3 - Extension au cas des
%fonctions complexes ; 4 - Suites récurrentes.\\
% \bu\ Exercices : feuille n° 20, ex. 5 (début).\vspace{.4cm}\\
% 
% \noindent\textbf{ Lundi 04 février 2015 \rm} (2$\times$2 heures + 3 heures)\\
% \es\ \bf Chapitre XXI \rm : Dérivabilité : 1 - Définitions et premières
%propriétés.\\
% \bu\ Exercices : feuille n° 19, ex. 22 à 26 et feuille n° 20, ex. 1 à 4.\\
% $\bullet$\ À faire pour mercredi 06/02 : feuille n° 20, ex. 5.\vspace{.4cm}\\
% 
% \noindent\textbf{ \bf Vendredi 01 février 2015 \rm}(4 heures)\\
% Devoir surveillé n° 6.\vspace{.4cm}\\
% 
% \noindent\textbf{ \bf Vendredi 01 février 2015 \rm}(2 heures)\\
% \bu\ Remise du DM n° 11.\\
% $\bullet$\ Cours : \bf Chapitre XX \rm : Comparaison de fonctions.\\
% \bu\ Exercices : feuille n° 19, ex. 17 à 21.\\
% $\bullet$\ À faire pour lundi 04/02 : feuille n° 19, ex. 22.\vspace{.4cm}\\
% 
% \noindent\textbf{ \bf Jeudi 31 janvier 2015 \rm}(3 heures)\\
% $\bullet$\ Cours : 6 - Endomorphismes particuliers.\\
% \bu\ Exercices : feuille n° 19, ex. 14 à 16.
% $\bullet$\ À faire pour vendredi 01/02 : feuille n° 19, ex. 17.\vspace{.4cm}\\
% 
% \noindent\textbf{\bf Mercredi 30 janvier 2015 \rm} (2 heures)\\
% \bu\ Ramassage du DM n° 11.\\
% $\bullet$\ Cours : 5 - Familles de vecteurs (fin).\\
% \bu\ Exercices : feuille n° 19, ex. 8 et 10 à 13.\\
% $\bullet$\ À faire pour jeudi 31/01 : feuille n° 19, ex. 14.\vspace{.4cm}\\
% 
% \noindent\textbf{ Lundi 28 janvier 2015 \rm} (2$\times$2 heures + 3 heures)\\
% $\bullet$\ Cours : 5 - Familles de vecteurs (suite).\\
% \bu\ Exercices : feuille n° 19, ex. 1, 2, et 4 à 10.\\
% $\bullet$\ À faire pour mercredi 30/01 : feuille n° 19, ex. 8 (fin) et 10
%(fin).\vspace{.4cm}\\
% 
% \noindent\textbf{ \bf Vendredi 25 janvier 2015 \rm}(2 heures)\\
% \bu\ Remise de l'interrogation n° 5.\\
% $\bullet$\ Cours : 4 - Applications linéaires (fin) ; 5 - Familles de vecteurs
%(début).\\
% \bu\ Exercices : feuille n° 18, ex. 18 et 19.\vspace{.4cm}\\
% 
% \noindent\textbf{ \bf Jeudi 24 janvier 2015 \rm}(3 heures)\\
% \bu\ Interrogation n° 5.\\
% $\bullet$\ Cours : 4 - Applications linéaires (début).\\
% \bu\ Exercices : feuille n° 18, ex. 14, 15 et 17.\vspace{.4cm}\\
% 
% \noindent\textbf{\bf Mercredi 23 janvier 2015 \rm} (2 heures)\\
% \bu\ Distribution du cours du chapitre XXI et du DM n° 11 (à rendre le 30
%janvier).\\
% \bu\ Ramassage du DM n° 10.\\
% $\bullet$\ Cours : 3 - Sous-espaces affines.\\
% \bu\ Exercices : feuille n° 18, ex. 11 à 13.\vspace{.4cm}\\
% 
% \noindent\textbf{ Lundi 21 janvier 2015 \rm} (2$\times$2 heures + 3 heures)\\
% \bu\ Remise des copies du DS n°5.\\
% $\bullet$\ Cours : 2 - Sev (fin).\\
% \bu\ Exercices : feuille n° 18, ex. 1 à 9.\\
% $\bullet$\ À faire pour mercredi 16/01 : feuille n° 18, ex. 9 (fin) et
%10.\vspace{.4cm}\\
% 
% \noindent\textbf{ \bf Vendredi 18 janvier 2015 \rm}(2 heures)\\
% $\bullet$\ Cours : 2 - Sev (début).\\
% \bu\ Exercices : feuille n° 17, ex. 16 et 17.\vspace{.4cm}\\
% 
% \noindent\textbf{ \bf Jeudi 17 janvier 2015 \rm}(3 heures)\\
% \bu\ Distribution du DM n° 10 à rendre le 23 janvier.\\
% $\bullet$\ Cours : 4 - Cas des fonctions à valeurs complexes ;\\
% \es\ \bf Chapitre XIX \rm : Espaces vectoriels : 1 - Ev.\\
% \bu\ Exercices : feuille n° 17, ex. 13 à 15.\vspace{.4cm}\\
% 
% \noindent\textbf{\bf Mercredi 16 janvier 2015 \rm} (2 heures)\\
% \bu\ Distribution du cours du chapitre XVIII et XX.\\
% $\bullet$\ Cours : 2 - Les grands théorèmes (fin) ; 3 - Fonctions
%lipschitziennes et uniforme continuité.\\
% \bu\ Exercices : feuille n° 17, ex. 8 à 12.\vspace{.4cm}\\
% 

% \bu\ DS n° 05 (4 heures).\\
% 
% \noindent\textbf{\bf Lundi 2 février 2015 \rm} (2$\times$2 heures + 3 heures)\\
% \bu\ Cours : Fractions rationnelles (fin).\\
% \bu\ Exercices : feuille n° 15, ex. 17, et feuille n° 16, ex. 1 et 6.\vspace{.4cm}\\
% 
% \noindent\textbf{ \bf Vendredi 30 janvier 2015 \rm}(2 heures)\\
% \bu\ Cours : Fractions rationnelles (suite).\\
% \bu\ Exercices : feuille n° 15, ex. 13, 14 et 16.\\
% \bu\ Devoir surveillé n° 6.\vspace{.4cm}\\
% 
% \noindent\textbf{ \bf Jeudi 29 janvier 2015 \rm}(3 heures)\\
% \bf Chapitre XXVII \rm : Fractions rationnelles (début).\\
% \bu\ Exercices : feuille n° 15, ex. 9 à 12.\vspace{.4cm}\\
% 
% \noindent\textbf{ Mercredi 28 janvier 2015 \rm} (2 heures)\\
% \bu\ Exercices : feuille n° 15, ex. 1 (fin), 4 (début), 6 (fin), 7 et 8.\vspace{.4cm}\\
% 
% \noindent\textbf{ Lundi 26 janvier 2015 \rm} (2$\times$2 heures + 3 heures)\\
% \bu\ Interrogation n° 12.\\
% \bu\ Cours : Dérivabilité (fin).\\
% \bu\ Exercices : feuille n° 14, ex. 15 (esquisse) et 16, et feuille n° 15, ex. 1 (début), 2, 3, 5 
% et 6 (début).\\
% $\bullet$\ À faire pour mercredi 28/01 : feuille n° 16, ex. 1 (fin) et
% 6 (fin).\vspace{.4cm}\\
% 
% \noindent\textbf{ \bf Vendredi 23 janvier 2015 \rm}(2 heures)\\
% \bu\ Cours : Dérivabilité (suite).\\
% \bu\ Exercices : feuille n° 14, ex. 19 et 20.\vspace{.4cm}\\
% 
% \noindent\textbf{ \bf Jeudi 22 janvier 2015 \rm}(3 heures)\\
% \bu\ Cours : Dérivabilité (suite).\\
% $\bullet$\ Distribution : DM n° 11 (à rendre le 29 janvier).\\
% \bu\ Exercices : feuille n° 14, ex. 17 et 18.\vspace{.4cm}\\
% 
% \noindent\textbf{ Mercredi 21 janvier 2015 \rm} (2 heures)\\
% $\bullet$\ Distribution : DM n° 11 (à rendre le 29 janvier).\\
% \bu\ Cours : Dérivabilité (suite).\\
% \bu\ Exercices : feuille n° 14, ex. 13 (fin) et 14.\vspace{.4cm}\\
% 
% \noindent\textbf{ Lundi 19 janvier 2015 \rm} (2$\times$2 heures + 3 heures)\\
% \bu\ Interrogation n° 11.\\
% \bu\ Cours : Polynômes (fin).\\
% \bf Chapitre XXV \rm : Dérivabilité (début).\\
% \bu\ Exercices : feuille n° 14, ex. 6, 8, 9, 11, 12 et 13 (début).\vspace{.4cm}\\
% 
% \noindent\textbf{ \bf Vendredi 16 janvier 2015 \rm}(2 heures)\\
% \bu\ Cours : Polynômes (suite).\\
% \bu\ Exercices : feuille n° 14, ex. 5, 7 et 10.\vspace{.4cm}\\
% 
% \noindent\textbf{ \bf Jeudi 15 janvier 2015 \rm}(3 heures)\\
% \bu\ Cours : Polynômes (suite).\\
% $\bullet$\ Distribution : DM n° 10 (à rendre le 22 janvier).\\
% \bu\ Exercices : feuille n° 14, ex. 2 à 4.\vspace{.4cm}\\
% 
% \noindent\textbf{ Mercredi 14 janvier 2015 \rm} (2 heures)\\
% \bu\ Cours : Polynômes (suite).\\
% \bu\ Exercices : feuille n° 14, ex. 1.\vspace{.4cm}\\
% 
% \noindent\textbf{ Lundi 12 janvier 2015 \rm} (2$\times$2 heures + 3 heures)\\
% \bu\ Cours : Polynômes (suite).\\
% \bu\ Exercices : feuille n° 13, ex. 3, 4, 6 (fin), 8 et 11 à 13 .\vspace{.4cm}\\
% 
% \noindent\textbf{ \bf Vendredi 09 janvier 2015 \rm}(2 heures)\\
% \bu\ Cours : Polynômes (suite).\\
% \bu\ Exercices : feuille n° 13, ex. 5 et 7.\\
% \bu\ Devoir surveillé n° 5.\vspace{.4cm}\\
% 
% \noindent\textbf{ \bf Jeudi 08 janvier 2015 \rm}(3 heures)\\
% \bu\ Cours : Polynômes (suite).\\
% \bu\ Exercices : feuille n° 12, ex. 5 et 7, et feuille n° 13, ex. 1 et 2.\vspace{.4cm}\\
% 
% \noindent\textbf{ Mercredi 07 janvier 2015 \rm} (2 heures)\\
% $\bullet$\ \bf Chapitre XXIV \rm : Polynômes (début).\\
% \bu\ Exercices : feuille n° 12, ex. 3 et 4.\vspace{.4cm}\\
% 
% \noindent\textbf{ Lundi 05 janvier 2015 \rm} (2$\times$2 heures + 3 heures)\\
% \bu\ Interrogation n° 10.\\
% \bu\ Cours : Continuité (fin).\\
% \bu\ Exercices : feuille n° 12, ex. 1, 2 et 6.\vspace{.4cm}\\
% 
%  
% \noindent\textbf{\bf Vacances de Noël \rm}\\
% 
% \noindent\textbf{ \bf Vendredi 12 décembre 2014 \rm}(2 heures)\\
% \bu\ Cours : Continuité (suite).\\
% \bu\ Exercices : feuille n° 11, ex. 9 et 11.\vspace{.4cm}\\
% 
% \noindent\textbf{ \bf Jeudi 18 décembre 2014 \rm}(3 heures)\\
% \bu\ Cours : Limites d'une fonction (fin).\\
% $\bullet$\ \bf Chapitre XXIII \rm : Continuité (début).\\
% \bu\ Exercices : feuille n° 11, ex. 8 et 10.\vspace{.4cm}\\
% 
% \noindent\textbf{\bf Mercredi 17 décembre 2014 \rm} (2 heures)\\
% \bu\ Cours : Limites d'une fonction (suite).\\
% \bu\ Exercices : feuille n° 11, ex. 6 et 7.\vspace{.4cm}\\
% 
% \noindent\textbf{ Lundi 15 décembre 2014 \rm} (2$\times$2 heures + 3 heures)\\
% \bu\ Interrogation n° 9.\\
% \bu\ Cours : Limites d'une fonction (suite).\\
% \bu\ Exercices : feuille n° 11, ex. 1 à 5.\vspace{.4cm}\\
% 
% \noindent\textbf{ \bf Vendredi 12 décembre 2014 \rm}(2 heures)\\
% \bu\ Cours : 
% \bf Chapitre XII \rm : Limites d'une fonction (début).\\
% \bu\ Exercices : feuille n° 10, ex. 18 et 19.\vspace{.4cm}\\
% 
% \noindent\textbf{ \bf Jeudi 11 décembre 2014 \rm}(3 heures)\\
% \bu\ Cours : Groupes, anneaux, corps (fin).\\
% $\bullet$\ Distribution : DM n° 9 (à rendre le 18 décembre).\\
% \bu\ Exercices : feuille n° 10, ex. 17.\vspace{.4cm}\\
% 
% \noindent\textbf{\bf Mercredi 10 décembre 2014 \rm} (2 heures)\\
% \bu\ Cours : Groupes, anneaux, corps (suite).\\
% \bu\ Exercices : feuille n° 10, ex. 15 et 16.\vspace{.4cm}\\
% 
% \noindent\textbf{ Lundi 08 décembre 2014 \rm} (2$\times$2 heures + 3 heures)\\
% \bu\ Interrogation n° 8.\\
% \bu\ Cours : 5 - Suites récurrentes ; 6 - Suites complexes.\\
% \bf Chapitre XI \rm : Groupes, anneaux, corps (début).\\
% \bu\ Exercices : feuille n° 10, ex. 12 à 15.\\
% \bu\ À faire pour mercedi 10 : feuille n° 10, ex. 15 (à finir)\vspace{.4cm}\\
% 
% \noindent\textbf{ \bf Vendredi 05 décembre 2014 \rm}(2 heures)\\
% $\bullet$\ Cours : 4 - Suites particulières (fin).\\
% \bu\ Exercices : feuille n° 10, ex. 3 et 4.\vspace{.4cm}\\
% 
% \noindent\textbf{ \bf Jeudi 04 décembre 2014 \rm}(3 heures)\\
% $\bullet$\ Cours : 4 - Suites particulières (début).\\
% $\bullet$\ Distribution : DM n° 8 (à rendre le 11 décembre).\\
% \bu\ Exercices : feuille n° 10, ex. 10 et 11.\\
% \bu\ À faire pour vendredi 05 : feuille n° 10, ex. 11 (à finir).\vspace{.4cm}\\
% 
% \noindent\textbf{\bf Mercredi 03 décembre 2014 \rm} (2 heures)\\
% \bu\ Cours : 3 - Résultats de convergence (fin).\\
% \bu\ Exercices : feuille n° 10, ex. 8 et 9.\\
% \bu\ À faire pour jeudi 04 : feuille n° 10, ex. 10.\vspace{.4cm}\\
% 
% \noindent\textbf{ Lundi 01 décembre 2014 \rm} (2$\times$2 heures + 3 heures)\\
% \bu\ Cours : 2.3 et 4 - Suites extraites et inégalités ; 3 - Résultats de convergence (début).\\
% \bu\ Exercices : feuille n° 10, ex. 1, 2 et 5 à 7.\vspace{.4cm}\\
% 
% \noindent\textbf{ \bf Vendredi 28 novembre 2014 \rm}(2 + 4 heures)\\
% \bu\ Cours : 2.2 - Opérations.\\
% \bu\ Exercices : feuille n° 9, ex. 17 et 
% 18.\\
% \bu\ Devoir surveillé n° 4.\vspace{.4cm}\\
% 
% \noindent\textbf{ \bf Jeudi 28 novembre 2014 \rm}(3 heures)\\
% \bu\ Cours : 2.1 - Définitions et 
% premières propriétés.\\
% \bu\ Exercices : feuille n° 9, ex. 12 à 16 et 
% 19.\vspace{.4cm}\\
%  
% \noindent\textbf{\bf Mercredi 26 novembre 2014 \rm} (2 heures)\\
% $\bullet$\ Cours : \bf Chapitre X :\rm Suites réelles et complexes : 1 - Vocabulaire.\\
% \bu\ Exercices : feuille n° 9, ex. 9 et 10.\\
% \bu\ À faire pour jeudi 27 : feuille n° 9, ex. 11.
% \vspace{.4cm}\\
%  
% \noindent\textbf{ Lundi 24 novembre 2014 \rm} (2$\times$2 heures + 3 heures)\\
% \bu\ Interrogation n° 7.\\
% \bu\ Cours : 3 - Nombres premiers.\\
% \bu\ Exercices : feuille n° 8, ex. 13 et feuille n° 9, ex. 1 à 
% 8.\\
% \bu\ À faire pour mercredi 26 : feuille n° 9, ex. 8 à finir, et ex. 10.\vspace{.4cm}\\
%  
% \noindent\textbf{ \bf Vendredi 21 novembre 2014 \rm}(2 heures)\\
% \bu\ Cours : 2.3 - PPCM.\\
% \bu\ Exercices : feuille n° 8, ex. 11 et 12.\\
% \bu\ À faire pour lundi 24 : feuille n° 8, ex. 13.\vspace{.4cm}\\
% 
% \noindent\textbf{ \bf Jeudi 20 novembre 2014 \rm}(3 heures)\\
% $\bullet$\ Distribution : DM n° 7 (à rendre le 27 novembre).\\
% \bu\ Cours : 2.1 - PGCD (fin).\\
% \bu\ Exercices : feuille n° 8, ex. 8 à 10.\vspace{.4cm}\\
%  
% \noindent\textbf{\bf Mercredi 19 novembre 2014 \rm} (2 heures)\\
% \bu\ Remise de l'interrogation surprise n° 5.\\
% \bu\ Cours : 2.1 - PGCD (début).\\
% \bu\ Exercices : feuille n° 8, ex. 6.\vspace{.4cm}\\
%  
% \noindent\textbf{\bf Lundi 17 novembre 2014 \rm} (2$\times$2 heures + 3
% heures)\\
% $\bullet$\ Cours : \bf Chapitre IX \rm : Arithmétique dans \Z\ : 1 - 
% Divisibilité.\\
% \bu\ Interrogation surprise n° 6.\\
% \bu\ Exercices : feuille n° 8, ex. 1 à 5.\vspace{.4cm}\\
%  
% \noindent\textbf{ \bf Vendredi 14 novembre 2014 \rm}(2 heures)\\
% \bu\ Cours : 5 - La relation d'ordre naturelle sur \R.\\
% \bu\ Exercices : feuille n° 7, ex. 9 et 10.\vspace{.4cm}\\
% 
% \noindent\textbf{ \bf Jeudi 13 novembre 2014 \rm}(3 heures)\\
% $\bullet$\ Distribution : DM n° 6 (à rendre le 20 novembre).\\
% \bu\ Cours : 4 - La relation d'ordre naturelle sur \N.\\
% \bu\ Exercices : feuille n° 7, ex. 6 (fin) et 8.\vspace{.4cm}\\
% 
% \noindent\textbf{\bf Mercredi 12 novembre 2014 \rm} (2 heures)\\
% $\bullet$\ Distribution : DM n° 6 (à rendre le 19 novembre).\\
% \bu\ Cours : 3 - Majorants, minorants et compagnie.\\
% \bu\ Exercices : feuille n° 7, ex. 6 (début).\vspace{.4cm}\\
%  
% \noindent\textbf{ Lundi 10 novembre 2014 \rm} (2$\times$2 heures + 3 heures)\\
% $\bullet$\ Cours : \bf Chapitre VIII \rm : Relations d'ordre : 1 - Relations 
% binaires ; 2 - Relations d'ordre.\\
% \bu\ Exercices : feuille n° 7, ex. 4 et 5.\vspace{.4cm}\\
% % 
% \noindent\textbf{ \bf Vendredi 07 novembre 2014 \rm}(2 + 4 heures)\\
% $\bullet$\ Cours : 4 - Équations différentielles linéaires du second ordre -
% équations avec second membre ; 5 - Un peu de physique.\\
% \bu\ Devoir surveillé n° 3.\vspace{.4cm}\\
% 
% \noindent\textbf{ \bf Jeudi 06 novembre 2014 \rm}(3 heures)\\
% $\bullet$\ Cours : 4 - Équations différentielles linéaires du second ordre -
% équations homogènes.\\
% \bu\ Exercices : feuille n° 7, ex. 2 et 3.\vspace{.4cm}\\
% 
% \noindent\textbf{\bf Mercredi 05 novembre 2014 \rm} (2 heures)\\
% $\bullet$\ Cours : 3.2 - Équations différentielles du premier ordre avec 
% second membre.\vspace{.4cm}\\
%  
% \noindent\textbf{ Lundi 03 novembre 2014 \rm} (2$\times$2 heures + 3 heures)\\
% $\bullet$\ Interrogation surprise n° 5.\\
% $\bullet$\ Cours : 2 - Généralités ; 3.1 - Équations différentielles 
% homogènes du premier ordre.\\
% \bu\ Exercices : feuille n° 6, ex. 15 et 16, et feuille n° 7, ex.
% 1.\vspace{.4cm}\\
% 
% \noindent\textbf{ Vacances de novembre \rm}\\
% 
% \noindent\textbf{ \bf Vendredi 17 octobre 2014 \rm}(2 heures)\\
% $\bullet$\ Cours : 1 - Résultats d'analyse relatifs aux fonctions à valeurs
% complexes d'une variable réelle, et intégration (fin).\\
% $\bullet$\ Exercices : feuille n° 6, ex. 9, 10, 11, 13 et 14.\vspace{.4cm}\\
% 
% \noindent\textbf{\bf Jeudi 16 octobre 2014 \rm} (3 heures)\\
% $\bullet$\ Distribution : DM n° 5 (à rendre le 06 novembre).\\
% $\bullet$\ Cours : 1 - Résultats d'analyse relatifs aux fonctions à valeurs
% complexes d'une 
% variable réelle, et intégration (suite).\\
% $\bullet$\ Exercices : feuille n° 6, ex. 7, 8 et 12.\vspace{.4cm}\\
% 
% \noindent\textbf{\bf Mercredi 15 octobre 2014 \rm} (2 heures)\\
% $\bullet$\ Distribution : DM n° 5 (à rendre le 06 novembre).\\
% \bf Chapitre VII \rm : Équations différentielles linéaires :\\
% \es 1 - Résultats d'analyse relatifs aux fonctions à valeurs complexes d'une 
% variable réelle, et intégration (début).\vspace{.4cm}\\
% 
% \noindent\textbf{ \bf Lundi 13 octobre 2014 \rm}(2$\times$2 heures + 3
% heures)\\
% $\bullet$\ Interrogation surprise n° 4.\\
% $\bullet$\ Cours :  6 - Fonctions trigonométriques inverses (fin) ; 7 -
% Fonctions hyperboliques.\\
% $\bullet$\ Exercices : feuille n° 5, ex. 8, et feuille n° 6, ex.
% 1 à 6.\vspace{.4cm}\\
% 
% \noindent\textbf{ \bf Vendredi 10 octobre 2014 \rm}(2 heures)\\
% $\bullet$\ Cours :  6 - Fonctions trigonométriques inverses (début).\\
% $\bullet$\ Exercices : feuille n° 5, ex. 9 à 12.\vspace{.4cm}\\
% 
% \noindent\textbf{\bf Jeudi 09 octobre 2014 \rm} (3 heures)\\
% $\bullet$\ Cours : 4 - Puissances entières, fonctions 
% polynomiales et rationnelles ; 5 - Fonctions exponentielle, logarithme et 
% exponentielles de base $a$.\\
% $\bullet$\ Exercices : feuille n° 5, ex. 5 et 6.\vspace{.4cm}\\
% 
% \noindent\textbf{\bf Mercredi 08 octobre 2014 \rm} (2 heures)\\
% $\bullet$\ Distribution : DM n° 4 (à rendre le 16 octobre).\\
% $\bullet$\ Cours : 2 - Théorèmes d'analyse admis ; 3 - Valeur absolue.\\
% $\bullet$\ Exercices : feuille n° 5, ex. 4.\vspace{.4cm}\\
%  
% \noindent\textbf{ \bf Lundi 06 octobre 2014 \rm}(2$\times$2 heures + 3
% heures)\\
% $\bullet$\ Cours : 5 - Images directe et réciproque.\\
% \es\ \bf Chapitre VI \rm : Fonctions usuelles : 1 - Vocabulaire usuel.\\
% $\bullet$\ Exercices : feuille n° 4, ex. 5, et feuille n° 5, ex. 1
% à 3.\vspace{.4cm}\\
%  
% \noindent\textbf{ \bf Vendredi 03 octobre 2014 \rm}(2 +4 heures)\\
% \bu\ Devoir surveillé n° 2.\vspace{.4cm}\\
% \bu\ Cours : 4 - Injectivité, surjectivité, bijectivité (fin).\\
% $\bullet$\ Exercices : feuille n° 4, ex. 7 et 8.\vspace{.4cm}\\
% 
% \noindent\textbf{ \bf Jeudi 02 octobre 2014 \rm}(3 heures)\\
% \bu\ Cours : 4 - Injectivité, surjectivité, bijectivité (début).\\
% $\bullet$\ Exercices : feuille n° 4, ex. 4 et 6.\vspace{.4cm}\\
% 
% \noindent\textbf{\bf Mercredi 01 octobre 2014 \rm} (2 heures)\\
% $\bullet$\ Cours : \bf Chapitre V \rm : Notion d'application. 1 - Vocabulaire ;
% 2 - Restriction et prolongement ; 3 - Composition.\\
% $\bullet$\ Exercices : feuille n° 4, ex. 1 à 3.\vspace{.4cm}\\
% 
% \noindent\textbf{ \bf Lundi 29 septembre 2014 \rm}(2$\times$2 heures + 3
% heures)\\
% $\bullet$\ Interrogation surprise n° 3.\\
% $\bullet$\ Cours : Théorie des ensembles (fin).\\
% $\bullet$\ Exercices : feuille n° 3, ex. 12, 14, 17, 19, 20 et
% 22.\vspace{.4cm}\\
% 
% \noindent\textbf{ \bf Vendredi 26 septembre 2014 \rm}(2 heures)\\
% $\bullet$\ Cours : \bf Chapitre IV  \rm: Théorie des ensembles (début).\\
% $\bullet$\ Exercices : feuille n° 3, ex. 21.\vspace{.4cm}\\
% 
% \noindent\textbf{ \bf Jeudi 25 septembre 2014 \rm}(3 heures)\\
% $\bullet$\ Cours : 5 - Systèmes linéaires et pivot de Gauss.\\
% $\bullet$\ Exercices : feuille n° 3, ex. 13, 16 et 18.\\
% $\bullet$\ À faire pour lundi 29 : feuille n° 3, ex. 12, 14 et
% 19.\vspace{.4cm}\\
% 
% \noindent\textbf{\bf Mercredi 24 septembre 2014 \rm} (2 heures)\\
% $\bullet$\ Distribution : DM n° 3 (à rendre le 01 octobre).\\
% $\bullet$\ Cours : 4 - Calcul matriciel élémentaire (fin).\\
% $\bullet$\ Exercices : feuille n° 3, ex. 10, 11 et 15.\vspace{.4cm}\\
%  
% \noindent\textbf{ \bf Lundi 22 septembre 2014 \rm}(2$\times$2 heures + 3
% heures)\\
% $\bullet$\ Interrogation surprise n° 2.\\
% $\bullet$\ Cours : 3 - Quelques formules (fin) ; 4 - Calcul matriciel
% élémentaire (début).\\
% $\bullet$\ Exercices : feuille n° 3, ex. 2 à 6, et 8.\\
% $\bullet$\ À faire pour mercredi 24 : feuille n° 3, ex. 10.\vspace{.4cm}\\
% 
% 
% \noindent\textbf{ \bf Vendredi 19 septembre 2014 \rm}(2 + 4 heures)\\
% $\bullet$\ Cours : 3 - Quelques formules (début).\\
% $\bullet$\ Exercices : feuille n° 2, ex. 9 et feuille n° 3, ex. 1.\\
% $\bullet$\ À faire pour lundi 22 : feuille n° 3, ex. 7.\vspace{.4cm}\\
% 
% \noindent\textbf{ \bf Jeudi 18 septembre 2014 \rm}(3 heures)\\
% $\bullet$\ Cours : \bf Chapitre III \rm : Un peu de calcul. 1 et 2 - Sommes
% et produits.\\
% $\bullet$\ Exercices : feuille n° 1, ex. 8 et 9.\\
% $\bullet$\ À faire pour vendredi 19 : feuille n° 3, ex. 1.\vspace{.4cm}\\
% 
% \noindent\textbf{ \bf Mercredi 17 septembre 2014 \rm}(2 heures)\\
% $\bullet$\ Distribution : DM n° 2 (à rendre le 24 septembre).\\
% $\bullet$\ Exercices : feuille n° 1, ex. 4, 5, 6 et 7.\\
% $\bullet$\ Cours :  ; 4 - Le raisonnement par récurrence.\\
% $\bullet$\ À faire pour jeudi 18 : feuille n° 1, ex. 8.\vspace{.4cm}\\
%  
% \noindent\textbf{ \bf Lundi 15 septembre 2014 \rm}(2$\times$2 heures + 3
% heures)\\
% $\bullet$\ Cours : \bf Chapitre I \rm : Quelques fondamentaux. 1 -
% Propositions ; 2 - Connecteurs logiques ; 3 - Quantificateurs.\\
% $\bullet$\ Exercices : feuille n° 2, ex. 13, 21, 24 et 25, et feuille n°
% 1, ex. 1.\\
% $\bullet$\ À faire pour mercredi 17 : feuille n° 1, ex. 5.\vspace{.4cm}\\
% 
% \noindent\textbf{ \bf Vendredi 12 septembre 2014 \rm}(2 + 4 heures)\\
% $\bullet$\ Exercices : feuille n° 2, ex. 18, 19, 20, 22 et 23.\\
% \bu\ Devoir surveillé n° 1.\vspace{.4cm}\\
% 
% \noindent\textbf{ \bf Jeudi 11 septembre 2014 \rm}(3 heures)\\
% $\bullet$\ Cours : 5 - Nombres complexes et géométrie plane (fin).\\
% $\bullet$\ Exercices : feuille n° 2, ex. 10, 11, 12, 14, 15, 16, et 17.
% 
% \noindent\textbf{ \bf Mercredi 10 septembre 2014 \rm}(2 heures)\\
% $\bullet$\ Cours : 5 - Nombres complexes et géométrie plane
% (début).\vspace{.4cm}\\
%  
% \noindent\textbf{ \bf Lundi 08 septembre 2014 \rm}(2$\times$2 heures + 3
% heures)\\
% $\bullet$\ Interrogation surprise n° 1.\\
% $\bullet$\ Cours : 4 - L'exponentielle complexe.\\
% $\bullet$\ Exercices : feuille n° 2, ex. 2 à 7.\\
% $\bullet$\ À faire pour mercredi 10 : feuille n° 2, ex. 10 et 13.\vspace{.4cm}\\
%  
% \noindent\textbf{ \bf Vendredi 05 septembre 2014 \rm}(2 heures)\\
% $\bullet$\ Cours : 3 - Équations du second degré.\\
% $\bullet$\ Exercices : feuille n° 2, ex. 1.\\
% 1.\\$\bullet$\ À faire pour lundi 08 : feuille n° 2, ex. 2, 4 et
% 5.\vspace{.4cm}\\
% 
% \noindent\textbf{ \bf Jeudi 04 septembre 2014 \rm}(3 heures)\\
% $\bullet$\ Cours : 2 - Le groupe \U\ des nombres complexes de module
% 1.\\$\bullet$\ À faire pour vendredi 05 : feuille n° 2, ex. 1.\vspace{.4cm}\\
% 
% \noindent\textbf{\bf Mercredi 03 septembre 2014 \rm} (2 heures)\\
% $\bullet$\ Cours : \bf Chapitre II \rm : Les nombres complexes. 1 - Construction
% de \C.\vspace{.4cm}\\

\noindent\textbf{Lundi 05 septembre 2016}\\
\bu\ Cours sur les nombres complexes (I) : partie 2 jusqu'à la remarque 2.3.4.\\
\bu\ Feuille n°1 : exercices 1 et 2.\\
\bu\ Pour la séance suivante : expliciter $\mathbf{U}_7$ et déterminer les racines 5-ièmes de $42 e^{i\pi/3}$\vspace{.4cm}\\

\noindent\textbf{Vendredi 02 septembre 2016}\\
\bu\ Cours sur les nombres complexes (I) : partie 1 terminée, partie 2 jusqu'à la définition 2.1.3.\\
\bu\ Pour la séance de TD du lundi : préparer les exercices n° 1 et 2 de la feuille 1. \vspace{.4cm}\\

\noindent\textbf{Jeudi 01 septembre 2016}\\
\bu\ Journée de rentrée ; distribution des feuilles de TD, des formulaires, des
chapitres I et II.  \\
\bu\ DM n° 1 (à
rendre le 9 septembre). \\
\bu\ Cours sur les nombres complexes (I) : jusqu'au théorème 1.4.3. 
\label{end}
\end{document}


