\documentclass[12pt,a4paper]{article}

\textheight=25cm
\topmargin=-50pt
\textwidth= 17cm
\oddsidemargin=-.5cm


\usepackage{enumerate}
\usepackage{stmaryrd} %\sslash chapitre 19 au moins (parallèle)
\usepackage{amsfonts}
\usepackage{fancybox}
\usepackage{color}
\usepackage{eurosym}
\usepackage{amssymb}
\usepackage[T1]{fontenc}
\usepackage{amsmath}
\usepackage{theorem}
\usepackage[french]{babel}
\usepackage[utf8]{inputenc}
\usepackage{latexsym}
\usepackage{amscd}
\usepackage{indentfirst}
\usepackage[dvips]{graphicx}
\usepackage{textcomp}
\usepackage{mathrsfs}

\newcommand{\G}{\ensuremath{\mcal{G}}}
\newcommand{\cG}{\ensuremath{\mcal{G}}}
\newcommand{\cL}{\ensuremath{\mcal{L}}}
\newcommand{\lin}{\ensuremath{\mcal{L}}}
\newcommand{\rmdim}{\dim}
\newcommand{\dprod}{\displaystyle\prod}
\newcommand{\dsum}{\displaystyle\sum}
\newcommand{\intt}{\displaystyle\int}
\newcommand{\ml}{\ensuremath{\mcal{L}}}
\newcommand{\E}{\ensuremath{\mathbb{E}}}
\newcommand{\I}{\ensuremath{\mathcal{I}}}
\newcommand{\N}{\ensuremath{\field{N}}}
\newcommand{\Z}{\ensuremath{\field{Z}}}
\newcommand{\D}{\ensuremath{\field{D}}}
\newcommand{\M}{\ensuremath{\mcal{M}}}
\newcommand{\Q}{\ensuremath{\field{Q}}}
\newcommand{\R}{\ensuremath{\field{R}}}
\newcommand{\Rp}{\ensuremath{\field{R}_+}}
\newcommand{\Rpe}{\ensuremath{\field{R}_+^{\ast}}}
\newcommand{\Ret}{\ensuremath{\field{R}^{\ast}}}
\newcommand{\Rd}{\ensuremath{\field{R}^2}}
\newcommand{\Rt}{\ensuremath{\field{R}^3}}
\newcommand{\C}{\ensuremath{\field{C}}}
\newcommand{\F}{\ensuremath{\field{F}}}
\newcommand{\U}{\ensuremath{\field{U}}}
\newcommand{\K}{\ensuremath{\field{K}}}
\newcommand{\limcur}{\ensuremath{\text{\cursive{l}}}}
\newcommand{\sfrak}{\ensuremath{\mathfrak{S}}}
\newcommand{\sfrakn}{\ensuremath{\mathfrak{S}_n}}
\newcommand{\bu}{\noindent\ensuremath{\bullet}}
\newcommand{\llbr}{\ensuremath{\llbracket}}
\newcommand{\rrbr}{\ensuremath{\rrbracket}}
\newcommand{\minus}{\ensuremath{\backslash}}
\newcommand{\eps}{\ensuremath{\varepsilon}}
\newcommand{\ssi}{si et seulement si}
\newcommand{\implique}{\Rightarrow}
\newcommand{\pgcd}{\text{pgcd\,}}
\newcommand{\ppcm}{\text{ppcm\,}}
\newcommand{\id}{\text{Id}}
\newcommand{\norm}[1]{\ensuremath{\Vert #1\Vert}}
\newcommand{\ch}{\mathop{\mathrm{ch}}\nolimits}
\newcommand{\sh}{\mathop{\mathrm{sh}}\nolimits}
\newcommand{\Vect}{\mathop{\mathrm{Vect}}\nolimits}
\renewcommand{\tanh}{\mathop{\mathrm{th}}\nolimits}
\renewcommand{\geq}{\geqslant}
\renewcommand{\leq}{\leqslant}
\newcommand{\tq}{\text{ tq }}
\newcommand{\chbox}{\LARGE\Checkedbox\normalsize\ }
\newcommand{\crbox}{\LARGE\Crossedbox\normalsize\ }
\newcommand{\Arcsin}{\mathop{\mathrm{Arcsin}}\nolimits}
\newcommand{\Arccos}{\mathop{\mathrm{Arccos}}\nolimits}
\newcommand{\Arctan}{\mathop{\mathrm{Arctan}}\nolimits}
\newcommand{\Argsh}{\mathop{\mathrm{Argsh}}\nolimits}
\newcommand{\Argch}{\mathop{\mathrm{Argch}}\nolimits}
\newcommand{\Argth}{\mathop{\mathrm{Argth}}\nolimits}
\newcommand{\abs}[1]{\left| #1 \right|}
\newcommand{\tend}{\ensuremath{\underset{n\to +\infty}{\longrightarrow}}}
\newcommand{\inv}{\ensuremath{^{-1}}}
\newcommand{\enx}[1]{\ensuremath{\underset{x\to #1}{=}}}
\newcommand{\simm}{\ensuremath{\underset{n\to +\infty}{\sim}}}
\newcommand{\simx}[1]{\ensuremath{\underset{x\to #1}{\sim}}}
\newcommand{\conj}[1]{\ensuremath{\overline{#1}}}






\begin{document}

\begin{center}
\Large\bf Cahier de textes de mathématiques en MPSI 2 :
\end{center}
\vspace{1cm}
\vspace{.4cm}\\

\noindent\textbf{Jeudi 8 juin 2017}\\
\bu\ Cours sur les séries numériques : jusqu'à l'exemple 2.6.\vspace{.4cm}\\

\noindent\textbf{Mercredi 7 juin 2017}\\
\bu\ Feuille d'exercices n°25 : exercices 1 à 4 corrigés.\\
\bu\ Cours sur les espaces euclidiens : terminé.\\
\bu\ Cours sur les séries numériques : jusqu'à l'exemple 1.4.\\
\bu\ Pour la séance suivante : étudier la nature et la somme d'une série géométrique de raison $q$.\vspace{.4cm}\\

\noindent\textbf{Vendredi 2 juin 2017}\\
\bu\ Cours sur les espaces euclidiens : parties 3.1, 3.2 et 3.3 terminées.\vspace{.4cm}\\

\noindent\textbf{Jeudi 1er juin 2017}\\
\bu\ Cours sur les espaces euclidiens : partie 2 terminée, partie 3 jusqu'à la proposition 3.1.10 (point ii).\vspace{.4cm}\\

\noindent\textbf{Mercredi 31 mai 2017}\\
\bu\ Cours sur les espaces euclidiens : partie 2.7 terminée, partie 2.8 jusqu'à l'exemple 2.8.7.\\
\bu\ Feuille d'exercices n°24 : terminée.\vspace{.4cm}\\

\noindent\textbf{Lundi 29 mai 2017}\\
\bu\ Interrogation écrite n°28.\\
\bu\ Cours sur les espaces euclidiens : partie 2.6 terminée.\\
\bu\ Feuille d'exercices n°24 : exercices n°10 à 14 corrigés.\vspace{.4cm}\\

\noindent\textbf{Mercredi 24 mai 2017}\\
\bu\ Cours sur les espaces euclidiens : jusqu'à la proposition 2.2.10.\\
\bu\ Feuille d'exercices n°24 : exercices n°8 et 9 corrigés.\vspace{.4cm}\\

\noindent\textbf{Lundi 22 mai 2017}\\
\bu\ Interrogation écrite n°27. \\
\bu\ Cours sur les espaces euclidiens : jusqu'à la proposition 1.0.17.\\
\bu\ Feuille d'exercices n°23 : terminée.\\
\bu\ Feuille d'exercices n°24 : exercices n°1 à 7 corrigés.\vspace{.4cm}\\

\noindent\textbf{Vendredi 19  mai 2017}\\
\bu\ Cours sur les déterminants : terminé.\vspace{.4cm}\\

\noindent\textbf{Jeudi 18  mai 2017}\\
\bu\ Cours sur les déterminants : parties 5.2 et 5.3 terminées.\\
\bu\ DM n°19, à rendre pour le 1er  juin.\vspace{.4cm}\\

\noindent\textbf{Mercredi 17  mai 2017}\\
\bu\ Cours sur les déterminants : parties 3, 4 et 5.1 terminées.\\
\bu\ Feuille d'exercices n°23 : exercices n°18 (matrice n°3) et 20 (système n°1).\vspace{.4cm}\\

\noindent\textbf{Lundi 15  mai 2017}\\
\bu\ Interrogation écrite n°26.\\
\bu\ Cours sur les déterminants : parties 3.1 et 3.2.a.\\
\bu\ Feuille d'exercices n°23 : exercices n°11, 13 (matrice n°3), 14 (matrices n° 5 et 7), 15, 17 (matrices n°2 et 3), 18 (matrice n°4) et 19.\vspace{.4cm}\\

\noindent\textbf{Vendredi 12  mai 2017}\\
\bu\ Cours sur les déterminants : parties 1 et 2 terminées.\\
\bu\ DS n°9 : probabilités.\vspace{.4cm}\\

\noindent\textbf{Jeudi 11 mai 2017}\\
\bu\ Cours sur les déterminants : partie 1.3 terminée.\\
\bu\ Feuille d'exercices n°23 : exercices 13 (matrice n°1), 14 (matrice n°2), 16 et 17 (matrice n°1) corrigés.\vspace{.4cm}\\

\noindent\textbf{Mercredi 10 mai 2017}\\
\bu\ Cours sur les matrices : terminé.\\
\bu\ Cours sur les déterminants : parties 1.1 et 1.2 terminées.\\
\bu\ Feuille d'exercices n°23 : exercices 8 à 10 corrigés.\vspace{.4cm}\\

\noindent\textbf{Vendredi 5 mai 2017}\\
\bu\ Cours sur les matrices : partie 7 terminée.\\
\bu\ Feuille d'exercices n°23 : exercices 4 à 7 corrigés.\vspace{.4cm}\\

\noindent\textbf{Jeudi 4 mai 2017}\\
\bu\ Cours sur les matrices : parties 5 et 6 terminées.\\
\bu\ Feuille d'exercices n°23 : exercices 1 à 3 corrigés.\vspace{.4cm}\\

\noindent\textbf{Mercredi 3 mai 2017}\\
\bu\ Cours sur les matrices : partie 5.1 terminée.\\
\bu\ Feuille d'exercices n°22 : terminée.\vspace{.4cm}\\

\noindent\textbf{Vendredi 14 avril 2017}\\
\bu\ Cours sur les matrices : jusqu'au théorème 5.1.4.\\
\bu\ Feuille d'exercices n°22 : exercice 9 corrigé.\vspace{.4cm}\\

\noindent\textbf{Jeudi 13 avril 2017}\\
\bu\ Cours sur les matrices : partie 3 terminée.\\
\bu\ Feuille d'exercices n°22 : exercice 8 corrigé.\vspace{.4cm}\\

\noindent\textbf{Mercredi 12 avril 2017}\\
\bu\ Cours sur les matrices : partie 2 terminée.\\
\bu\ Feuille d'exercices n°22 : exercice 7 corrigé.\vspace{.4cm}\\

\noindent\textbf{Lundi 10 avril 2017}\\
\bu\ Interrogation écrite n°25.\\
\bu\ Cours sur les matrices : partie 1 terminée, partie 2 jusqu'à la proposition 2.2.11.\\
\bu\ Feuille d'exercices n°22 : exercices 1 à 6 corrigés.\vspace{.4cm}\\

\noindent\textbf{Vendredi 7 avril 2017}\\
\bu\ Cours de probas : terminé.\\
\bu\ DS n°8 : ECE 2005, E3A 2014 MP, endomorphismes cycliques.\vspace{.4cm}\\

\noindent\textbf{Jeudi 6 avril 2017}\\
\bu\ Cours de probas : partie 2, jusqu'à la remarque 2.6.14.\vspace{.4cm}\\

\noindent\textbf{Mercredi 5 avril 2017}\\
\bu\ Feuille d'exercices n°21 : terminée.\\
\bu\ Cours de probas : partie 2, jusqu'à la remarque 2.5.19.\vspace{.4cm}\\

\noindent\textbf{Lundi 3 avril 2017}\\
\bu\ Interrogation écrite n°24.\\
\bu\ Feuille d'exercices n°21 : exercices 1 à 12 corrigés.\\
\bu\ Cours de probas : partie 2, jusqu'à l'exercice 2.5.4.\vspace{.4cm}\\

\noindent\textbf{Vendredi 31 mars 2017}\\
\bu\ Cours de probas : parties 2.1 et 2.2 terminées.\vspace{.4cm}\\

\noindent\textbf{Jeudi 30 mars 2017}\\
\bu\ Cours de probas : partie 1 terminée.\\
\bu\ Pour la séance suivante : chercher l'exercice 1.3.12. du cours de probas.\vspace{.4cm}\\

\noindent\textbf{Mercredi 29 mars 2017}\\
\bu\ Feuille d'exercices n°20 : terminée.\\
\bu\ Cours de probas : jusqu'à la proposition 1.3.5.\\
\bu\ Pour la séance suivante : chercher l'exercice 1.3.5. du cours de probas.\vspace{.4cm}\\

\noindent\textbf{Lundi 27 mars 2017}\\
\bu\ Interrogation écrite n°23.\\
\bu\ Feuille d'exercices n°19 : terminée. Les exercices 24 (cours) et 29 (très proche des autres) n'ont pas été corrigés.\\
\bu\ Cours sur les ev de dimension finie : terminé.\\
\bu\ Cours de probas : jusqu'à l'exemple 1.1.4.\vspace{.4cm}\\

\noindent\textbf{Vendredi 24 mars 2017}\\
\bu\ Cours sur les ev de dimension finie : partie 2 terminée, partie 3 jusqu'au théorème 3.2.8.\vspace{.4cm}\\

\noindent\textbf{Jeudi 23 mars 2017}\\
\bu\ Cours sur les ev de dimension finie : partie 2, jusqu'à l'exemple 2.3.5.\vspace{.4cm}\\

\noindent\textbf{Mercredi 22 mars 2017}\\
\bu\ DM n°15, à rendre le 30 mars.\\
\bu\ Feuille d'exercices n°19 : exercices 12, 16, 17 et 18 corrigés. Les exercices 14, 15 et 16 ne seront pas intégralement corrigés. \\
\bu\ Cours sur les ev de dimension finie : partie 1 terminée.\vspace{.4cm}\\

\noindent\textbf{Lundi 20 mars 2017}\\
\bu\ Interrogation écrite n°22.\\
\bu\ Feuille d'exercices n°19 : exercices 4 à 11 corrigés.\\
\bu\ Cours sur les ev de dimension finie : partie 1, jusqu'à l'exemple 1.4.5.\vspace{.4cm}\\

\noindent\textbf{Vendredi 17  mars 2017}\\
\bu\ Feuille d'exercices n°19 : exercice n°3 corrigé.\\
\bu\ Cours de dénombrement : terminé.\\
\bu\ DS n°07 : petites mines 2003 et polynômes de Bernoulli.\vspace{.4cm}\\

\noindent\textbf{Jeudi 16  mars 2017}\\
\bu\ Feuille d'exercices n°19 : exercice n°2 corrigé.\\
\bu\ Cours de dénombrement : partie 1 terminée, partie 2 jusqu'à la définition 2.3.2.\vspace{.4cm}\\

\noindent\textbf{Mercredi 15  mars 2017}\\
\bu\ Feuille d'exercices n°18 : terminée.\\
\bu\ Feuille d'exercices n°19 : exercice n°1 corrigé.\\
\bu\ Cours d'intégration : terminé. Exercice : déterminer un équivalent lorsque $n \to +\infty$ de $1+2^p+\dots+n^p$.\\
\bu\ Cours de dénombrement : partie 1 débutée, jusqu'au théorème 1.0.8.\vspace{.4cm}\\

\noindent\textbf{Lundi 13 mars 2017}\\
\bu\ Interrogation écrite n°21.\\
\bu\ Feuille d'exercices n°18 : exercices 10 à 18 corrigés.\\
\bu\ Cours d'intégration : parties 3, 4, 5 et 6 terminées. Partie 7 entamée (théorème de convergence des sommes de Riemann en cours de démonstration).\vspace{.4cm}\\

\noindent\textbf{Vendredi 10 mars 2017}\\
\bu\ Cours d'intégration : parties 2 et 3.1 terminées.\vspace{.4cm}\\

\noindent\textbf{Jeudi 9 mars 2017}\\
\bu\ Cours d'intégration : partie 2 jusqu'à la définition 2.2.5.\vspace{.4cm}\\

\noindent\textbf{Mercredi 8 mars 2017}\\
\bu\ Cours sur les espaces vectoriels : terminé.\\
\bu\ Cours d'intégration : partie 1 terminée, partie 2 jusqu'à la remarque 2.1.3.\\
\bu\ Feuille d'exercices n°18 : exercices 5 à 9 corrigés.\vspace{.4cm}\\

\noindent\textbf{Lundi 6 mars 2017}\\
\bu\ Interrogation écrite n°20.\\
\bu\ Cours sur les espaces vectoriels :  parties 5, 6.1 et 6.2 terminées.\\
\bu\ Feuille d'exerices n°17 : terminée.\\
\bu\ Feuille d'exercices n°18 : exercices 1 à 4 corrigés.\vspace{.4cm}\\

\noindent\textbf{Vendredi 17 février 2017}\\
\bu\ Cours sur les espaces vectoriels :  parties 5.4 et 5.5 terminées.\vspace{.4cm}\\

\noindent\textbf{Jeudi 16 février 2017}\\
\bu\ Cours sur les espaces vectoriels :  partie 5 jusqu'à la proposition 5.4.4.\\
\bu\ Feuille d'exercices n°17 : exercice 26 corrigé.\vspace{.4cm}\\

\noindent\textbf{Mercredi 15 février 2017}\\
\bu\ DM n°14, à rendre le 16 mars.\\
\bu\ Cours sur les espaces vectoriels : partie 4 terminée, partie 5 : proposition 5.1.1. énoncée.\\
\bu\ Feuille d'exercices n°17 : exercices 21, 22 (1), 23 (3), 24 (2) et 25 corrigés.\vspace{.4cm}\\

\noindent\textbf{Lundi 13 février 2017}\\
\bu\ Interrogation écrite n°19.\\
\bu\ Cours sur les espaces vectoriels : partie 3 terminée, parties 4.1 et 4.2 terminées.\\
\bu\ Feuille d'exercices n°17 : exercices 14 (1-2-3-12-14), 15 (1 à 4), 16, 17 (1 à 3), 18 (1 et 2), 19 (1 et 2), 20 corrigés.\vspace{.4cm}\\

\noindent\textbf{Vendredi 10 février 2017}\\
\bu\ Cours sur les espaces vectoriels : partie 2 terminée.\\
\bu\ DS n°06 : interpolation de Hermite, théorème de Mason et applications.\vspace{.4cm}\\

\noindent\textbf{Jeudi 9 février 2017}\\
\bu\ Cours sur les espaces vectoriels : partie 2 jusqu'à la définition 2.3.17.\vspace{.4cm}\\

\noindent\textbf{Mercredi 8 février 2017}\\
\bu\ Feuille d'exercices n°17 : exercices 9 à 14 corrigés.\\
\bu\ Cours sur les espaces vectoriels : partie 1 terminée, partie 2 jusqu'à la définition 2.3.4.\vspace{.4cm}\\

\noindent\textbf{Lundi 6 février 2017}\\
\bu\ Interrogation écrite n°18.\\
\bu\ Feuille d'exercices n°17 : exercices 1 à 8 corrigés.\\
\bu\ Cours d'analyse asymptotique : terminé.\\
\bu\ Cours sur les espaces vectoriels : parties 1.1 à 1.3 terminées.\vspace{.4cm}\\

\noindent\textbf{Vendredi 3 février 2017}\\
\bu\ Cours d'analyse asymptotique : partie 3.5.b. terminée.\vspace{.4cm}\\

\noindent\textbf{Jeudi 2 février 2017}\\
\bu\ Cours d'analyse asymptotique : partie 3.2. terminée.\\
\bu\ Feuille d'exercices n°16 : intégrale n°6 de l'exercice 6 calculée.\\
\bu\ Feuille d'exercices n°16 : exercices 1 à 4, 6 et 7 corrigés.\vspace{.4cm}\\

\noindent\textbf{Mercredi 1er février 2017}\\
\bu\ Cours d'analyse asymptotique : partie 3 avancée jusqu'à la partie 3.2.a.\\
\bu\ Feuille d'exercices n°16 : DES n°4 de l'exercice 5 calculée.\\
\bu\ DM n°13, à rendre le 9 février.\vspace{.4cm}\\

\noindent\textbf{Lundi 30 janvier 2017}\\
\bu\ Interrogation écrite n°17.\\
\bu\ Cours d'analyse asymptotique : partie 2 terminée.\\
\bu\ Feuille d'exercices n°15 : terminée.\\
\bu\ Feuille d'exercices n°16 : DES n°1 et 2 de l'exercice 5 calculées.\vspace{.4cm}\\

\noindent\textbf{Vendredi 27 janvier 2017}\\
\bu\ Cours d'analyse asymptotique : partie 1 terminée. \vspace{.4cm}\\

\noindent\textbf{Jeudi 26 janvier 2017}\\
\bu\ Cours sur les fractions rationnelles : terminé.\\
\bu\ Cours d'analyse asymptotique : partie 1.1. terminée.\\
\bu\ Feuille d'exercices n°15 : exercices 4 à 8 terminés.\\
\bu\ Pour la séance suivante : primitiver $t\mapsto \dfrac{2t}{t^2+t+1}$.\vspace{.4cm}\\


\noindent\textbf{Mercredi 25 janvier 2017}\\
\bu\ Cours sur les fractions rationnelles : partie 2 avancée jusqu'à la partie 2.5.\\
\bu\ DM n°12, à rendre pour le 2 février.\vspace{.4cm}\\

\noindent\textbf{Lundi 23 janvier 2017}\\
\bu\ Interrogation écrite n°16.\\
\bu\ Cours sur les fractions rationnelles : partie 1 terminée, partie 2 avancée jusqu'à la partie 2.4.\\
\bu\ Feuille d'exercices n°14 : terminée.\\
\bu\ Feuille d'exercices n°15 : exercices 1 à 3 corrigés.\vspace{.4cm}\\

\noindent\textbf{Vendredi 20 janvier 2017}\\
\bu\ Cours sur la dérivabilité : terminé..\\
\bu\ Cours sur les fractions rationnelles : partie 1 traitée jusqu'à la remarque 1.2.3.\\
\bu\ DS n°5 : entiers de Gauss, suite de Fibonacci et limite supérieure et inférieure d'une suite.\vspace{.4cm}\\

\noindent\textbf{Jeudi 19 janvier 2017}\\
\bu\ Cours sur la dérivabilité : partie 2.5. terminée.\\
\bu\ Feuille d'exercices n°14 : exercices 10 à 14 corrigés.\vspace{.4cm}\\

\noindent\textbf{Mercredi 18 janvier 2017}\\
\bu\ Cours sur la dérivabilité : partie 2 avancée jusqu'au théorème 2.3.3.\vspace{.4cm}\\

\noindent\textbf{Lundi 16 janvier 2017}\\
\bu\ Interrogation écrite n°15.\\
\bu\ Cours sur la dérivabilité : partie 1 terminée.\\
\bu\ Feuille d'exercices n°14 : exercices 1 à 9 corrigés. \vspace{.4cm}\\

\noindent\textbf{Vendredi 13 janvier 2017}\\
\bu\ Cours sur les polynômes : terminé.\\
\bu\ Pour la séance suivante : soit $P_0,\dots,P_n$ de degrés respectifs $0,\dots,n$, soit $\lambda_0,\dots,\lambda_n \in \K$. Montrer que si $\lambda_0 P_0 + \dots + \lambda_nP_n = 0$, alors $\lambda_0 = \dots = \lambda_n$. \vspace{.4cm}\\


\noindent\textbf{Jeudi 12 janvier 2017}\\
\bu\ Cours sur les polynômes : parties 4.1 à 4.3 terminées.\\
\bu\ Feuille d'exercices n°13 : terminée.\\
\bu\ DM n°11, à rendre le 19 janvier.\vspace{.4cm}\\

\noindent\textbf{Mercredi 11 janvier 2017}\\
\bu\ Cours sur les polynômes : partie 3 terminée, partie 4 jusqu'à la proposition 4.1.7.\vspace{.4cm}\\

\noindent\textbf{Lundi 9 janvier 2017}\\
\bu\ Interrogation écrite n°14.\\
\bu\ Cours sur les polynômes : partie 2 terminée, partie 3 jusqu'à la proposition 3.2.1.\\
\bu\ Feuille d'exercices n°13 : exercices 3 à 10 corrigés.\\
\bu\ Pour la séance suivante : factoriser $X^4+1$ sur $\R$.\vspace{.4cm}\\

\noindent\textbf{Vendredi 6 janvier 2017}\\
\bu\ Cours sur les polynômes : partie 1, parties 2.1 à 2.3 terminées.\\
\bu\ Pour la séance suivante : chercher l'exercice 2.3.7. du chapitre XIV.\vspace{.4cm}\\

\noindent\textbf{Jeudi 5 janvier 2017}\\
\bu\ Cours sur les polynômes : partie  1 jusqu'à la partie 1.6 (terminée).\\
\bu\ Feuille d'exercices n°12 : terminée.\\
\bu\ Feuille d'exercices n°13 : exercices 1 et 2 corrigés.\\
\bu\ DM n°10, à rendre le 12 janvier.\vspace{.4cm}\\

\noindent\textbf{Mercredi 4 janvier 2017}\\
\bu\ Cours sur les polynômes : partie  1 jusqu'au théorème 1.4.1.\vspace{.4cm}\\

\noindent\textbf{Vendredi 16 décembre 2016}\\
\bu\ Cours sur la continuité : terminé.\\
\bu\ Feuille d'exercices n°12 : exercices 2 et 3 corrigés.\vspace{.4cm}\\

\noindent\textbf{Jeudi 15 décembre 2016}\\
\bu\ Cours sur la continuité : partie 2.1 à 2.3 terminées.\\
\bu\ Feuille d'exercices n°11 terminée.\\
\bu\ Feuille d'exercices n°12 : limites 1 à 9 de l'exercices 1 corrigées.\vspace{.4cm}\\

\noindent\textbf{Mercredi 14 décembre 2016}\\
\bu\ Cours sur la continuité : partie 1 terminée. \vspace{.4cm}\\

\noindent\textbf{Lundi 12 décembre 2016}\\
\bu\ Interrogation écrite n°13.\\
\bu\ Cours sur les limites de fonctions : terminé.\\
\bu\ Feuille d'exercices n°11 : exercices 1 à 9 corrigés. \vspace{.4cm}\\

\noindent\textbf{Vendredi 9 décembre 2016}\\
\bu\ Cours sur les limites de fonctions : partie 2 terminée, partie 3 jusqu'au théorème 3.1.2. (énoncé).\\
\bu\ DS n°4 : bornes supérieures, arithmétique et un peu de suites. \vspace{.4cm}\\

\noindent\textbf{Jeudi 8 décembre 2016}\\
\bu\ Cours sur les groupes, anneaux et corps : terminé.\\
\bu\ Cours sur les limites de fonctions : partie 1 terminée, partie 2 jusq'à la définition 2.1.3.\\
\bu\ Feuille d'exercices n°10 : terminée.\\
\bu\ Pour la séance suivante : écrire les définitions quantifiées de limite d'une fonction en $-\infty$. \vspace{.4cm}\\

\noindent\textbf{Mercredi 7 décembre 2016}\\
\bu\ Cours sur les groupes, anneaux et corps :  partie 2 terminée, partie 3 jusqu'au théorème 3.0.27.\vspace{.4cm}\\

\noindent\textbf{Lundi 5 décembre 2016}\\
\bu\ Interrogation écrite n°12.\\
\bu\ Cours sur les groupes, anneaux et corps :  partie 1 terminée, partie 2 jusqu'à l'exemple 2.3.6.\\
\bu\ Feuille d'exercices n°10 : exercices 4 à 11 corrigés.\vspace{.4cm}\\

\noindent\textbf{Vendredi 2 décembre 2016}\\
\bu\ Cours sur les suites numériques : terminé.\vspace{.4cm}\\

\noindent\textbf{Jeudi 1er décembre 2016}\\
\bu\ Cours sur les suites numériques : parties 6.1 à 6.3 terminées. \\
\bu\ Feuille d'exercices n°10 : exercices 1 à 3 corrigés. \\
\bu\ DM n°9 distribué, à rendre pour le 8 décembre.\vspace{.4cm}\\

\noindent\textbf{Mercredi 30 novembre 2016}\\
\bu\ Cours sur les suites numériques : partie 4 et 5 terminées. \vspace{.4cm}\\

\noindent\textbf{Lundi 28 novembre 2016}\\
\bu\ Interrogation écrite n°11.\\
\bu\ Cours sur les suites numériques : partie 3 terminée.\\
\bu\ Feuille d'exercices n°9 : terminée.\\
\bu\ Feuille d'exercices n°10 : questions 1 à 3 de l'exercice n°1 corrigées. \vspace{.4cm}\\

\noindent\textbf{Vendredi 25 novembre 2016}\\
\bu\ Cours sur les suites numériques : partie 2 terminée.\vspace{.4cm}\\

\noindent\textbf{Jeudi 24 novembre 2016}\\
\bu\ Cours sur les suites numériques : parties 1 et 2.1 terminées.\\
\bu\ Feuille d'exercices n°9 : exercices 4 à 10 corrigés. \\
\bu\ DM n°8 distribué, à rendre pour le 1\up{er} décembre.\vspace{.4cm}\\

\noindent\textbf{Mercredi 23 novembre 2016}\\
\bu\ Cours d'arithmétique sur les entiers : terminé. \\
\bu\ Cours sur les suites numériques : partie 1 débutée, jusqu'à la définition 1.0.8.\vspace{.4cm}\\

\noindent\textbf{Lundi 21 novembre 2016}\\
\bu\ Interrogation écrite n°10.\\
\bu\ Cours d'arithmétique sur les entiers : partie 2 terminée, partie 3 traitée jusqu'au théorème 3.0.14. \\
\bu\ Feuille d'exercices n°8 terminée. \\
\bu\ Feuille d'exercices n°9 : exercices 1 à 3 corrigés.\vspace{.4cm}\\

\noindent\textbf{Vendredi 18 novembre 2016}\\
\bu\ Cours d'arithmétique sur les entiers : partie 2 traitée jusqu'au théorème 2.3.3. \\
\bu\ DS n°3 : fonctions  usuelles et équations différentielles.\vspace{.4cm}\\

\noindent\textbf{Jeudi 17 novembre 2016}\\
\bu\ Cours d'arithmétique sur les entiers : partie 2.1 traitée jusqu'à la proposition 2.1.7. \\
\bu\ Feuille d'exercices n°8 : exercices 4 à 7 corrigés.\vspace{.4cm}\\

\noindent\textbf{Mercredi 16 novembre 2016}\\
\bu\ Cours d'arithmétique sur les entiers : partie 1 terminée. \\
\bu\ Pour la séance suivante : traiter l'exemple 1.2.9 et l'exercice 1.2.10.\vspace{.4cm}\\

\noindent\textbf{Lundi 14 novembre 2016}\\
\bu\ Interrogation écrite n°9.\\
\bu\ Cours sur les relations : terminé. \\
\bu\ Feuille d'exercices n°7 : terminée. Feuille d'exercices n°8 : exercices 1 à 3 corrigés.\vspace{.4cm}\\

\noindent\textbf{Jeudi 10 novembre 2016}\\
\bu\ Cours sur les relations : parties 4 et 5 terminées. \\
\bu\ DM n°7 distribué, à rendre le jeudi 17 novembre. \\
\bu\ Feuille d'exercices n°7 : équations n°1,2 et 4 de l'exercice 4, équation n°2 de l'exercice 7 corrigées. Exercices 5 et 6 corrigés.\vspace{.4cm}\\

\noindent\textbf{Mercredi 9 novembre 2016}\\
\bu\ Cours sur les relations : partie 3 terminée, partie 4 traitée jusqu'à l'exemple 4.3.2. \\
\bu\ Feuille d'exercices n°7 : équation n°3 de l'exercice 4 corrigée.\vspace{.4cm}\\

\noindent\textbf{Lundi 7 novembre 2016}\\
\bu\ Interrogation écrite n°8.\\
\bu\ Cours sur les relations : parties 1 et 2 terminées. \\
\bu\ Feuille d'exercices n°7 : exercices 1, 2 et 3 corrigés.\\
\bu\ Pour la séance suivante : résoudre l'équation $y''+3y'+2y=(1+2t)\mathrm{e}^{-t}$.\vspace{.4cm}\\

\noindent\textbf{Vendredi 4 novembre 2016}\\
\bu\ Cours sur les équations différentielles : terminé (la partie 5 n'a pas été traitée, le théorème 4.2.8. n'a pas été démontré). \\
\bu\ Feuille d'exercices n°7 : intégrales n° 1,2 et 3 de l'exercice 1 corrigées.\vspace{.4cm}\\

\noindent\textbf{Jeudi 3 novembre 2016}\\
\bu\ Cours sur les équations différentielles : partie 4 commencée (structure des solutions complexes traitée). \\
\bu\ DM n°6 distribué, à rendre le jeudi 10 novembre. \\
\bu\ Feuille d'exercices n°6 : terminée.\vspace{.4cm}\\

\noindent\textbf{Mercredi 19 octobre 2016}\\
\bu\ Cours sur les équations différentielles : partie 3 terminée. \vspace{.4cm}\\

\noindent\textbf{Mardi 18 octobre 2016}\\
\bu\ Interrogation écrite n°7.\\
\bu\ Cours sur les équations différentielles : parties 1 et 2 terminées, partie 3 traitée jusqu'au théorème 3.1.1.\\
\bu\ À faire pour la séance suivante : résoudre l'équation $y'+2xy=0$. \vspace{.4cm}\\

\noindent\textbf{Lundi 17 octobre 2016}\\
\bu\ Feuille d'exercices n°6 : exercices 2 à 9 corrigés. \vspace{.4cm}\\

\noindent\textbf{Vendredi 14 octobre 2016}\\
\bu\ Cours sur les équations différentielles : parties 1.3 et 1.4.a. terminées. \vspace{.4cm}\\

\noindent\textbf{Jeudi 13 octobre 2016}\\
\bu\ Cours sur les fonctions usuelles : terminé.\\
\bu\ Cours sur les équations différentielles : parties 1.1 et 1.2 terminées. \\
\bu\ Feuille d'exercices n°5 : terminée.\\
\bu\ Feuille d'exercices n°6 : exercice 1 corrigé. \vspace{.4cm}\\

\noindent\textbf{Mercredi 12 octobre 2016}\\
\bu\ Cours sur les fonctions usuelles : partie 6 terminée.\vspace{.4cm}
\bu\ DM n°5 distribué, à rendre le mercredi 19 octobre.\\

\noindent\textbf{Lundi 10 octobre 2016}\\
\bu\ Interrogation écrite n°6.\\
\bu\ Cours sur les fonctions usuelles : parties 3, 4 et 5 terminées.  \\
\bu\ Feuille n°5 : exercices 1 à 6 et question 1 de l'exercice 7 corrigés. \\
\bu\ La fin de l'exercice 7 et l'exercice 8 de la feuille 5 ne seront pas corrigés en classe. Vous pouvez me les rendre rédigés si vous le désirez, cela fera le devoir facultatif des vacances d'automne.\vspace{.4cm}\\

\noindent\textbf{Vendredi 7 octobre 2016}\\
\bu\ Cours sur les fonctions usuelles : parties 1 et 2 terminées.  \\
\bu\ DS n°2 : calculs algébriques.\vspace{.4cm}\\

\noindent\textbf{Jeudi 6 octobre 2016}\\
\bu\ Cours sur les fonctions usuelles : partie 1 traitée jusqu'à la définition 1.2.4. (parties paires et impaires d'une fonction).  \\
\bu\ Feuille n°4 : terminée. \vspace{.4cm}\\

\noindent\textbf{Mercredi 5 octobre 2016}\\
\bu\ Cours sur les applications : terminé. \\
\bu\ Feuille n°5 : questions 1 et 2 de l'exercice 4 corrigé. \vspace{.4cm}\\

\noindent\textbf{Lundi 3 octobre 2016}\\
\bu\ Interrogation écrite n°5. \\
\bu\ Cours sur les applications : partie 4 traitée jusqu'à la remarque 4.3.8. \\
\bu\ Feuille n°3 terminée (l'exercice 19 n'a pas été corrigé : si vous voulez corriger un de ces systèmes, manifestez vous). \\
\bu\ Feuille n°4 : exercice 1 corrigé. \\
\bu\ Pour la séance suivante : chercher les questions 1 et 2 de l'exercice 4 de la feuille 5. \vspace{.4cm}\\

\noindent\textbf{Vendredi 30 septembre 2016}\\
\bu\ Cours sur les applications : parties 1 à 3 traitées. \vspace{.4cm}\\

\noindent\textbf{Jeudi 29 septembre 2016}\\
\bu\ DM n°4 distribué (à rendre le 6 octobre).\\
\bu\ Cours sur les ensembles : terminé.\\
\bu\ Feuille n°3 : exercices 9 à 14 corrigés. \vspace{.4cm}\\


\noindent\textbf{Mercredi 28 septembre 2016}\\
\bu\ Cours sur les ensembles : partie 2 jusqu'à l'exercice 2.3.5. (en cours).\vspace{.4cm}\\


\noindent\textbf{Lundi 26 septembre 2016}\\
\bu\ Interrogation écrite n°4.\\
\bu\ DS n°1 rendu. \\
\bu\ Feuille n°3 : exercices 1 à 8 corrigés. \\
\bu\ Cours sur les calculs algébriques : terminé.\vspace{.4cm}\\

\noindent\textbf{Vendredi 23 septembre 2016}\\
\bu\ Cours sur les calculs algébriques : parties 4  et 5.1 terminées.\vspace{.4cm}\\

\noindent\textbf{Jeudi 22 septembre 2016}\\
\bu\ Cours sur les calculs algébriques : partie 3 terminée, partie 4 jusqu'à la définition 4.2.3.\\
\bu\ Feuille n°2 terminée. \vspace{.4cm}\\

\noindent\textbf{Mercredi 21 septembre 2016}\\
\bu\ DM n°3 distribué (à rendre le 29 septembre). \\
\bu\ Cours sur les calculs algébriques : partie 1 et 2 terminées, partie 3 jusqu'au corollaire 3.0.6.\\
\bu\ Pour la séance suivante : réfléchir à l'hypothèse de récurrence à écrire pour montrer le corollaire 3.0.6. et comprendre pourquoi les récurrences proposées dans la partie 4.9 du chapitre 2 sont fausses. \vspace{.4cm}\\

\noindent\textbf{Lundi 19 septembre 2016}\\
\bu\ Interrogation écrite n°3.\\
\bu\ Cours de logique (II) : terminé.\\
\bu\ Cours sur les calculs algébriques : partie 1 jusqu'à l'exemple 1.0.6.\vspace{.4cm}\\

\noindent\textbf{Vendredi 16 septembre 2016}\\
\bu\ Cours de logique (II) : parties 2 et 3 terminées.\\
\bu\ Pour la séance suivante : exercices 3.2.5. et 3.4.1.\\
\bu\ DS n°1 : nombres complexes.\vspace{.4cm}\\

\noindent\textbf{Jeudi 15 septembre 2016}\\
\bu\ Cours de logique (II) : partie 2.3 terminée.\\
\bu\ Feuille n°1 : terminée.\vspace{.4cm}\\

\noindent\textbf{Mercredi 14 septembre 2016}\\
\bu\ Cours de logique (II) : parties 1, 2.1 et 2.2 traitées.\\
\bu\ Pour la séance suivante : travailler l'exercice n°2.2.5.\\
\bu\ Feuille n°1 : exercices 15, 18, 19 (question 1) corrigés.\vspace{.4cm}\\

\noindent\textbf{Lundi 12 septembre 2016}\\
\bu\ Interrogation écrite n°2.\\
\bu\ Cours sur les nombres complexes (I) terminé.\\
\bu\ Pour la séance suivante : travailler l'exemple n° 5.3.8.\\
\bu\ Feuille n°1 : exercices 7 à 14, exercice 15 débuté, exercice 16 (question 1).\\
\bu\ La feuille d'exercice sera corrigée jeudi, il convient donc de chercher ces exercices.\vspace{.4cm}\\

\noindent\textbf{Vendredi 09 septembre 2016}\\
\bu\ Cours sur les nombres complexes (I) : partie 4 terminée, partie 5 jusqu'à la définition 5.3.1 (similitudes et isométries).\\
\bu\ Feuille n°1 : exercice 6.\vspace{.4cm}\\

\noindent\textbf{Jeudi 08 septembre 2016}\\
\bu\ Cours sur les nombres complexes (I) : partie 3 terminée, partie 4 jusqu'à la partie 4.2.c (factorisation de sommes).\\
\bu\ DM n°2 (à rendre le 15 septembre). \\
\bu\ Feuille n°1 : exercices 3 à 5. Pour la séance suivante : préparer l'exercice 6.\\
\bu\ Pour la séance suivante : écrire la sixième ligne du triangle de Pascal et développer $(a+b)^6$.\vspace{.4cm}\\

\noindent\textbf{Mercredi 07 septembre 2016}\\
\bu\ Cours sur les nombres complexes (I) : partie 2 terminé, partie 3 jusqu'au début du théorème 3.2.1 (forme canonique et racines).\\
\bu\ Pour la séance suivante : Factoriser $z^4-1$.\vspace{.4cm}\\

\noindent\textbf{Lundi 05 septembre 2016}\\
\bu\ Interrogation écrite n°1.\\
\bu\ Cours sur les nombres complexes (I) : partie 2 jusqu'à la remarque 2.3.4.\\
\bu\ Feuille n°1 : exercices 1 et 2.\\
\bu\ Pour la séance suivante : expliciter $\mathbf{U}_7$ et déterminer les racines 5-ièmes de $42 e^{i\pi/3}$\vspace{.4cm}\\

\noindent\textbf{Vendredi 02 septembre 2016}\\
\bu\ Cours sur les nombres complexes (I) : partie 1 terminée, partie 2 jusqu'à la définition 2.1.3.\\
\bu\ Pour la séance de TD du lundi : préparer les exercices n° 1 et 2 de la feuille 1. \vspace{.4cm}\\

\noindent\textbf{Jeudi 01 septembre 2016}\\
\bu\ Journée de rentrée ; distribution des feuilles de TD, des formulaires, des
chapitres I et II.  \\
\bu\ DM n° 1 (à
rendre le 9 septembre). \\
\bu\ Cours sur les nombres complexes (I) : jusqu'au théorème 1.4.3. 
\label{end}
\end{document}


