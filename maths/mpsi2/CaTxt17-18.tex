\documentclass[12pt,a4paper]{article}

\textheight=25cm
\topmargin=-50pt
\input{/home/skanderk/.latex/intro2.sty}

\begin{document}

\begin{center}
\Large\bf CAHIER DE TEXTES DE MATHÉMATIQUES\\
MPSI 2 La Martinière Monplaisir\\ 2017-2018
\end{center}
\vspace{1cm}
\vspace{.4cm}

% 
% \noindent\textbf{\bf Jeudi 18 juin 2017} \\
% 
% 
% \noindent\textbf{\bf Mercredi 17 juin 2017} \\
% 
% \noindent\textbf{\bf Lundi 15 juin 2017} \\
% \bu\ Interrogation n° 24.\\
% \bu\ Cours : \bf Chapitre XXV \rm : Dénombrement : fin.\\
% \bu\ Exercices : feuille n° 24, ex. 20 à 22 et feuille n° 25, ex. 2, 3, 6 et 7.\vspace{.4cm}\\
% 
% \noindent\textbf{\bf Vendredi 12 juin 2017} \\
% \bu\ Exercices : feuille n° 24, ex. 17 à 19.\vspace{.4cm}\\
% 
% \noindent\textbf{\bf Jeudi 11 juin 2017} \\
% \bu\ Exercices : feuille n° 24, ex. 12 à 16.\vspace{.4cm}\\
% 
% \noindent\textbf{\bf Mercredi 10 juin 2017} \\
% \bu\ Cours : 3 - Automorphismes orthogonaux (fin).\\
% \bu\ Exercices : feuille n° 24, ex. 6 à 11.\vspace{.4cm}\\
% 
% \noindent\textbf{\bf Lundi 08 juin 2017} \\
% \bu\ Interrogation n° 23.\\
% \bu\ Cours : 3 - Automorphismes orthogonaux (suite).\\
% \bu\ Exercices : feuille n° 23, ex. 14 et feuille n° 24, ex. 1 à 5.\vspace{.4cm}\\
% 
% \noindent\textbf{Vendredi 05 juin 2017}\\ 

% 
% \noindent\textbf{ Mercredi 29 avril 2017} \\
% \bu\ Cours : 4 - Séries absolument convergentes ; 5 - Représentation décimale des réels ; 6 - 
% Compléments.\\
% \bu\ Exercices : feuille n° 20, ex. 19, et feuille n° 21, ex. 1.\vspace{.4cm}\\
%

%\bu\ Exercices : feuille n° 20, ex. 11 et 13 à 18.\vspace{.4cm}\\

% \noindent\textbf{\bf Lundi 13 juin 2018}\\
%  ; 2 - Séries à termes réels positifs ; 3 - Comparaison série - intégrale.\\
% \bu\ Exercices : feuille n° 25, ex. 14 à 22.\vspace{.4cm}\\
% 
% \noindent\textbf{Vendredi 10 juin 2018}\\
% \bu\ Devoir surveillé n° 10.\\
% \bu\ Exercices : feuille n° 25, ex. 12 et 13.\vspace{.4cm}\\

% \noindent\textbf{Vendredi 09 juin 2018}\\
% \bu\ Cours : \bf Chapitre XXV \rm : Séries : 1 - Prolégomènes.\\
% \bu\ Exercices : feuille n° 25, ex. 20 et 16.\vspace{.4cm}\\
% 
% \noindent\textbf{\bf Jeudi 08 juin 2018}\\
% \bu\ Exercices : feuille n° 25, ex. 12, 14, 15, 17, 19 et 21.\vspace{.4cm}\\
% 
% \noindent\textbf{\bf Mardi 06 juin 2018} \\
% \bu\ Interrogation n° 19.\\
% \bu\ Cours : 3 - Automorphismes orthogonaux (fin).\\
% \bu\ Exercices : feuille n° 25, ex. 9 et 10.\\
% \bu\ À faire pour jeudi 08 juin : feuille n° 25, ex. 21 et 22.\vspace{.4cm}\\
% 
% \noindent\textbf{Vendredi 02 juin 2018}\\
% \bu\ Cours : 3 - Automorphismes orthogonaux (suite).\\
% \bu\ Exercices : feuille n° 25, ex. 11 et 13.\vspace{.4cm}\\
% 
% \noindent\textbf{\bf Jeudi 01 juin 2018}\\
% \bu\ Distribution : DM n° 20 (à rendre le 08 juin).\\
% \bu\ Cours : 3 - Automorphismes orthogonaux (début).\\
% \bu\ Exercices : feuille n° 25, ex. 7 et 8.\vspace{.4cm}\\
% 
% \noindent\textbf{Mardi 30 mai 2018}\\
% \bu\ Cours : 2 - Orthogonalité (fin).\\
% \bu\ Exercices : feuille n° 25, ex. 5 et 6.\vspace{.4cm}\\
% 
% \noindent\textbf{\bf Lundi 29 mai 2018} \\
% \bu\ Interrogation n° 18.\\
% \bu\ Cours :  2 - Orthogonalité (suite).\\
% \bu\ Exercices : feuille n° 24, ex. 14 et 17, et feuille n° 25, ex. 1 à 4.\vspace{.4cm}\\
% 
% \noindent\textbf{Mardi 23 mai 2018}\\
% \bu\ Cours : \bf Chapitre XXIV \rm : Espaces vectoriels euclidiens : 1 - Produits scalaires, normes et distances 
% (fin).\\
% \bu\ Cours :  2 - Orthogonalité (début).\vspace{.4cm}\\
% 
% \noindent\textbf{\bf Lundi 22 mai 2018} \\
% \bu\ Interrogation n° 17.\\
% \bf Chapitre XXIV \rm : Espaces vectoriels euclidiens : 1 - Produits scalaires, normes et distances 
% (suite).\\
% \bu\ Exercices : feuille n° 24, ex. 10, 12 et 13.\vspace{.4cm}\\
% 
% \noindent\textbf{Vendredi 19 mai 2018}\\
% \bf Chapitre XXIV \rm : Espaces vectoriels euclidiens : 1 - Produits scalaires, normes et distances 
% (début).\\
% \bu\ Exercices : feuille n° 24, ex. 1 à 5.\vspace{.4cm}\\
% 
% \noindent\textbf{\bf Jeudi 18 mai 2018} \\
% \bu\ Distribution : DM n° 19 (à rendre le 01 juin).\\
% \bu\ Cours :  4 - Déterminant d'un endomorphisme.\\
% \bu\ Cours : 5 - Déterminant d'une matrice carrée.\vspace{.4cm}\\
% 
% \noindent\textbf{Mardi 16 mai 2018}\\
% \bu\ Cours : 3. Déterminant d’une famille de vecteurs.\vspace{.4cm}\\
% 
% \noindent\textbf{\bf Lundi 15 mai 2018} \\
% \bu\ Cours : \bf Chapitre XXIII \rm : Déterminants : 1 - Groupe symétrique (fin); 2. Applications multilinéaires.\\
% \bu\ Exercices : feuille n° 23, ex. 18, 20, 21 et 19 (début).\vspace{.4cm}\\
% 
% \noindent\textbf{Vendredi 12 mai 2018}\\
% \bu\ Cours : \bf Chapitre XXIII \rm : Déterminants : 1 - Groupe symétrique (début).\\
% \bu\ Exercices : feuille n° 23, ex. 13, 14 et 17.\vspace{.4cm}\\
% 
% \noindent\textbf{\bf Jeudi 11 mai 2018} \\
% \bu\ Cours : 7 - Matrices semblables et trace (fin) ; 8 - Matrices par blocs.\\
% \bu\ Exercices : feuille n° 23, ex. 2, 7, 10 et 11.\vspace{.4cm}\\
% 
% \noindent\textbf{Mardi 09 mai 2018}\\
% \bu\ Cours : 6 - Systèmes linéaires ; 7 - Matrices semblables et trace (début).\\
% \bu\ Exercices : feuille n° 2ex. 3 à 6, 8 et 9.\vspace{.4cm}\\
% 
% \noindent\textbf{\bf Jeudi 04 mai 2018} \\
% \bu\ Distribution : DM n° 18 (à rendre le 11 mai).\\
% \bu\ Interrogation n° 16.\\
% \bu\ Cours : 5 - Rang (fin).\\
% \bu\ Exercices : feuille n° 22, ex. 12.\\
% \bu\ À faire pour mardi 09/05 : feuille n° 23, 4, 6, 8 et 9, et pour vendredi 12/05, ex. 13 à 18.\vspace{.4cm}\\
% 
% \noindent\textbf{Mercredi 03 mai 2018}\\
% \bu\ Cours : 4 - Opérations élémentaires ; 5 - Rang (début).\\
% \bu\ Exercices : feuille n° 22, ex. 9 (fin).\vspace{.4cm}\\
% 
% \noindent\textbf{Mardi 02 mai 2018}\\
% \bu\ Cours : 3 - Matrices remarquables (fin).\\
% \bu\ Exercices : feuille n° 22, ex. 7, 10, 11 et 9 (suite).\\
% \bu\ À faire pour jeudi 04/05 : feuille n° 22, ex. 12.\vspace{.4cm}\\
% 
% \noindent\textbf{\bf Vacances de printemps}.\vspace{.4cm}\\
%  
% \noindent\textbf{Vendredi 14 avril 2018}\\
% \bu\ Cours : 2 - Matrices, familles de vecteurs et applications linéaires (fin) ; 3 - Matrices 
% remarquables (début).\\
% \bu\ Exercices : feuille n° 22, ex. 8 (fin) et 9 (début).\\
% \bu\ À faire pour mardi 02/05 : feuille n° 22, ex. 7, 10 et 11.\vspace{.4cm}\\
% 
% \noindent\textbf{Jeudi 13 avril 2018}\\
% \bu\ Distribution : DM n° 17 (à rendre le 04 mai).\\
% \bu\ Cours : 2 - Matrices, familles de vecteurs et applications linéaires (début).\\
% \bu\ Exercices : feuille n° 22, ex. 5, 6 et 8 (début).\vspace{.4cm}\\
% 
% \noindent\textbf{Mardi 11 avril 2018}\\
% \bf Chapitre XXII \rm : Matrices ; 1 - Structure de $\mcal M_{n,p}(\K)$.\vspace{.4cm}\\
%  
% \noindent\textbf{\bf Lundi 10 avril 2018} \\
% \bu\ Cours : 2 - Variables aléatoires (fin).\\
% \bu\ Exercices : feuille n° 22, ex. 1 à 4.\\
% \bu\ À faire pour mardi 11/04 : feuille n° 22, ex. 5.\vspace{.4cm}\\
%  
% \noindent\textbf{Vendredi 07 avril 2018}\\
% \bu\ Devoir surveillé n° 08.\\
% \bu\ Cours : 2 - Variables aléatoires (suite).\\
% \bu\ Exercices : feuille n° 21, ex. 16.\vspace{.4cm}\\
% 
% \noindent\textbf{Jeudi 06 avril 2018}\\
% \bu\ Cours : 2 - Variables aléatoires (suite).\\
% \bu\ Exercices : feuille n° 21, ex. 14 à 15 et 17 à 20.\vspace{.4cm}\\
% 
% \noindent\textbf{Mardi 04 avril 2018}\\
% \bu\ Cours : 2 - Variables aléatoires (début).\vspace{.4cm}\\
%  
% \noindent\textbf{\bf Lundi 03 avril 2018} \\
% \bu\ Interrogation n° 15.\\
% \bu\ Cours : 1 - Événements (fin).\\
% \bu\ Exercices : feuille n° 21, ex. 8 à 13.\vspace{.4cm}\\
% 
% \noindent\textbf{Vendredi 31 mars 2018}\\
% \bu\ Cours : Probabilités : 1 - Événements (suite).\\
% \bu\ Exercices : feuille n° 21, ex. 1 à 7.\vspace{.4cm}\\
% 
% \noindent\textbf{Jeudi 30 mars 2018}\\
% \bu\ Distribution : DM n° 16 (à rendre le 06 avril).\\
% \bu\ Cours : 4 - Formes linéaires et hyperplans.\\
% \bf Chapitre XXII \rm : Probabilités : 1 - Événements (début).\\
% \bu\ Exercices : feuille n° 20, ex. 7 et 8.\vspace{.4cm}\\
% 
% \noindent\textbf{\bf Mardi 28 mars 2018} \\
% \bu\ Cours : 2 - Sev en dimension finie (fin).\\
% 3 - Applications linéaires en dimension finie.\\
% \bu\ Exercices : feuille n° 20, ex. 3 à 6.\vspace{.4cm}\\
% 
% \noindent\textbf{\bf Lundi 27 mars 2018} \\
% \bu\ Interrogation n° 14 .\\
% \bu\ Cours : 2 - Sev en dimension finie (début).\\
% \bu\ Exercices : feuille n° 19, ex. 20 et 30, et feuille n° 20, ex. 1 et 2.\vspace{.4cm}\\
% 
% \noindent\textbf{Vendredi 24 mars 2018}\\
% \bu\ Cours : 1 - Notion de dimension (fin).\\
% \bu\ Exercices : feuille n° 19, ex. 25 à 29.\vspace{.4cm}\\
% 
% \noindent\textbf{Jeudi 23 mars 2018}\\
% \bu\ Distribution : DM n° 15 (à rendre le 30 mars).\\
% \bu\ Cours : 1 - Notion de dimension (suite).\\
% \bu\ Exercices : feuille n° 19, ex. 10, 12, et 16 à 19.\\
% \bu\ À faire pour vendredi 24/03 : feuille n° 19, ex. 25 à 27.\vspace{.4cm}\\
% 
% \noindent\textbf{\bf Lundi 20 mars 2018} \\
% \bu\ Cours : Dénombrement (fin).\\
% \bf Chapitre XXI \rm : Espaces vectoriels de dimension finie  : 1 - Notion de dimension (début).\\
% \bu\ Exercices : feuille n° 19, ex. 1 à 9, et 11.\\
% \bu\ À faire pour jeudi 23/03 : feuille n° 19, ex. 13 à 16, 18 et 24.\vspace{.4cm}\\
% 
% \noindent\textbf{Vendredi 17 mars 2018}\\
% \bu\ Devoir surveillé n° 07.\\
% \bu\ Cours : Intégration (fin).\\
% \bf Chapitre XX \rm : Dénombrement (début).\vspace{.4cm}\\
% 
% \noindent\textbf{Jeudi 16 mars 2018}\\
% \bu\ Cours : Intégration (suite).\\
% \bu\ Exercices : feuille n° 18, ex. 21 et 22.\vspace{.4cm}\\
% 
% \noindent\textbf{\bf Mardi 14 mars 2018} \\
% \bu\ Cours : Espaces vectoriels : 6 - Endomorphismes particuliers.\\
% \bf Chapitre XIX : \rm : Intégration (début).\\
% \bu\ Exercices : feuille n° 18, ex. 20 et 23.\vspace{.4cm}\\
% 
% \noindent\textbf{\bf Lundi 13 mars 2018} \\
% \bu\ Interrogation n° 13.\\
% \bu\ Cours : Espaces vectoriels : 5 - Familles de vecteurs (fin).\\
% \bu\ Exercices : feuille n° 18, ex. 14 à 19.\vspace{.4cm}\\
% 
% \noindent\textbf{Vendredi 10 mars 2018}\\
% \bu\ Cours : Espaces vectoriels : 5 - Familles de vecteurs (suite).\\
% \bu\ Exercices : feuille n° 18, ex. 13.\vspace{.4cm}\\
% 
% \noindent\textbf{Jeudi 09 mars 2018}\\
% \bu\ Cours : Espaces vectoriels : 4 - Applications linéaires (fin) ; 5 - Familles de vecteurs (début).\\
% \bu\ Exercices : feuille n° 18, ex. 6 à 10, et 12.\\
% \bu\ À faire pour vendredi 10/03 : feuille n° 18, ex. 11.\\
% \bu\ À faire pour lundi 13/03 : feuille n° 18, ex. 18.\vspace{.4cm}\\
% 
% \noindent\textbf{\bf Mardi 07 mars 2018} \\
% \bu\ Cours : Espaces vectoriels : 4 - Applications linéaires (début).\vspace{.4cm}\\
% 
% \noindent\textbf{\bf Lundi 06 mars 2018} \\
% \bu\ Interrogation n° 12.\\
% \bu\ Cours : Espaces vectoriels : 3 - Sea.\\
% \bu\ Exercices : feuille n° 17, ex. 30 et feuille n° 18, ex. 1 à 5.\\
% \bu\ À faire pour mardi 07/03 : feuille n° 18, ex. 6.\vspace{.4cm}\\
% 
% \noindent\textbf{\bf Vacances d'hiver }\\
% 
% \noindent\textbf{Vendredi 17 février 2018}\\
% \bu\ Exercices : feuille n° 17, ex. 22, 25 à 29 et 32.\vspace{.4cm}\\
% 
% \noindent\textbf{Jeudi 16 février 2018}\\
% \bu\ Distribution : DM n° 14 (à rendre le 16 mars).\\
% \bu\ Cours : Espaces vectoriels : 2 - Sev (fin).\\
% \bu\ Exercices : feuille n° 17, ex. 18, 20 et 21.\vspace{.4cm}\\
% 
% \noindent\textbf{\bf Mardi 14 février 2018} \\
% \bu\ Cours : Espaces vectoriels : 2 - Sev (suite).\\
% \bu\ Exercices : feuille n° 17, ex. 16 (fin) et des calculs.\vspace{.4cm}\\
% 
% \noindent\textbf{\bf Lundi 13 février 2018} \\
% \bu\ Cours : Espaces vectoriels : 2 - Sev (début).\\
% \bu\ Exercices : feuille n° 17, ex. 10 à 13, 14 (quelques questions) et 16 (début).\\
% \bu\ À faire pour mardi 14/02 : feuille n° 17, ex. 16 à finir, et ex. 17, qu. 1 à 3.\vspace{.4cm}\\
% 
% \noindent\textbf{Vendredi 10 février 2018}\\
% \bu\ Devoir surveillé n° 06.\\
% \bu\ Cours : \bf Chapitre XVIII \rm : Espaces vectoriels : 1 - Ev.\vspace{.4cm}\\ 
% 
% \noindent\textbf{Jeudi 09 février 2018}\\
% \bu\ Cours : Analyse asymptotique (fin).\\
% \bu\ Exercices : feuille n° 17, ex. 7 à 9.\vspace{.4cm}\\
% 
% \noindent\textbf{\bf Mardi 07 février 2018} \\
% \bu\ Interrogation n° 11.\\
% \bu\ Cours : Analyse asymptotique (suite).\vspace{.4cm}\\
% 
% \noindent\textbf{\bf Lundi 06 février 2018} \\
% \bu\ Cours : Analyse asymptotique (suite).\\
% \bu\ Exercices : feuille n° 16, ex. 8, et feuille n° 17, ex. 1 à 6.\vspace{.4cm}\\
% 
% \noindent\textbf{Vendredi 03 février 2018}\\
% \bu\ Cours : Analyse asymptotique (suite).\\
% \bu\ Exercices : feuille n° 16, ex. 7.\vspace{.4cm}\\
% 
% \noindent\textbf{Jeudi 02 février 2018}\\
% \bu\ Distribution : DM n° 13 (à rendre le 09 février).\\
% \bf Chapitre XVII \rm : Analyse asymptotique (début).\\
% \bu\ Exercices : feuille n° 16, ex. 4, 5 et 6 (partiellement) et 7 (début).\vspace{.4cm}\\
% 
% \noindent\textbf{ Mardi 31 janvier 2018} \\
% \bu\ Cours : Fractions rationnelles (fin).\\
% \bu\ Exercices : feuille n° 16, ex. 1 et 3.\vspace{.4cm}\\
% 
% \noindent\textbf{ Lundi 30 janvier 2018} \\
% \bu\ Cours : Fractions rationnelles (suite).\\
% \bu\ Exercices : feuille n° 15, ex. 14 à 18, et feuille n° 16, ex. 2.\\
% \bu\ À faire pour vendredi 3 : feuille n° 16, ex. 5 et 6.\vspace{.4cm}\\ 
% 
% \noindent\textbf{Vendredi 27 janvier 2018}\\
% \bu\ Cours : Fractions rationnelles (suite).\\
% \bu\ Exercices : feuille n° 15, ex. 11 à 13.\\
% \bu\ À faire pour lundi 30 : feuille n° 15, ex. 17.\vspace{.4cm}\\  
% 
% \noindent\textbf{Jeudi 26 janvier 2018}\\
% \bu\ Distribution : DM n° 12 (à rendre le 02 février).\\
% \bf Chapitre XVI \rm : Fractions rationnelles (début).\\
% \bu\ Exercices : feuille n° 15, ex. 3, 4 et 7 à 10.\\
% \bu\ À faire pour vendredi 27 : feuille n° 15, ex. 11.\vspace{.4cm}\\  
% 
% \noindent\textbf{ Mardi 24 janvier 2018} \\
% \bu\ Cours : Dérivabilité (fin).\\
% \bu\ Exercices : feuille n° 15, ex. 3, 4 et 6.\\
% \bu\ À faire pour mardi 24 : feuille n° 15, ex. 3 et 4 (fin).\vspace{.4cm}\\  
% 
% \noindent\textbf{ Lundi 23 janvier 2018} \\
% \bu\ Cours : Dérivabilité (suite).\\
% \bu\ Exercices : feuille n° 14, ex. 18 et 19, et feuille n° 15, ex. 2, 4 (début) et 5.\\
% \bu\ À faire pour mardi 24 : feuille n° 15, ex. 3 et 4 (fin).\vspace{.4cm}\\  
% 
% \noindent\textbf{Vendredi 20 janvier 2018}\\
% \bu\ Devoir surveillé n° 05.\\
% \bu\ Cours : Dérivabilité (suite).\\
% \bu\ Exercices : feuille n° 14, ex. 15 à 17.\vspace{.4cm}\\
% 
% \noindent\textbf{Jeudi 19 janvier 2018}\\
% \bu\ Cours : \bf Chapitre XV \rm : Dérivabilité (début).\\
% \bu\ Exercices : feuille n° 14, ex. 8 à 12.\\
% \bu\ À faire pour vendredi 20 : feuille n° 14, ex. 15.\vspace{.4cm}\\ 
% 
% \noindent\textbf{Mardi 17 janvier 2018}\\
% \bu\ Cours : Polynômes (fin).\\
% \bu\ Exercices : feuille n° 14, ex. 1, 2, 5.\\
% \bu\ À faire pour jeudi 19 : feuille n° 14, ex. 8.\\  
% \bu\ À faire pour lundi 23 : feuille n° 14, ex. 13 et 14.\vspace{.4cm}\\  
%  
% \noindent\textbf{ Lundi 16 janvier 2018}\\
% \bu\ Interrogation n° 10.\\
% \bu\ Cours : Polynômes (suite).\\
% \bu\ Exercices : feuille n° 14, ex. 3, 4, 5 (début), 6 et 7.\\
% \bu\ À faire pour mardi 17 : feuille n° 14, ex. 1, 2 et 5.\vspace{.4cm}\\  
% 
% \noindent\textbf{Vendredi 13 janvier 2018}\\
% \bu\ Cours : Polynômes (suite).\\
% \bu\ Exercices : feuille n° 13, ex. 9 et 12 à 14.\\
% \bu\ À faire pour lundi 16 : feuille n° 13, ex. 11.\vspace{.4cm}\\
% 
% \noindent\textbf{Jeudi 12 janvier 2018}\\
% \bu\ Distribution : DM n° 11 (à rendre le 19 janvier).\\
% \bu\ Cours : Polynômes (suite).\\
% \bu\ Exercices : feuille n° 13, ex. 4 à 6, 8 et 9 (début).\\
% \bu\ À faire pour vendredi 13 : feuille n° 13, ex. 9 à finir.\vspace{.4cm}\\
% 
% \noindent\textbf{Mardi 10 janvier 2018} \\
% \bu\ Cours : Polynômes (suite).\\
% \bu\ Exercices : feuille n° 13, ex. 1 à 3.\\
% \bu\ À faire pour jeudi 12 : feuille n° 13, ex. 5.\vspace{.4cm}\\
%  
% \noindent\textbf{ Lundi 09 janvier 2018} \\
% \bu\ Interrogation n° 09.\\
% \bu\ Cours : Polynômes (suite).\\
% \bu\ Exercices : feuille n° 12, ex. 1 à 7, et feuille n° 13, ex. 1 (début).\\
% \bu\ À faire pour mardi 10 : feuille n° 13, ex. 1 à finir.\vspace{.4cm}\\
% 
% \noindent\textbf{Vendredi 06 janvier 2018}\\
% \bu\ Cours : Polynômes (suite).\\
% \bu\ À faire pour vendredi 06 : feuille n° 11, ex. 10 à 12.\vspace{.4cm}\\
% 
% \noindent\textbf{Jeudi 05 janvier 2018}\\
% \bu\ Distribution : DM n° 10 (à rendre le 12 janvier).\\
% \bu\ Cours : Continuité (fin).\\
% \bu\ \bf Chapitre XIV \rm : Polynômes (début).\\
% \bu\ Exercices : feuille n° 11, ex. 1 et 3 à 9.\\
% \bu\ À faire pour vendredi 06 : feuille n° 11, ex. 10.\\
% \bu\ À faire pour lundi 09 : feuille n° 12, ex. 1 et 3.\vspace{.4cm}\\
% 
% \noindent\textbf{Mardi 03 janvier 2018} \\
% \bu\ Cours : Continuité (suite).\vspace{.4cm}\\
%  
% \noindent\textbf{\bf Vacances de Noël}.\vspace{.4cm}\\
% 
% \noindent\textbf{Vendredi 16 décembre 2017}\\
% \bu\ \bf Chapitre XIII \rm : Continuité (début).\\
% \bu\ Exercices : feuille n° 10, ex. 17, et feuille n° 11, ex. 2.\\
% \bu\ À faire pour mardi 03 : feuille n° 11, ex. 1 et 3.\vspace{.4cm}\\
% 
% \noindent\textbf{Mardi 13 décembre 2017} \\
% \bf Chapitre XII \rm : Limites d'une fonction (fin).\\
% \bu\ Exercices : feuille n° 10, ex. 15 et 16.\vspace{.4cm}\\
% 
% \noindent\textbf{Lundi 12 décembre 2017} \\
% \bf Chapitre XII \rm : Limites d'une fonction (suite).\\
% \bu\ Exercices : feuille n° 10, ex. 6, 8, 9, 13 et 14.\vspace{.4cm}\\
%   
% \noindent\textbf{Vendredi 09 décembre 2017}\\
% \bu\ Devoir surveillé n° 04.\\
% \bf Chapitre XII \rm : Limites d'une fonction (début).\vspace{.4cm}\\
% 
% \noindent\textbf{Jeudi 08 décembre 2017}\\
% \bu\ Cours : Groupes, anneaux, corps (fin).\\
% \bu\ Exercices : feuille n° 10, ex. 10 et 12.\\
% \bu\ À faire pour vendredi 09 : feuille n° 10, ex. 6.\vspace{.4cm}\\
% 
% \noindent\textbf{Mardi 06 décembre 2017} \\
% \bu\ Cours : Groupes, anneaux, corps (suite).\\
% \bu\ Exercices : feuille n° 10, ex. 2, 3, 5 et 7.\\
% \bu\ À faire pour jeudi 08 : feuille n° 10, ex. 4.\vspace{.4cm}\\
% 
% \noindent\textbf{Lundi 05 décembre 2017} \\
% \bu\ Interrogation n° 08.\\
% \bu\ Cours : \bf Chapitre XI \rm : Groupes, anneaux, corps (début).\\
% \bu\ Exercices : feuille n° 10, ex. 1.\vspace{.4cm}\\
%   
% \noindent\textbf{Vendredi 02 décembre 2017}\\
% \bu\ Cours : 6 - Suites récurrentes ; 7 : suites complexes ; 8 - Premières séries numériques.\vspace{.4cm}\\
%   
% \noindent\textbf{Jeudi 01 décembre 2017}\\
% \bu\ Distribution : DM n° 9 (à rendre le 08 décembre).\\
% \bu\ Cours : 4 et 5 - Suites particulières.\vspace{.4cm}\\
% 
% \noindent\textbf{Mardi 29 novembre 2017}\\
% \bu\ Cours : 3 - Résultats de convergence (fin).\\
% \bu\ Exercices : feuille n° 9, ex. 17 à 19.\vspace{.4cm}\\
% 
% \noindent\textbf{Lundi 28 novembre 2017} \\
% \bu\ Interrogation n° 07.\\
% \bu\ Cours : 3 - Résultats de convergence (début).\\
% \bu\ Exercices : feuille n° 9, ex. 8, 11 et 13 à 16.\vspace{.4cm}\\
% 
% \noindent\textbf{Vendredi 25 novembre 2017}\\
% \bu\ Cours : 2 - Limite d'une suite réelle (fin).\\
% \bu\ À faire pour lundi 28 : feuille n° 9, ex. 8 et 11.\vspace{.4cm}\\
% 
% \noindent\textbf{Jeudi 24 novembre 2017}\\
% \bu\ Distribution : DM n° 08 (à rendre le 01 décembre).\\
% \bu\ Cours : 3 - Nombres premiers (fin).\\
% \bf Chapitre X :\rm Suites réelles et complexes : 1 - Vocabulaire ; 2 - Limite 
% d'une suite réelle (début).\\
% \bu\ Exercices : feuille n° 9, ex. 4, 6, 9, 10 et 12.\vspace{.4cm}\\
% 
% \noindent\textbf{Mardi 22 novembre 2017}\\
% \bu\ Cours : 2 - PGCD (fin), PPCM ; 3 - Nombres premiers (début).\vspace{.4cm}\\
%    
% \noindent\textbf{Lundi 21 novembre 2017}\\
% \bu\ Cours : 2 - PGCD (début).\\
% \bu\ Exercices : feuille n° 8, ex. 2, 9, 10 et 11 et feuille n° 9, ex. 1, 2, 3, 5 et 7.\vspace{.4cm}\\
% 
% \noindent\textbf{Vendredi 18 novembre 2017}\\
% \bu\ Devoir surveillé n° 3.\\
% \bf Cours : 1 - Divisibilité (fin).\\
% \bu\ Exercices : feuille n° 8, ex. 6 (fin) et 8.\\
% \bu\ À faire pour lundi 16 novembre : feuille n° 8, ex. 10.\vspace{.4cm}\\
%  
% \noindent\textbf{Jeudi 17 novembre 2017}\\
% \bu\ Cours : 5 - La relation d'ordre naturelle sur \R\ (fin) ;\\
% \bf Chapitre IX \rm : Arithmétique dans \Z\ : 1 - Divisibilité (début).\\
% \bu\ Exercices : feuille n° 8, ex. 3, 4, 6 (début) et 7.\\
% \bu\ À faire pour vendredi 13 novembre : feuille n° 8, ex. 6 (fin).\vspace{.4cm}\\
% 
% \noindent\textbf{Mardi 15 novembre 2017}\\
% \bu\ Cours : 4 - Majorants, minorants et compagnie (fin) ; 5 - La relation d'ordre naturelle sur \N\ ; 6 - La relation 
% d'ordre naturelle sur \R\ (début).\\
% \bu\ Exercices : feuille n° 8, ex. 5.\\
% \bu\ À faire pour jeudi 17 : feuille n° 8, ex. 3.\vspace{.4cm}\\
% 
% \noindent\textbf{Lundi 14 novembre 2017}\\
% \bu\ Interrogation surprise n° 06.\\
% \bu\ Cours : 2 - Relations d'ordre ;  4 - Majorants, minorants et compagnie (début).\\
% \bu\ Exercices : feuille n° 7, ex. 7 (fin), 2 (fin), 9 et 10, et feuille n° 8, ex. 1.\\
% \bu\ À faire pour mardi 15 : feuille n° 8, ex. 5, qu. 1.\vspace{.4cm}\\
% 
% \noindent\textbf{Jeudi 10 novembre 2017}\\
% \bu\ Distribution : DM n° 07 (à rendre le 17 novembre).\\
% \bu\ Cours : 2 - Relations d'équivalence.\\
% \bu\ Exercices : feuille n° 7, ex. 4 (fin), 5, 6, 7 (début) et 8.\vspace{.4cm}\\
% 
% \noindent\textbf{Mardi 08 novembre 2017}\\
% \bu\ Cours : 3.2 - Équations différentielles du second ordre avec second membre ; 5. Circuits RL et RLC.\\
% \bf Chapitre VIII \rm : Relations d'ordre : 1 - Relations binaires.\\
% \bu\ Exercices : feuille n° 7, ex. 4 (début).\vspace{.4cm}\\
% 
% \noindent\textbf{Lundi 07 novembre 2017} \\
% \bu\ Interrogation surprise n° 05.\\
% \bu\ Cours : 3.2 - Équations différentielles du premier ordre avec second membre (fin) ; 4.1 - Équations 
% différentielles homogènes du second ordre.\\
% \bu\ Exercices : feuille n° 6, ex. 11, et feuille n° 7, ex. 1, 2 (début) et 3.\vspace{.4cm}\\
% 
% \noindent\textbf{Vendredi 04 novembre 2017}\\
% \bu\ Cours : 3.2 - Équations différentielles du premier ordre avec 
% second membre (début).\\
% \bu\ Exercices : feuille n° 5, ex. 10, et feuille n° 6, ex. 2, 9, 10 et 12 à 16.\\
% \bu\ À faire pour lundi 07 :feuille n° 6, ex. 11.\vspace{.4cm}\\
% 
% \noindent\textbf{Jeudi 03 novembre 2017}\\
% \bu\ Distribution : DM n° 06 (à rendre le 10 novembre).\\
% \bu\ Cours : 3.1 - Équations différentielles homogènes du premier ordre.\\
% \bu\ Exercices : feuille n° 6, ex. 6 à 8.\vspace{.4cm}\\
% 
% \noindent\textbf{ Vacances de novembre }\vspace{.4cm}\\
% 
% \noindent\textbf{\bf Mardi 18 octobre 2017}\\
% \bu\ Cours : 2 - Généralités.\\
% \bu\ Exercices : feuille n° 6, ex. 1, 3, 4 et 5.\\
% \bu\ À faire pour jeudi 03 :feuille n° 5, ex. 10, et feuille n° 6, ex. 2.\vspace{.4cm}\\
% 
% \noindent\textbf{Lundi 17 octobre 2017}\\
% \bu\ Interrogation surprise n° 4.\\
% \bu\ Cours : 1 - Résultats d'analyse relatifs aux fonctions à valeurs complexes d'une variable réelle, et 
% intégration (fin).\\
% \bu\ Exercices : feuille n° 5, ex. 6, 7, 9, 11 (fin) et 12.\\
% \bu\ À faire pour mardi 18 :feuille n° 6, ex. 3 et 4.\vspace{.4cm}\\
%    
% \noindent\textbf{Vendredi 14 octobre 2017}\\
% \bu\ Exercices : feuille n° 5, ex. 11 (début).\\
% \bu\ À faire pour lundi 17 :feuille n° 5, ex. 9 et 11.2.b.\vspace{.4cm}\\
% 
% \noindent\textbf{Jeudi 13 octobre 2017}\\
% \bu\ Cours :  6 - Fonctions trigonométriques inverses (fin) ; 7 - Fonctions hyperboliques.\\
% \bu\ Exercices : feuille n° 5, ex. 3, 4 et 5.\vspace{.4cm}\\
% 
% \noindent\textbf{\bf Mardi 11 octobre 2017}\\
% \bu\ Distribution : DM n° 5 (à rendre le 19 octobre).\\
% \bu\ Cours : 4 - Puissances entières, fonctions 
% polynomiales et rationnelles ; 5 - Fonctions exponentielle, logarithme et 
% exponentielles de base $a$ ; 6 - Fonctions trigonométriques inverses (début).\\
% \bu\ À faire pour jeudi 13 : feuille n° 5, ex. 4, qu. 1 et 2.\vspace{.4cm}\\
% 
% \noindent\textbf{Lundi 10 octobre 2017}\\
% \bu\ Cours :  2 - Théorèmes d'analyse admis ; 3 - Valeur absolue.\\
% \bu\ Exercices : feuille n° 5, ex. 1 et 2.\vspace{.4cm}\\
%    
% \noindent\textbf{Vendredi 07 octobre 2017}\\
% \bu\ Devoir surveillé n° 2.\\
% \bu\ Exercices : feuille n° 4, ex. 7 et 8.\\
% \bu\ À faire pour lundi 10 : exercices du cours.\vspace{.4cm}\\
% 
% \noindent\textbf{Jeudi 06 octobre 2017}\\
% \bu\ Cours : 4 - Bijectivité ; 5 - Images directe et réciproque.\\
% \bu\ Exercices : feuille n° 4, ex. 5 et 6 (fin).\\
% \bu\ À faire pour vendredi 03 : feuille n° 4, ex. 8.\vspace{.4cm}\\
% 
% \noindent\textbf{\bf Mardi 04 octobre 2017}\\
% \bu\ Cours : 4 - Injectivité, surjectivité.\\
% \bu\ Exercices : feuille n° 4, ex. 4 et 6 (début).\\
% \bu\ À faire pour jeudi 06 : feuille n° 4, ex. 5.\vspace{.4cm}\\
% 
% \noindent\textbf{Lundi 03 octobre 2017}\\
% \bu\ Interrogation surprise n° 3.\\
% \bu\ Cours : 3 - Composition.\\
% \bu\ Exercices : feuille n° 3, ex. 16 à 21.\\
% \bu\ À faire pour mardi 04 : feuille n° 4, ex. 4.\vspace{.4cm}\\
%   
% \noindent\textbf{Vendredi 30 septembre 2017}\\
% \bu\ Cours : \bf Chapitre V \rm : Notion d'application. 1 - Vocabulaire ; 2 - Restriction et prolongement.\\
% \bu\ Exercices : feuille n° 3, ex. 12, 13 et 14 (fin), .\\
% \bu\ À faire pour lundi 03 : feuille n° 3, ex. 18 à 21.\vspace{.4cm}\\
% 
% \noindent\textbf{Jeudi 29 septembre 2017}\\
% \bu\ Distribution : DM n° 4 (à rendre le 04 octobre).\\
% \bu\ Cours : Théorie des ensembles (début).\\
% \bu\ Exercices : feuille n° 3, ex. 8, 11, 13 et 14 (début).\\
% \bu\ À faire pour vendredi 30 : feuille n° 3, ex. 14 (fin).\vspace{.4cm}\\



\noindent\textbf{Vendredi 13 octobre 2017}\\
\bu\ Cours : 1 - Résultats d'analyse relatifs aux fonctions à valeurs complexes d'une variable réelle, et intégration 
(fin).\\
\bu\ Exercices : feuille n° 04, ex. 18 et 19.\\
\bu\ À faire pour lundi 16 : feuille n° 04, ex. 20 et 22.\\
\bu\ À faire pour mercredi 18 : feuille n° 05, ex. 1.\\
\bu\ À faire pour jeudi 19 : feuille n° 04, ex. 23.\\
\bu\ À faire pour vendredi 20 : feuille n° 04, ex. 26.\vspace{.4cm}\\

\noindent\textbf{Jeudi 12 octobre 2017}\\
\bu\ Distribution du DM n° 05 (à rendre le 09 novembre).\\
\bu\ Cours : 1 - Résultats d'analyse relatifs aux fonctions à valeurs complexes d'une variable réelle, et intégration 
(suite).\\
\bu\ Exercices : feuille n° 04, ex. 12 et 16.\vspace{.4cm}\\

\noindent\textbf{\bf Mercredi 11 octobre 2017}\\
\bu\ Cours : \bf Chapitre VII \rm : Équations différentielles linéaires : 1 - Résultats d'analyse relatifs aux 
fonctions à valeurs complexes d'une variable réelle, et intégration (début).\\
\bu\ Exercices : feuille n° 04, ex. 8 et 9.\vspace{.4cm}\\

\noindent\textbf{\bf Lundi 09 octobre 2017}\\
\bu\ Interrogation surprise n° 04.\\
\bu\ Cours : 5 - Nombres complexes et géométrie plane (fin).\\
\bu\ Exercices : feuille n° 04, ex. 2, 3, 4 et 7.\vspace{.4cm}\\

\noindent\textbf{Vendredi 06 octobre 2017}\\
\bu\ Cours : 4 - Techniques de calcul (fin) ; 5 - Nombres complexes et géométrie plane (début).\\
\bu\ Exercices : feuille n° 03, ex. 11.\\
\bu\ À faire pour lundi 09 : feuille n° 04, ex. 3, 4 et 7.\\
\bu\ À faire pour mercredi 11 : feuille n° 04, ex. 8 et 9.\\
\bu\ À faire pour jeudi 12 : feuille n° 04, ex. 16.\\
\bu\ À faire pour vendredi 13 : feuille n° 04, ex. 18 et 19.\vspace{.4cm}\\

\noindent\textbf{Jeudi 05 octobre 2017}\\
\bu\ Distribution du DM n° 04 (à rendre le 12 octobre).\\
\bu\ Cours : 2 - Le groupe \U\ des nombres complexes de module 1 (fin) ; 3 - Équations du second degré ; 4 - Techniques 
de calcul (début).\\
\bu\ Exercices : feuille n° 03, ex. 7 à 9 et 12, et feuille n° 4, ex. 1 et 6.\vspace{.4cm}\\

\noindent\textbf{\bf Mercredi 04 octobre 2017}\\
\bu\ Distribution du DM n° 04 (à rendre le 12 octobre).\\
\bu\ Cours : \bf Chapitre IV \rm : 1 - Construction de \C\ (fin) ; 2 - Le groupe \textbf{U} des nombres 
complexes de module 1 (début).\\
\bu\ Exercices : feuille n° 03, ex. 4 à 6.\vspace{.4cm}\\

\noindent\textbf{\bf Lundi 02 octobre 2017}\\
\bu\ Interrogation surprise n° 03.\\
\bu\ Cours : \bf Chapitre IV \rm : 1 - Construction de \C\ (début).\\
\bu\ Exercices : feuille n° 02, ex. 15, et 20 (fin), et feuille n° 3, ex. 1 à 3.\\
\bu\ À faire pour mercredi 04 : feuille n° 03, ex. 4, 5 et 6.\\
\bu\ À faire pour jeudi 05 : feuille n° 03, ex. 8 et 9.\\
\bu\ À faire pour vendredi 06 : feuille n° 03, ex. 11.\vspace{.4cm}\\

\noindent\textbf{Vendredi 29 septembre 2017}\\
\bu\ Cours : Quelques fondamentaux (fin).\\
\bu\ Exercices : feuille n° 01, ex. 16, 19 et 20 (début).\\
\bu\ À faire pour lundi 02 : récurrences erronées et feuille n° 02, ex. 20 à finir.\vspace{.4cm}\\

\noindent\textbf{Jeudi 28 septembre 2017}\\
\bu\ Distribution du DM n° 03 (à rendre le 05 octobre).\\
\bu\ Cours : \bf Chapitre III \rm : Quelques fondamentaux (début).\\
\bu\ Exercices : feuille n° 02, ex. 13, 14, 17 et 18.\vspace{.4cm}\\

\noindent\textbf{\bf Mercredi 27 septembre 2017}\\
\bu\ Cours : 5 - Systèmes linéaires (fin).\\
\bu\ Exercices : feuille n° 02, ex. 9, 11 et 12.\vspace{.4cm}\\

\noindent\textbf{\bf Lundi 25 septembre 2017}\\
\bu\ Cours : 4 - Matrices (fin) ; 5 - Systèmes linéaires (début).\\
\bu\ Exercices : feuille n° 02, ex. 2, 3, 5, 6, 7, 8 et 10.\\
\bu\ À faire pour mercredi 27 : feuille n° 02, ex. 12.\\
\bu\ À faire pour jeudi 28 : feuille n° 02, ex. 14.\\
\bu\ À faire pour vendredi 29 : feuille n° 02, ex. 16.\vspace{.4cm}\\

\noindent\textbf{Vendredi 22 septembre 2017}\\
\bu\ Devoir surveillé n° 01.\\
\bu\ Cours : 4 - Matrices (suite).\\
\bu\ Exercices : feuille n° 02, ex. 4.\\
\bu\ À faire pour lundi 25 : feuille n° 02, ex. 6.\vspace{.4cm}\\

\noindent\textbf{Jeudi 21 septembre 2017}\\
\bu\ Cours : 3 - Quelques formules (fin) ; 4 - Matrices (début).\\
\bu\ Exercices : feuille n° 01, ex. 16, 18 et 19.\vspace{.4cm}\\

\noindent\textbf{\bf Mercredi 20 septembre 2017}\\
\bu\ Cours : 3 - Quelques formules (début).
\bu\ Exercices : feuille n° 01, ex. 15, et feuille n° 02, ex. 1.\vspace{.4cm}\\

\noindent\textbf{\bf Lundi 18 septembre 2017}\\
\bu\ Interrogation surprise n° 02.\\
\bf Chapitre II \rm : Un peu de calcul. 1 - Sommes ; 2 - Produits.\\
\bu\ Exercices : feuille n° 01, ex. 11 à 14.\\
\bu\ À faire pour mercredi 20 : feuille n° 01, ex. 15, et feuille n° 02, ex. 1.\\
\bu\ À faire pour jeudi 21 : feuille n° 01, ex. 16 et 18.\\
\bu\ À faire pour vendredi 22 : feuille n° 02, ex. 4.\vspace{.4cm}\\

\noindent\textbf{Vendredi 15 septembre 2017}\\
\bu\ Cours : 8 - Fonctions circulaires inverses (fin) ; 9 - Fonctions hyperboliques.\\
\bu\ Exercices : feuille n° 01, ex. 9 et 10.\vspace{.4cm}\\

\noindent\textbf{\bf Jeudi 14 septembre 2017}\\
\bu\ Distribution du DM n° 02 (à rendre le 21 septembre).\\
\bu\ Cours : 7 - Fonctions circulaires ; 8 - Fonctions circulaires inverses (début).\vspace{.4cm}\\

\noindent\textbf{\bf Mercredi 13 septembre 2017}\\
\bu\ Cours : 6 - Exponentielle et logarithme (fin).\\
\bu\ Exercices : feuille n° 01, ex. 3 (fin), 4, 7 et 8.\vspace{.4cm}\\

\noindent\textbf{\bf Lundi 11 septembre 2017}\\
\bu\ Interrogation surprise n° 01.\\
\bu\ Cours : 5 - Puissances (fin) ; 6 - Exponentielle et logarithme (début).\\
\bu\ Exercices : feuille n° 01, ex. 1 (fin), 5, 6 et 3 (début).\\
\bu\ À faire pour mercredi 13 : feuille n° 01, ex. 7 et 8.\\
\bu\ À faire pour vendredi 15 : feuille n° 01, ex. 9 et 10.\vspace{.4cm}\\

\noindent\textbf{Vendredi 08 septembre 2017}\\
\bu\ Cours : 3 - Réciproques ; 4 - Fonction valeur absolue ; 5 - Puissances (début).\vspace{.4cm}\\

\noindent\textbf{\bf Jeudi 07 septembre 2017}\\
\bu\ Cours : 1 - Vocabulaire (fin) ; 2 - Effet d'une transformation sur le graphe ; 3 - Composée de fonctions.\\
\bu\ Exercices : feuille n° 01, ex. 2 et 1 (début).\vspace{.4cm}\\
   
\noindent\textbf{\bf Mercredi 06 septembre 2017}\\
\bu\ Cours : \bf Chapitre VI \rm : Fonctions usuelles : 1 - Vocabulaire (début).\\
\bu\ À faire pour jeudi 07 : feuille n° 01, ex. 2.\\
\bu\ À faire pour vendredi 08 : feuille n° 01, ex. 3.\\
\bu\ À faire pour lundi 11 : feuille n° 01, ex. 4.\vspace{.4cm}\\

\noindent\textbf{\bf Mardi 05 septembre 2017}\\
Journée de rentrée ; distribution des feuilles de TD, des formulaires, des
chapitres I et II, et du DM n° 01 (à rendre le 14 septembre).\vspace{.4cm}\\


\label{end}
\end{document}


