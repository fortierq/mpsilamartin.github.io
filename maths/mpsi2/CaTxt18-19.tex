\documentclass[12pt,a4paper]{article}

\textheight=25cm
\topmargin=-50pt
\input{/home/skanderk/.latex/intro2.sty}

\begin{document}

\begin{center}
\Large\bf CAHIER DE TEXTES DE MATHÉMATIQUES\\
MPSI 2 La Martinière Monplaisir\\ 2018-2019
\end{center}
\vspace{1cm}
\vspace{.4cm}

% 
% \noindent\textbf{\bf Jeudi 18 juin 2018} \\
% 
% 
% \noindent\textbf{\bf Mercredi 17 juin 2018} \\
% 
% \noindent\textbf{\bf Lundi 15 juin 2018} \\
% \bu\ Interrogation n° 24.\\
% \bu\ Cours : \bf Chapitre XXV \rm : Dénombrement : fin.\\
% \bu\ Exercices : feuille n° 24, ex. 20 à 22 et feuille n° 25, ex. 2, 3, 6 et 7.\vspace{.4cm}\\
% 
% \noindent\textbf{\bf Vendredi 12 juin 2018} \\
% \bu\ Exercices : feuille n° 24, ex. 17 à 19.\vspace{.4cm}\\
% 
% \noindent\textbf{\bf Jeudi 11 juin 2018} \\
% \bu\ Exercices : feuille n° 24, ex. 12 à 16.\vspace{.4cm}\\
% 
% \noindent\textbf{\bf Mercredi 10 juin 2018} \\
% \bu\ Cours : 3 - Automorphismes orthogonaux (fin).\\
% \bu\ Exercices : feuille n° 24, ex. 6 à 11.\vspace{.4cm}\\
% 
% \noindent\textbf{\bf Lundi 08 juin 2018} \\
% \bu\ Interrogation n° 23.\\
% \bu\ Cours : 3 - Automorphismes orthogonaux (suite).\\
% \bu\ Exercices : feuille n° 23, ex. 14 et feuille n° 24, ex. 1 à 5.\vspace{.4cm}\\
% 
% \noindent\textbf{Vendredi 05 juin 2018}\\ 

% 
% \noindent\textbf{ Mercredi 29 avril 2018} \\
% \bu\ Cours : 4 - Séries absolument convergentes ; 5 - Représentation décimale des réels ; 6 - 
% Compléments.\\
% \bu\ Exercices : feuille n° 20, ex. 19, et feuille n° 21, ex. 1.\vspace{.4cm}\\
%

%\bu\ Exercices : feuille n° 20, ex. 11 et 13 à 18.\vspace{.4cm}\\

% \noindent\textbf{\bf Lundi 13 juin 2019}\\
%  ; 2 - Séries à termes réels positifs ; 3 - Comparaison série - intégrale.\\
% \bu\ Exercices : feuille n° 25, ex. 14 à 22.\vspace{.4cm}\\
% 
% \noindent\textbf{Vendredi 10 juin 2019}\\
% \bu\ Devoir surveillé n° 10.\\
% \bu\ Exercices : feuille n° 25, ex. 12 et 13.\vspace{.4cm}\\

% \noindent\textbf{Vendredi 09 juin 2019}\\
% \bu\ Cours : \bf Chapitre XXV \rm : Séries : 1 - Prolégomènes.\\
% \bu\ Exercices : feuille n° 25, ex. 20 et 16.\vspace{.4cm}\\
% 
% \noindent\textbf{\bf Jeudi 08 juin 2019}\\
% \bu\ Exercices : feuille n° 25, ex. 12, 14, 15, 17, 19 et 21.\vspace{.4cm}\\
% 
% \noindent\textbf{\bf Mardi 06 juin 2019} \\
% \bu\ Interrogation n° 19.\\
% \bu\ Cours : 3 - Automorphismes orthogonaux (fin).\\
% \bu\ Exercices : feuille n° 25, ex. 9 et 10.\\
% \bu\ À faire pour jeudi 08 juin : feuille n° 25, ex. 21 et 22.\vspace{.4cm}\\
% 
% \noindent\textbf{Vendredi 02 juin 2019}\\
% \bu\ Cours : 3 - Automorphismes orthogonaux (suite).\\
% \bu\ Exercices : feuille n° 25, ex. 11 et 13.\vspace{.4cm}\\
% 
% \noindent\textbf{\bf Jeudi 01 juin 2019}\\
% \bu\ Distribution : DM n° 20 (à rendre le 08 juin).\\
% \bu\ Cours : 3 - Automorphismes orthogonaux (début).\\
% \bu\ Exercices : feuille n° 25, ex. 7 et 8.\vspace{.4cm}\\
% 
% \noindent\textbf{Mardi 30 mai 2019}\\
% \bu\ Cours : 2 - Orthogonalité (fin).\\
% \bu\ Exercices : feuille n° 25, ex. 5 et 6.\vspace{.4cm}\\
% 
% \noindent\textbf{\bf Lundi 29 mai 2019} \\
% \bu\ Interrogation n° 18.\\
% \bu\ Cours :  2 - Orthogonalité (suite).\\
% \bu\ Exercices : feuille n° 24, ex. 14 et 17, et feuille n° 25, ex. 1 à 4.\vspace{.4cm}\\
% 
% \noindent\textbf{Mardi 23 mai 2019}\\
% \bu\ Cours : \bf Chapitre XXIV \rm : Espaces vectoriels euclidiens : 1 - Produits scalaires, normes et distances 
% (fin).\\
% \bu\ Cours :  2 - Orthogonalité (début).\vspace{.4cm}\\
% 
% \noindent\textbf{\bf Lundi 22 mai 2019} \\
% \bu\ Interrogation n° 17.\\
% \bf Chapitre XXIV \rm : Espaces vectoriels euclidiens : 1 - Produits scalaires, normes et distances 
% (suite).\\
% \bu\ Exercices : feuille n° 24, ex. 10, 12 et 13.\vspace{.4cm}\\
% 
% \noindent\textbf{Vendredi 19 mai 2019}\\
% \bf Chapitre XXIV \rm : Espaces vectoriels euclidiens : 1 - Produits scalaires, normes et distances 
% (début).\\
% \bu\ Exercices : feuille n° 24, ex. 1 à 5.\vspace{.4cm}\\
% 
% \noindent\textbf{\bf Jeudi 18 mai 2019} \\
% \bu\ Distribution : DM n° 19 (à rendre le 01 juin).\\
% \bu\ Cours :  4 - Déterminant d'un endomorphisme.\\
% \bu\ Cours : 5 - Déterminant d'une matrice carrée.\vspace{.4cm}\\
% 
% \noindent\textbf{Mardi 16 mai 2019}\\
% 
% \noindent\textbf{\bf Lundi 15 mai 2019} \\

% \bu\ Exercices : feuille n° 23, ex. 18, 20, 21 et 19 (début).\vspace{.4cm}\\
% 
% \noindent\textbf{Vendredi 12 mai 2019}\\
% \bu\ Exercices : feuille n° 23, ex. 13, 14 et 17.\vspace{.4cm}\\
% 
% \noindent\textbf{\bf Jeudi 11 mai 2019} \\
% \bu\ Exercices : feuille n° 23, ex. 2, 7, 10 et 11.\vspace{.4cm}\\
% 
% \noindent\textbf{Mardi 09 mai 2019}\\
% \bu\ Exercices : feuille n° 2ex. 3 à 6, 8 et 9.\vspace{.4cm}\\
% 
% \noindent\textbf{\bf Jeudi 04 mai 2019} \\
% \bu\ Distribution : DM n° 18 (à rendre le 11 mai).\\
% 
% \noindent\textbf{Vendredi 01 juin 2019}\\
% \bu\ Exercices : feuille n° 23, ex. 12 à 17, et 19.\vspace{.4cm}\\
% 
% \noindent\textbf{Jeudi 31 mai 2019}\\
% \bu\ Cours : 3. Déterminant d’une famille de vecteurs.\\
% \bu\ Exercices : feuille n° 23, ex. 10.\vspace{.4cm}\\
% 
% \noindent\textbf{Mercredi 30 mai 2019}\\
% \bu\ Cours : 2. Applications multilinéaires.\\
% \bu\ Exercices : feuille n° 23, ex. 08, 09 et 11.\vspace{.4cm}\\
% 
% \noindent\textbf{Lundi 28 mai 2019}\\
% \bu\ Cours : \bf Chapitre XXIII \rm : Déterminants : 1 - Groupe symétrique.\\
% \bu\ Exercices : feuille n° 23, ex. 01, 03, 04, 05, 06 et 07.\\
% \bu\ À faire pour mercredi 30 : feuille n° 23, ex. 08 et 09.\\
% \bu\ À faire pour jeudi 31 : feuille n° 23, ex. 10.\\
% \bu\ À faire pour vendredi 01 : feuille n° 23, ex. 12 à 16.\vspace{.4cm}\\
% 
% \noindent\textbf{Vendredi 25 mai 2019}\\
% \bu\ Devoir surveillé n° 09.\\
% \bu\ Cours : 7 - Matrices semblables et trace ; 8 - Matrices par blocs.\\
% \bu\ Exercices : feuille n° 23, ex. 03.2).\\
% \bu\ À faire pour lundi 28 : feuille n° 23, ex. 01.\vspace{.4cm}\\
% 
% \noindent\textbf{Jeudi 24 mai 2019}\\
% \bu\ Cours : 5 - Rang ; 6 - Systèmes linéaires.\vspace{.4cm}\\
% 
% \noindent\textbf{Mercredi 23 mai 2019}\\
% \bu\ Cours : 4 - Opérations élémentaires.\\
% \bu\ Exercices : feuille n° 22, ex. 14 et feuille n° 23, ex. 02.\vspace{.4cm}\\
% 
% \noindent\textbf{Vendredi 18 mai 2019}\\
% \bu\ Cours : 3 - Matrices remarquables.\\
% \bu\ Exercices : feuille n° 22, ex. 13 et 16 à 18.\vspace{.4cm}\\
% 
% \noindent\textbf{Jeudi 17 mai 2019}\\
% \bu\ Distribution : DM n° 20 (à rendre le 24 mai).\\
% \bu\ Cours : 2 - Matrices, familles de vecteurs et applications linéaires (fin).\\
% \bu\ Exercices : feuille n° 22, ex. 15.\vspace{.4cm}\\
% 
% \noindent\textbf{Mercredi 16 mai 2019}\\
% \bu\ Cours : 2 - Matrices, familles de vecteurs et applications linéaires (début).\\
% \bu\ Exercices : feuille n° 22, ex. 12.\vspace{.4cm}\\
% 
% \noindent\textbf{Lundi 14 mai 2019}\\
% \bu\ Interrogation n° 20.\\
% \bu\ Cours : \bf Chapitre XXII \rm : Matrices ; 1 - Structure de $\mcal M_{n,p}(\K)$.\\
% \bu\ Exercices : feuille n° 22, ex. 08 à 11.\\
% \bu\ À faire pour mercredi 16 : feuille n° 22, ex. 12.\\
% \bu\ À faire pour jeudi 17 : feuille n° 22, ex. 15.\\
% \bu\ À faire pour vendredi 18 : feuille n° 22, ex. 17.\vspace{.4cm}\\
% 
% \noindent\textbf{Vendredi 11 mai 2019}\\
% \bu\ Cours : 2 - Variables aléatoires (fin).\\
% \bu\ Exercices : feuille n° 22, ex. 07.\vspace{.4cm}\\
% 
% \noindent\textbf{Mercredi 09 mai 2019}\\
% \bu\ Distribution : DM n° 19 (à rendre le 17 mai).\\
% \bu\ Cours : 2 - Variables aléatoires (suite).\\
% \bu\ Exercices : feuille n° 22, ex. 06.\vspace{.4cm}\\
% 
% \noindent\textbf{Lundi 07 mai 2019}\\
% \bu\ Cours : 2 - Variables aléatoires (suite).\\
% \bu\ Exercices : feuille n° 22, ex. 01 à 05.\vspace{.4cm}\\
%   
% \noindent\textbf{Vendredi 04 mai 2019}\\
% \bu\ Devoir surveillé n° 08.\\
% \bu\ Cours : 2 - Variables aléatoires (suite).\\
% \bu\ Exercices : feuille n° 21, ex. 16 à 20.\vspace{.4cm}\\
% 
% \noindent\textbf{Jeudi 03 mai 2019}\\
% \bu\ Cours : 2 - Variables aléatoires (début).\\
% \bu\ Exercices : feuille n° 21, ex. 13 et 15.\vspace{.4cm}\\
% 
% \noindent\textbf{Mercredi 02 mai 2019}\\
% \bu\ Cours : 1 - Événements (fin).\\
% \bu\ Exercices : feuille n° 21, ex. 03, 11, 12 et 14.\vspace{.4cm}\\
% 
% \noindent\textbf{Lundi 30 avril 2019}\\
% \bu\ Interrogation surprise n° 19.\\
% \bu\ Cours : Événements (suite).\\
% \bu\ Exercices : feuille n° 21, ex. 01, 02 et 04 à 09.\\
% \bu\ À faire pour mercredi 02 : feuille n° 21, ex. 03.\\
% \bu\ À faire pour jeudi 03 : feuille n° 21, ex. 10.\\
% \bu\ À faire pour vendredi 04 : feuille n° 21, ex. 13.\vspace{.4cm}\\
%   
% \noindent\textbf{Vendredi 27 avril 2019}\\
% \bf Chapitre XXII \rm : Probabilités : 1 - Événements (début).\\
% \bu\ Exercices : feuille n° 20, ex. 03, 05 et 07.\vspace{.4cm}\\
% 
% \noindent\textbf{Jeudi 26 avril 2019}\\
% \bu\ Cours : 4 - Formes linéaires et hyperplans.\\
% \bu\ Exercices : feuille n° 20, ex. 08.\vspace{.4cm}\\
% 
% \noindent\textbf{Mercredi 25 avril 2019}\\
% \bu\ Distribution : DM n° 18 (à rendre le 03 mai).\\
% \bu\ Cours : 2 - Sev en dimension finie (fin) ; 3 - Applications linéaires en dimension finie.\\
% \bu\ Exercices : feuille n° 20, ex. 01 et 04.\vspace{.4cm}\\
% 
% \noindent\textbf{Lundi 23 avril 2019}\\
% \bu\ Interrogation surprise n° 18.\\
% \bu\ Cours : 2 - Sev en dimension finie (début).\\
% \bu\ Exercices : feuille n° 20, ex. 02 et 06.\\
% \bu\ À faire pour mercredi 25 : feuille n° 20, ex. 04.\\
% \bu\ À faire pour jeudi 26 : feuille n° 20, ex. 08.\\
% \bu\ À faire pour vendredi 27 : feuille n° 20, ex. 05.\vspace{.4cm}\\
% 
% \noindent\textbf{\bf Vacances de printemps}.\vspace{.4cm}\\
%   
% \noindent\textbf{Vendredi 06 avril 2019}\\
% \bu\ Exercices : feuille n° 19, ex. 10, 11 et 26.\vspace{.4cm}\\
% 
% \noindent\textbf{Jeudi 05 avril 2019}\\
% \bu\ Cours : 1 - Notion de dimension (fin).\\
% \bu\ Exercices : feuille n° 19, ex. 16 à 19, et 25.\vspace{.4cm}\\
% 
% \noindent\textbf{Mercedi 04 avril 2019}\\
% \bu\ Distribution : DM n° 17 (à rendre le 26 avril).\\
% \bf Chapitre XXI \rm : Espaces vectoriels de dimension finie  : 1 - Notion de dimension (début).\\
% \bu\ Exercices : feuille n° 19, ex. 05 à 08.\vspace{.4cm}\\
% 
% \noindent\textbf{Vendredi 30 mars 2019}\\
% \bu\ Cours : Dénombrement (fin).\\
% \bu\ Devoir surveillé n° 07.\vspace{.4cm}\\
% 
% \noindent\textbf{Jeudi 29 mars 2019}\\
% \bu\ Cours : \bf Chapitre XX \rm : Dénombrement (début).\\
% \bu\ Exercices : feuille n° 19, ex. 01, 02, 04 et 09.\vspace{.4cm}\\
% 
% \noindent\textbf{Mercredi 28 mars 2019} \\
% \bu\ Cours : Intégration (fin).\vspace{.4cm}\\
% 
% \noindent\textbf{\bf Lundi 26 mars 2019} \\
% \bu\ Interrogation surprise n° 17.\\
% \bu\ Cours : Intégration (suite).\\
% \bu\ Exercices : feuille n° 18, ex. 16, 18, 20, 21, 22 et 23, et feuille n° 19, ex. 03.\vspace{.4cm}\\
% 
% \noindent\textbf{Vendredi 23 mars 2019}\\
% \bf Chapitre XX \rm : Intégration (début).\vspace{.4cm}\\
% 
% \noindent\textbf{Jeudi 22 mars 2019}\\
% \bu\ Distribution : DM n° 16 (à rendre le 29 mars).\\
% \bu\ Cours : Espaces vectoriels : 6 - Endomorphismes particuliers.\\
% \bu\ Exercices : feuille n° 18, ex. 15 et 17.\vspace{.4cm}\\
% 
% \noindent\textbf{Mercredi 21 mars 2019} \\
% \bu\ Cours : Espaces vectoriels : 5 - Familles de vecteurs (fin).\\
% \bu\ Exercices : feuille n° 18, ex. 12 à 14.\vspace{.4cm}\\
% 
% \noindent\textbf{\bf Lundi 19 mars 2019} \\
% \bu\ Interrogation surprise n° 16.\\
% \bu\ Cours : Espaces vectoriels : 5 - Familles de vecteurs (suite).\\
% \bu\ Interrogation surprise n° 16.\\
% \bu\ Exercices : feuille n° 18, ex. 07 à 11.\vspace{.4cm}\\
% 
% \noindent\textbf{Vendredi 16 mars 2019}\\
% \bu\ Cours : Espaces vectoriels : 4 - Applications linéaires (fin) ; 5 - Familles de vecteurs (début).\\
% \bu\ Exercices : feuille n° 18, ex. 02, 05 et 06.\vspace{.4cm}\\
% 
% \noindent\textbf{Jeudi 15 mars 2019}\\
% \bu\ Distribution : DM n° 15 (à rendre le 22 mars).\\
% \bu\ Cours : Espaces vectoriels : 4 - Applications linéaires (début).\\
% \bu\ Exercices : feuille n° 17, ex. 30 à 32, et feuille n° 18, ex. 01 et 03.\vspace{.4cm}\\
% 
% \noindent\textbf{Mercredi 14 mars 2019} \\
% \bu\ Cours : Espaces vectoriels : 3 - Sea (fin).\\
% \bu\ Exercices : feuille n° 17, ex. 28 et 29.\vspace{.4cm}\\
% 
% \noindent\textbf{\bf Lundi 12 mars 2019} \\
% \bu\ Interrogation surprise n° 15.\\
% \bu\ Cours : Espaces vectoriels : 2 - Sev (fin) ; 3 - Sea (début).\\
% \bu\ Exercices : feuille n° 17, ex. 12, 13, 16, 20, 21 (début), 25 et 27.\\
% \bu\ À faire pour mercredi 14 : feuille n° 18, ex. 01.\\
% \bu\ À faire pour jeudi 15 : feuille n° 18, ex. 03.\\
% \bu\ À faire pour vendredi 16 : feuille n° 18, ex. 05.\vspace{.4cm}\\
% 
% \noindent\textbf{Vendredi 09 mars 2019}\\
% \bu\ Cours : Espaces vectoriels : 2 - Sev (suite).\\
% \bu\ Exercices : feuille n° 17, ex. 07.\vspace{.4cm}\\
% 
% \noindent\textbf{Jeudi 08 mars 2019}\\
% \bu\ Distribution : DM n° 14 (à rendre le 15 mars).\\
% \bu\ Cours : Espaces vectoriels : 2 - Sev (suite).\\
% \bu\ Exercices : feuille n° 17, ex. 08 et 11.\vspace{.4cm}\\
% 
% \noindent\textbf{Mercredi 07 mars 2019} \\
% \bu\ Cours : Espaces vectoriels : 2 - Sev (début).\\
% \bu\ Exercices : feuille n° 17, ex. 06, 09 et 10.\vspace{.4cm}\\
% 
% \noindent\textbf{\bf Lundi 05 mars 2019} \\
% \bu\ Interrogation surprise n° 14.\\
% \bu\ Cours : \bf Chapitre XVIII \rm : Espaces vectoriels : 1 - Ev.\\
% \bu\ Exercices : feuille n° 16, ex. 07 et 08, et feuille n° 17, ex. 01 à 04.\\
% \bu\ À faire pour mercredi 07 : feuille n° 17, ex. 06.\\
% \bu\ À faire pour jeudi 08 : feuille n° 17, ex. 07.\\
% \bu\ À faire pour vendredi 09 : feuille n° 17, ex. 08.\vspace{.4cm}\\
% 
% \vspace{.4cm}\\ 
% 
% \noindent\textbf{Vendredi 02 mars 2019}\\
% \bu\ Devoir surveillé n° 06.\\
% \bu\ Cours : Analyse asymptotique (fin).\vspace{.4cm}\\
% 
% \noindent\textbf{Jeudi 01 mars 2019}\\
% \bu\ Cours : Analyse asymptotique (suite).\\
% \bu\ Exercices : feuille n° 16, ex. 05 et 06.\vspace{.4cm}\\
% 
% \noindent\textbf{Mercredi 28 février 2019} \\
% \bu\ Cours : Analyse asymptotique (suite).\\
% \bu\ Exercices : feuille n° 16, ex. 01 à 03.\vspace{.4cm}\\
% 
% \noindent\textbf{Mardi 27 février 2019} \\
% \bu\ Cours : Analyse asymptotique (suite).\vspace{.4cm}\\
% 
% \noindent\textbf{\bf Lundi 26 février 2019} \\
% \bu\ Exercices : feuille n° 15, ex. 12 à 18.\\
% \bu\ À faire pour jeudi 01 : feuille n° 16, ex. 05 et 06.\vspace{.4cm}\\
% 
% \noindent\textbf{\bf Vacances d'hiver }\\
% 
% \noindent\textbf{Vendredi 09 février 2019}\\
% \bu\ Cours : Analyse asymptotique (suite).\\
% \bu\ Exercices : feuille n° 15, ex. 07, 10 et 11.\vspace{.4cm}\\
% 
% \noindent\textbf{Jeudi 08 février 2019}\\
% \bu\ Distribution : DM n° 13 (à rendre le 01 mars).\\
% \bf Chapitre XVII \rm : Analyse asymptotique (début).\\
% \bu\ Exercices : feuille n° 15, ex. 06, 08 et 09.\vspace{.4cm}\\
% 
% \noindent\textbf{Mercredi 07 février 2019} \\
% \bu\ Cours : Fractions rationnelles (fin).\\
% \bu\ Exercices : feuille n° 15, ex. 01, 03 et 05.\vspace{.4cm}\\
% 
\noindent\textbf{\bf Lundi 04 février 2019} \\
\bu\ Interrogation surprise n° 12.\\
\bu\ Cours :  Dérivabilité (fin) ;\\
\bf Chapitre XVI \rm : Fractions rationnelles (début).\\
\bu\ Exercices : feuille n° 14, ex. 05, 13, 14 et 19, et feuille n° 15, ex. 02, 03, 05, 07 et 08.\\
\bu\ À faire pour mercredi 06 : feuille n° 15, ex. 09.\\
\bu\ À faire pour jeudi 07 : feuille n° 15, ex. 10.\\
\bu\ À faire pour vendredi 08 : feuille n° 15, ex. 12.\vspace{.4cm}\\
 
\noindent\textbf{Vendredi 01 février 2019}\\
\bu\ Cours : Dérivabilité (suite).\\
\bu\ Exercices : feuille n° 14, ex. 19.\vspace{.4cm}\\
 
\noindent\textbf{Jeudi 31 janvier 2019}\\
\bu\ Distribution : DM n° 12 (à rendre le 08 février).\\
\bu\ Cours : Dérivabilité (suite).\\
\bu\ Exercices : feuille n° 14, ex. 10, 17 et 18.\vspace{.4cm}\\
 
\noindent\textbf{Mercredi 30 janvier 2019} \\
\bu\ Cours : Dérivabilité (suite).\\
\bu\ Exercices : feuille n° 14, ex. 11, 15 et 16.\vspace{.4cm}\\
 
\noindent\textbf{Lundi 28 janvier 2019} \\
\bu\ Cours : \bf Chapitre XV \rm : Dérivabilité (début).\\
\bu\ Exercices : feuille n° 14, ex. 01 à 04, 06 à 10 et 12.\\
\bu\ À faire pour mercredi 30 : feuille n° 14, ex. 11.\\
\bu\ À faire pour jeudi 31 : feuille n° 14, ex. 17.\\
\bu\ À faire pour vendredi 25 : feuille n° 14, ex. 05, 13, 14 et 19.\vspace{.4cm}\\
 
\noindent\textbf{Vendredi 25 janvier 2019}\\
\bu\ Devoir surveillé n° 05.\\
\bu\ Cours : Polynômes (fin).\\
\bu\ Exercices : feuille n° 13, ex. 13 et 14.\vspace{.4cm}\\

\noindent\textbf{Mercredi 23 janvier 2019} \\
\bu\ Cours : Polynômes (suite).\\
\bu\ Exercices : feuille n° 13, ex. 07 et 08.\\
\bu\ À faire pour lundi 28 : feuille n° 14, ex. 01.\vspace{.4cm}\\
 
\noindent\textbf{Mardi 22 janvier 2019} \\
\bu\ Cours : Polynômes (suite).\\
\bu\ Exercices : feuille n° 13, ex. 03 et 07 (début).\\
\bu\ À faire pour mercredi 23 : feuille n° 13, ex. 07 (fin).\\
\bu\ À faire pour vendredi 25 : feuille n° 13, ex. 13.\vspace{.4cm}\\

\noindent\textbf{Lundi 21 janvier 2019} \\
\bu\ Cours : Polynômes (suite).\\
\bu\ Exercices : feuille n° 13, ex. 04, 05, 09, 10, 11 et 12.\vspace{.4cm}\\

\noindent\textbf{Vendredi 17 janvier 2019}\\
\bu\ Cours : Polynômes (suite).\\
\bu\ Exercices : feuille n° 13, ex. 02.\\
\bu\ À faire pour lundi 21 : feuille n° 13, ex. 04.\vspace{.4cm}\\

\noindent\textbf{Mercredi 16 janvier 2019} \\
\bu\ Distribution : DM n° 11 (à rendre le 25 janvier).\\
\bu\ Cours : Polynômes (suite).\\
\bu\ Exercices : feuille n° 13, ex. 01.\vspace{.4cm}\\

\noindent\textbf{Mardi 15 janvier 2019} \\
\bu\ Cours : Polynômes (suite).\\
\bu\ Exercices : feuille n° 12, ex. 01.\\
\bu\ À faire pour mercredi 16 : feuille n° 13, ex. 01.\\
\bu\ À faire pour vendredi 18 : feuille n° 13, ex. 02.\vspace{.4cm}\\
 
\noindent\textbf{Lundi 14 janvier 2019} \\
\bu\ Interrogation surprise n° 10.\\
\bu\ \bf Chapitre XIV \rm : Polynômes (début).\\
\bu\ Exercices : feuille n° 12, ex. 02 à 07.\vspace{.4cm}\\

\noindent\textbf{Vendredi 11 janvier 2019}\\
\bu\ Cours : Continuité (fin).\\
\bu\ Exercices : feuille n° 11, ex. 11 et 12.\\
\bu\ À faire pour lundi 15 : feuille n° 12, ex. 01.\vspace{.4cm}\\
 
\noindent\textbf{Jeudi 10 janvier 2019}\\
\bu\ Cours : Continuité (suite).\\
\bu\ Distribution : DM n° 10 (à rendre le 18 janvier).\\
\bu\ Exercices : feuille n° 11, ex. 08 et 09.\vspace{.4cm}\\
 
\noindent\textbf{Mercredi 09 janvier 2019} \\
\bu\ \bf Chapitre XIII \rm : Continuité (début).\\
\bu\ Exercices : feuille n° 11, ex. 07.\vspace{.4cm}\\
 
\noindent\textbf{Lundi 07 janvier 2019} \\
% \bu\ Interrogation surprise n° 10.\\
\bu\ Cours : Limites d'une fonction (fin).\\
\bu\ Exercices : feuille n° 11, ex. 01, 03, 04, 05, 06 et 10.\vspace{.4cm}\\

\noindent\textbf{\bf Vacances de Noël}.\vspace{.4cm}\\

\noindent\textbf{Vendredi 22 décembre 2018}\\ 
\bu\ Cours : Limites d'une fonction (suite).\\
\bu\ Exercices : feuille n° 10, ex. 16, 17 et feuille n° 11, ex. 02.\vspace{.4cm}\\
 
\noindent\textbf{Jeudi 21 décembre 2018}\\
\bu\ Cours : \bf Chapitre XII \rm : Limites d'une fonction (début).\\
\bu\ Exercices : feuille n° 10, ex. 13, 14 et 15.\vspace{.4cm}\\
 
\noindent\textbf{\bf Mercredi 20 décembre 2018}\\
\bu\ Cours : Anneaux et corps.\\ 
\bu\ Exercices : feuille n° 10, ex. 12.\vspace{.4cm}\\
 
\noindent\textbf{Lundi 17 décembre 2018}\\
\bu\ Interrogation surprise n° 09.\\
\bu\ Cours : 2 - Groupes (fin).\\ 
\bu\ Exercices : feuille n° 10, ex. 08 et 09.\\
\bu\ À faire pour mercredi 20 : feuille n° 10, ex. 12.\\
\bu\ À faire pour jeudi 21 : feuille n° 10, ex. 13.\\
\bu\ À faire pour vendredi 22 : feuille n° 10, ex. 16.\vspace{.4cm}\\ 

\noindent\textbf{Vendredi 14 décembre 2018}\\
\bu\ Cours : \bf Chapitre XI \rm : Groupes, anneaux, corps  : 1 - Lci ; 2 - Groupes (début).\\ 
\bu\ Exercices : feuille n° 10, ex. 06.\vspace{.4cm}\\
 
\noindent\textbf{Jeudi 13 décembre 2018}\\
\bu\ Cours : 7 - Suites complexes ; 8 - Premières séries numériques.\\
\bu\ Exercices : feuille n° 10, ex. 03, 04, 07 et 11.\vspace{.4cm}\\
 
\noindent\textbf{\bf Mercredi 12 décembre 2018}\\
\bu\ Distribution : DM n° 09 (à rendre le 20 décembre).\\
\bu\ Cours : Suites récurrentes.\\
\bu\ À faire pour jeudi 13 : feuille n° 10, ex. 07.\\
\bu\ À faire pour vendredi 14 : feuille n° 10, ex. 06.\vspace{.4cm}\\

\noindent\textbf{Lundi 10 décembre 2018}\\
\bu\ Cours : 3 - Résultats de convergence (fin) ; 4 - Traduction séquentielle de certaines propriétés ; 5 - Suites particulières.\\
\bu\ Exercices : feuille n° 10, ex. 01, 02 et 05.\vspace{.4cm}\\
 
\noindent\textbf{Vendredi 07 décembre 2018}\\
\bu\ Devoir surveillé n° 04.\\
\bu\ Cours : 3 - Résultats de convergence (suite).\\
\bu\ Exercices : feuille n° 09, ex. 14 et 17.\\
\bu\ À faire pour lundi 03 : feuille n° 10, ex. 05.\vspace{.4cm}\\
 
\noindent\textbf{Jeudi 06 décembre 2018}\\
\bu\ Cours : 3 - Résultats de convergence (suite).\\
\bu\ Exercices : feuille n° 09, ex. 06 (fin), 10, 11, 12, 13, 15, 16, 18 et 19.\vspace{.4cm}\\
 
\noindent\textbf{\bf Mercredi 05 décembre 2018}\\
\bu\ Cours : 3 - Résultats de convergence (début).\\
\bu\ Exercices : feuille n° 09, ex. 08 et 06 (début).\vspace{.4cm}\\
 
\noindent\textbf{Lundi 03 décembre 2018}\\
\bu\ Interrogation surprise n° 08.\\
\bu\ Cours : 2 - Limite d'une suite réelle (fin).\\
\bu\ Exercices : feuille n° 08, ex. 07, 10 et 11, feuille n° 09, ex. 01, 02, 03, 04, 05 et 07.\\
\bu\ À faire pour mercredi 06 : feuille n° 09, ex. 08.\\
\bu\ À faire pour jeudi 07 : feuille n° 09, ex. 11.\\
\bu\ À faire pour vendredi 08 : feuille n° 09, ex. 14.\vspace{.4cm}\\
 
\noindent\textbf{Vendredi 30 novembre 2018}\\
\bu\ Cours : 2 - Limite d'une suite réelle (début).\\
\bu\ Exercices : feuille n° 08, ex. 08, 09 et 12.\\
\bu\ À faire pour lundi 03 : feuille n° 08, ex. 07.\vspace{.4cm}\\
 
\noindent\textbf{Jeudi 29 novembre 2018}\\
\bu\ Distribution : DM n° 08 (à rendre le 06 décembre).\\
\bf Chapitre X :\rm Suites réelles et complexes : 1 - Vocabulaire.\\
\bu\ Exercices : feuille n° 08, ex. 02 et 04.\vspace{.4cm}\\
 
\noindent\textbf{\bf Mercredi 28 novembre 2018}\\
\bu\ Cours : 3 - Nombres premiers (fin).\\
\bu\ Exercices : feuille n° 08, ex. 06.\vspace{.4cm}\\
 
\noindent\textbf{Lundi 26 novembre 2018}\\
\bu\ Interrogation surprise n° 07.\\
\bu\ Cours : 2 - PPCM ; 3 - Nombres premiers (début).\\
\bu\ Exercices : feuille n° 07, ex. 10 (fin), et feuille n° 08, ex. 01, 03 et 05.\\
\bu\ À faire pour mercredi 28 : feuille n° 08, ex. 06.\\
\bu\ À faire pour jeudi 29 : feuille n° 08, ex. 04.\\
\bu\ À faire pour vendredi 30 : feuille n° 08, ex. 08.\vspace{.4cm}\\
 
\noindent\textbf{Vendredi 24 novembre 2018}\\
\bu\ Cours : 2 - PGCD (fin).\\
\bu\ Exercices : feuille n° 07, ex. 10 (début).\vspace{.4cm}\\
 
\noindent\textbf{Jeudi 23 novembre 2018}\\
\bu\ Distribution : DM n° 07 (à rendre le 29 novembre).\\
\bu\ Cours : 1 - Divisibilité (fin) ; 2 - PGCD (début).\\
\bu\ Exercices : feuille n° 07, ex. 07 et 09.\vspace{.4cm}\\
 
\noindent\textbf{\bf Mercredi 22 novembre 2018}\\
\bf Chapitre IX \rm : Arithmétique : 1 - Divisibilité (début).\\
\bu\ Exercices : feuille n° 07, ex. 08 et 11.\vspace{.4cm}\\
 
\noindent\textbf{Lundi 20 novembre 2018}\\
\bu\ Cours : 6 - La relation d'ordre naturelle sur \R\ (fin).\\
\bu\ Exercices : feuille n° 07, ex. 03, 04, 05 et 06.\vspace{.4cm}\\

\noindent\textbf{Vendredi 16 novembre 2018}\\
\bu\ Devoir surveillé n° 03.\\
\bu\ Cours : 5 - La relation d'ordre naturelle sur \N\ ; 6 - La relation d'ordre naturelle sur \R\ (début).\\
\bu\ Exercices : feuille n° 06, ex. 08.\vspace{.4cm}\\
 
\noindent\textbf{Jeudi 15 novembre 2018}\\
\bu\ Cours : 3 - Relations d'ordre ;  4 - Majorants, minorants et compagnie.\\
\bu\ Exercices : feuille n° 06, ex. 07 et feuille n° 07, ex. 01 et 02.\vspace{.4cm}\\

\noindent\textbf{\bf Mercredi 14 novembre 2018}\\
\bf Chapitre VIII \rm : Relations d'ordre : 1 - Relations binaires ; 2 - Relations d'équivalence.\\
\bu\ Exercices : feuille n° 06, ex. 05.\vspace{.4cm}\\
 
\noindent\textbf{Lundi 12 novembre 2018}\\
\bu\ Interrogation surprise n° 06.\\
\bu\ Cours : 4 - Injectivité, surjectivité, bijectivité ; 5 - Images directe et réciproque.\\
\bu\ Exercices : feuille n° 06, ex. 01 à 04.\\
\bu\ À faire pour mercredi 14 : ex. 5.2.5 du chapitre de cours VII.\\
\bu\ À faire pour jeudi 15 : feuille n° 06, ex. 07.\\
\bu\ À faire pour vendredi 16 : feuille n° 06, ex. 08.\vspace{.4cm}\\

\noindent\textbf{Vendredi 09 novembre 2018}\\
\cours\ 2 - Restriction et prolongement ; 3 - Composition ; 4 - Injectivité.\\
\bu\ Exercices : feuille n° 05, ex. 08 et 09.\vspace{.4cm}\\

\noindent\textbf{Jeudi 08 novembre 2018}\\
\bu\ Distribution du DM n° 06 (à rendre le 15 novembre).\\
\bu\ Cours : Théorie des ensembles (fin).\\
\bf Chapitre VII \rm : Notion d'application. 1 - Vocabulaire.\\
\bu\ Exercices : feuille n° 05, ex. 06.\\
\bu\ À faire pour vendredi 09 : feuille n° 05, ex. 08.\vspace{.4cm}\\
 
\noindent\textbf{\bf Mercredi 07 novembre 2018}\\
\bu\ Cours : Théorie des ensembles (suite).\\
\bu\ Cours : Théorie des ensembles (fin).\\
\bu\ Exercices : feuille n° 05, ex. 07.\vspace{.4cm}\\

\noindent\textbf{Lundi 05 novembre 2018}\\
\bu\ Interrogation surprise n° 05.\\
\bu\ Cours : \bf Chapitre VI \rm : Théorie des ensembles (début).\\
\bu\ Exercices : feuille n° 05, ex. 02, 04 et 10.\vspace{.4cm}\\

\noindent\textbf{ Vacances de novembre }\vspace{.4cm}\\

\noindent\textbf{Vendredi 19 octobre 2018}\\
\bu\ Cours : 4 - Équations différentielles du second ordre (fin) ; 5 - Circuits RL et RLC.\\
\bu\ Exercices : feuille n° 04, ex. 26.\\
\bu\ À faire pour lundi 05 : feuille n° 05, ex. 04 et 07.\vspace{.4cm}\\

\noindent\textbf{Jeudi 18 octobre 2018}\\
\bu\ Cours : 3.2 - Équations différentielles du premier ordre avec second membre ; 4.1 - Équations 
différentielles homogènes du second ordre (début)
\bu\ Exercices : feuille n° 05, ex. 03.\vspace{.4cm}\\
 
\noindent\textbf{\bf Mercredi 17 octobre 2018}\\
\bu\ Cours : 2 - Généralités (fin) ; 3.1 - Équations différentielles homogène du premier ordre.\\
\bu\ Exercices : feuille n° 04, ex. 26 et feuille n° 05, ex. 01 (début).\vspace{.4cm}\\
 
\noindent\textbf{Lundi 15 octobre 2018}\\
\bu\ Distribution du DM n° 05 (à rendre le 08 novembre).\\
\bu\ Cours : 1 - Résultats d'analyse relatifs aux fonctions à valeurs complexes d'une variable réelle, et intégration 
(fin) ; 2 - Généralités (début).\\
\bu\ Exercices : feuille n° 04, ex. 13, 18, 19 et 21 à 25, et feuille n° 05, ex. 01 (début).\\
\bu\ À faire pour mercredi 17 : feuille n° 04, ex. 26.\\
\bu pour les autres jours : regarder les exercices de la feuille supplémentaire sur l'intégration.\vspace{.4cm}\\
 
\noindent\textbf{Vendredi 12 octobre 2018}\\
\bu\ Devoir surveillé n° 02.\\
\bu\ Cours : 1 - Résultats d'analyse relatifs aux fonctions à valeurs complexes d'une variable réelle, et intégration 
(suite).\\
\bu\ Exercices : feuille n° 04, ex. 14 à 17, et 20.\\
\bu\ À faire pour lundi 15 : feuille n° 04, ex. 18 et 19.\vspace{.4cm}\\
 
\noindent\textbf{Jeudi 11 octobre 2018}\\
\bu\ Cours : \bf Chapitre V \rm : Équations différentielles : 1 - Résultats d'analyse relatifs aux fonctions à valeurs complexes d'une variable réelle, et intégration (début).\\
\bu\ Exercices : feuille n° 04, ex. 05 et 08 à 12.\vspace{.4cm}\\
 
\noindent\textbf{\bf Mercredi 10 octobre 2018}\\
\bu\ Cours : 5 - Nombres complexes et géométrie plane (fin).\\
\bu\ Exercices : feuille n° 04, ex. 05 (début).\vspace{.4cm}\\
 
\noindent\textbf{\bf Lundi 08 octobre 2018}\\
\bu\ Interrogation surprise n° 04.\\ 
\bu\ Cours : 4 - Techniques de calcul ; 5 - Nombres complexes et géométrie plane (début).\\
\bu\ Exercices : feuille n° 03, ex. 10 à 12 et feuille n° 04, ex. 01 à 04, 06 et 07.\\
\bu\ À faire pour mercredi 10 : feuille n° 04, ex. 05 et 08.\\
\bu\ À faire pour jeudi 11 : feuille n° 04, ex. 12.\\
\bu\ À faire pour vendredi 12 : feuille n° 04, ex. 17.\vspace{.4cm}\\
 
\noindent\textbf{Vendredi 05 octobre 2018}\\
\bu\ Cours : 3 - Équations du second degré.\\
\bu\ Exercices : feuille n° 03, ex. 01, 02, 04, 05, 08 et 09.\\
\bu\ À faire pour lundi 08 : feuille n° 03, ex. 10 à 12.\vspace{.4cm}\\
% 
\noindent\textbf{Jeudi 04 octobre 2018}\\
\bu\ Distribution du DM n° 04 (à rendre le 11 octobre).\\
\bu\ Cours : 2 - Le groupe \U\ des nombres complexes de module 1 (fin).
\bu\ Exercices : feuille n° 02, ex. 20, et feuille n° 03, ex. 03, 06 et 07.\vspace{.4cm}\\

\noindent\textbf{\bf Mercredi 03 octobre 2018}\\
\bu\ Cours : 1 - Construction de \C\ (fin) ; 2 - Le groupe \textbf{U} des nombres 
complexes de module 1 (début).\\
\bu\ Exercices : feuille n° 02, ex. 15 (fin).\vspace{.4cm}\\

\noindent\textbf{\bf Lundi 01 octobre 2018}\\
\bu\ Interrogation surprise n° 03.\\
\bu\ Cours : \bf Chapitre IV \rm : Le corps des complexes : 1 - Construction de \C\ (début).\\
\bu\ Exercices : feuille n° 02, ex. 14 à 17 et 19, et feuille n° 03, ex. 03 et 07.\\
\bu\ À faire pour mercredi 03 : feuille n° 02, ex. 15 (fin).\\
\bu\ À faire pour jeudi 04 : feuille n° 02, ex. 20.\\
\bu\ À faire pour vendredi 05 : feuille n° 03, ex. 01, 02, 04 et 05.\vspace{.4cm}\\ 

\noindent\textbf{Vendredi 28 septembre 2018}\\
\bu\ Cours : Quelques fondamentaux (fin).\\
\bu\ Exercices : feuille n° 02, ex. 05 et 10 à 12.\\
\bu\ À faire pour lundi 02 : récurrences erronées.\vspace{.4cm}\\

\noindent\textbf{Jeudi 27 septembre 2018}\\
\bu\ Cours : \bf Chapitre III \rm : Quelques fondamentaux (suite).\\
\bu\ Exercices : feuille n° 02, ex. 02 et 06 à 09.\vspace{.4cm}\\

\noindent\textbf{\bf Mercredi 26 septembre 2018}\\
\bu\ Distribution du DM n° 03 (à rendre le 04 octobre).\\
\bu\ Cours : \bf Chapitre III \rm : Quelques fondamentaux (début).\vspace{.4cm}\\

\noindent\textbf{\bf Lundi 24 septembre 2018}\\
\bu\ Cours : 4 - Matrices (fin) ; 5 - Systèmes linéaires.\\
\bu\ Exercices : feuille n° 01, ex. 01, 03 et 04.\\
\bu\ À faire pour mercredi 26 : feuille n° 02, ex. 06 et 07.\\
\bu\ À faire pour jeudi 27 : feuille n° 02, ex. 02.\\
\bu\ À faire pour vendredi 28 : feuille n° 02, ex. 05.\vspace{.4cm}\\ 

\noindent\textbf{Vendredi 21 septembre 2018}\\
\bu\ Devoir surveillé n° 01.\\
\bu\ Cours : 4 - Matrices (début).\\
\bu\ Exercices : feuille n° 01, ex. 16 et 17.\vspace{.4cm}\\
 
\noindent\textbf{Jeudi 20 septembre 2018}\\
\bu\ Cours : 3 - Quelques formules (fin).\\
\bu\ Exercices : feuille n° 01, ex. 19.\vspace{.4cm}\\

\noindent\textbf{\bf Mercredi 19 septembre 2018}\\
\bu\ Cours : 3 - Quelques formules (suite).\\
\bu\ Exercices : feuille n° 01, ex. 15 et 18.\vspace{.4cm}\\

\noindent\textbf{\bf Lundi 17 septembre 2018}\\ 
\bu\ Interrogation surprise n° 02.\\
\bu\ 1 - Sommes (fin) ; 2 - Produits ; 3 - Quelques formules (début).\\
\bu\ Exercices : feuille n° 01, ex. 12 à 14.\\
\bu\ À faire pour mercredi 19 : feuille n° 01, ex. 18.\\
\bu\ À faire pour jeudi 20 : feuille n° 01, ex. 19.\\
\bu\ À faire pour vendredi 21 : feuille n° 01, ex. 16.\vspace{.4cm}\\

\noindent\textbf{Vendredi 14 septembre 2018}\\
\bu\ Cours : 9 - Fonctions hyperboliques.\\
\bu\ \textbf{Chapitre II} \rm : Un peu de calcul. 1 - Sommes (début).\\
\bu\ À faire pour lundi 10 : feuille n° 01, ex. 10 et 11.\vspace{.4cm}\\

\noindent\textbf{\bf Jeudi 13 septembre 2018}\\
\bu\ Cours : 7 - Fonctions circulaires ; 8 - Fonctions circulaires inverses.\\
\bu\ Exercices : feuille n° 01, ex. 9.\vspace{.4cm}\\
    
\noindent\textbf{\bf Mercredi 12 septembre 2018}\\
\bu\ Cours : 6 - Exponentielle et logarithme.\\
\bu\ Distribution du DM n° 02 (à rendre le 20 septembre).\\
\bu\ Exercices : feuille n° 01, ex. 7.\vspace{.4cm}\\

\noindent\textbf{\bf Lundi 10 septembre 2018}\\
\bu\ Cours : 5 - Puissances.\\
\bu\ Interrogation surprise n° 01.\\
\bu\ Exercices : feuille n° 01, ex. 1 à 6, et 8.\vspace{.4cm}\\

\noindent\textbf{Vendredi 07 septembre 2018}\\
\bu\ Cours : 3 - Réciproques ; 4 - Fonction valeur absolue.\\
\bu\ Exercices : feuille n° 01, ex. 3 (fin).\\
\bu\ À faire pour lundi 10 : feuille n° 01, ex. 1.\vspace{.4cm}\\

\noindent\textbf{\bf Jeudi 06 septembre 2018}\\
\bu\ Cours : 1 - Vocabulaire (fin) ; 2 - Effet d'une transformation sur le graphe ; 3 - Composée de fonctions (début).\\
\bu\ Exercices : feuille n° 01, ex. 2 et 3 (début).\\
\bu\ À faire pour vendredi 07 : feuille n° 01, ex. 3.\vspace{.4cm}\\
    
\noindent\textbf{\bf Mercredi 05 septembre 2018}\\
\bu\ Cours : \bf Chapitre I \rm : Fonctions usuelles : 1 - Vocabulaire (début).\vspace{.4cm}\\
 
\noindent\textbf{\bf Mardi 04 septembre 2018}\\
Journée de rentrée ; distribution des feuilles de TD, des formulaires, des
chapitres I et II, et du DM n° 01 (à rendre le 13 septembre).\vspace{.4cm}\\


\label{end}
\end{document}


