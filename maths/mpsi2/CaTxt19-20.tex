\documentclass[12pt,a4paper]{article}

\textheight=25cm
\topmargin=-50pt
\input{/home/skanderk/.latex/intro2.sty}

\begin{document}

\begin{center}
\Large\bf CAHIER DE TEXTES DE MATHÉMATIQUES\\
MPSI 2 La Martinière Monplaisir\\ 2020-2020
\end{center}
\vspace{1cm}
\vspace{.4cm}

% 
% \noindent\textbf{\bf Jeudi 18 juin 2019} \\
% 
% 
% \noindent\textbf{\bf Mercredi 17 juin 2019} \\
% 
% \noindent\textbf{\bf Lundi 15 juin 2019} \\
% \bu\ Interrogation n° 24.\\
% \bu\ Cours : \bf Chapitre XXV \rm : Dénombrement : fin.\\
% \bu\ Exercices : feuille n° 24, ex. 20 à 22 et feuille n° 25, ex. 2, 3, 6 et 7.\vspace{.4cm}\\
% 
% \noindent\textbf{\bf Vendredi 12 juin 2019} \\
% \bu\ Exercices : feuille n° 24, ex. 17 à 19.\vspace{.4cm}\\
% 
% \noindent\textbf{\bf Jeudi 11 juin 2019} \\
% \bu\ Exercices : feuille n° 24, ex. 12 à 16.\vspace{.4cm}\\
% 
% \noindent\textbf{\bf Mercredi 10 juin 2019} \\
% \bu\ Cours : 3 - Automorphismes orthogonaux (fin).\\
% \bu\ Exercices : feuille n° 24, ex. 6 à 11.\vspace{.4cm}\\
% 
% \noindent\textbf{\bf Lundi 08 juin 2019} \\
% \bu\ Interrogation n° 23.\\
% \bu\ Cours : 3 - Automorphismes orthogonaux (suite).\\
% \bu\ Exercices : feuille n° 23, ex. 14 et feuille n° 24, ex. 1 à 5.\vspace{.4cm}\\
% 
% \noindent\textbf{Vendredi 05 juin 2019}\\ 

% 
% \noindent\textbf{ Mercredi 29 avril 2019} \\
% \bu\ Cours : 4 - Séries absolument convergentes ; 5 - Représentation décimale des réels ; 6 - 
% Compléments.\\
% \bu\ Exercices : feuille n° 20, ex. 19, et feuille n° 21, ex. 1.\vspace{.4cm}\\
%

%\bu\ Exercices : feuille n° 20, ex. 11 et 13 à 18.\vspace{.4cm}\\

% \noindent\textbf{\bf Lundi 13 juin 2020}\\
%  ; 2 - Séries à termes réels positifs ; 3 - Comparaison série - intégrale.\\
% \bu\ Exercices : feuille n° 25, ex. 14 à 22.\vspace{.4cm}\\
% 
% \noindent\textbf{Vendredi 10 juin 2020}\\
% \bu\ Devoir surveillé n° 10.\\
% \bu\ Exercices : feuille n° 25, ex. 12 et 13.\vspace{.4cm}\\

% \noindent\textbf{Vendredi 09 juin 2020}\\
% \bu\ Cours : \bf Chapitre XXV \rm : Séries : 1 - Prolégomènes.\\
% \bu\ Exercices : feuille n° 25, ex. 20 et 16.\vspace{.4cm}\\
% 
% \noindent\textbf{\bf Jeudi 08 juin 2020}\\
% \bu\ Exercices : feuille n° 25, ex. 12, 14, 15, 17, 19 et 21.\vspace{.4cm}\\
% 
% \noindent\textbf{\bf Mardi 06 juin 2020} \\
% \bu\ Interrogation n° 19.\\
% \bu\ Cours : 3 - Automorphismes orthogonaux (fin).\\
% \bu\ Exercices : feuille n° 25, ex. 9 et 10.\\
% \bu\ À faire pour jeudi 08 juin : feuille n° 25, ex. 21 et 22.\vspace{.4cm}\\
% 
% \noindent\textbf{Vendredi 02 juin 2020}\\
% \bu\ Cours : 3 - Automorphismes orthogonaux (suite).\\
% \bu\ Exercices : feuille n° 25, ex. 11 et 13.\vspace{.4cm}\\
% 
% \noindent\textbf{\bf Jeudi 01 juin 2020}\\
% \bu\ Distribution : DM n° 20 (à rendre le 08 juin).\\
% \bu\ Cours : 3 - Automorphismes orthogonaux (début).\\
% \bu\ Exercices : feuille n° 25, ex. 7 et 8.\vspace{.4cm}\\
% 
% \noindent\textbf{Mardi 30 mai 2020}\\
% \bu\ Cours : 2 - Orthogonalité (fin).\\
% \bu\ Exercices : feuille n° 25, ex. 5 et 6.\vspace{.4cm}\\
% 
% \noindent\textbf{\bf Lundi 29 mai 2020} \\
% \bu\ Interrogation n° 18.\\
% \bu\ Cours :  2 - Orthogonalité (suite).\\
% \bu\ Exercices : feuille n° 24, ex. 14 et 17, et feuille n° 25, ex. 1 à 4.\vspace{.4cm}\\
% 
% \noindent\textbf{Mardi 23 mai 2020}\\
% \bu\ Cours : \bf Chapitre XXIV \rm : Espaces vectoriels euclidiens : 1 - Produits scalaires, normes et distances 
% (fin).\\
% .\vspace{.4cm}\\
% 
% \noindent\textbf{\bf Lundi 22 mai 2020} \\
% \bu\ Interrogation n° 17.\\
% \bf Chapitre XXIV \rm : Espaces vectoriels euclidiens : 1 - Produits scalaires, normes et distances 
% (suite).\\
% \bu\ Exercices : feuille n° 24, ex. 10, 12 et 13.\vspace{.4cm}\\
% 
% \noindent\textbf{Vendredi 19 mai 2020}\\
% (début).\\
% \bu\ Exercices : feuille n° 24, ex. 1 à 5.\vspace{.4cm}\\
% 
% \noindent\textbf{\bf Jeudi 18 mai 2020} \\
% \bu\ Distribution : DM n° 19 (à rendre le 01 juin).\\
% 
% 
% \noindent\textbf{Mercredi 12 juin 2020}\\
% \bu\ Cours :  2 - Orthogonalité (début).\\
% \bu\ Exercices : feuille n° 25, ex. 8 à 14.\vspace{.4cm}\\
% 
% \noindent\textbf{Vendredi 07 juin 2020}\\
% \bf Chapitre XXVI \rm : Espaces vectoriels euclidiens : 1 - Produits scalaires, normes et distances .\\
% \bu\ Exercices : feuille n° 25, ex. 14.1).\vspace{.4cm}\\
%  
% \noindent\textbf{Jeudi 06 juin 2020}\\
% \bu\ Cours : 5 - Déterminant d'une matrice carrée.
% \bu\ Exercices : feuille n° 25, ex. 6 et 7.\vspace{.4cm}\\
% 
% \noindent\textbf{Mercredi 05 juin 2020}\\
% \bu\ Distribution : DM n° 22 (à rendre le 13 juin).\\
% \bu\ Exercices : feuille n° 25, ex. 1 à 5.\vspace{.4cm}\\
%  
% \noindent\textbf{Lundi 3 juin 2020}\\
% \bu\ Interrogation n° 21.\\
% \bu\ Cours : 2. Applications multilinéaires - 3. Déterminant d’une famille de vecteurs.\\
% \bu\ Exercices : feuille n° 24, ex. 12 à 20.\vspace{.4cm}\\
% 
% \noindent\textbf{Mercredi 29 mai 2020}\\
% \bu\ Cours : \bf Chapitre XXV \rm : Déterminants : 1 - Groupe symétrique.\\
% \bu\ Exercices : feuille n° 24, ex. 2, 3 et 11.\vspace{.4cm}\\
% 
% \noindent\textbf{Lundi 27 mai 2020}\\
% \bu\ Cours : 7 - Matrices semblables et trace ; 8 - Matrices par blocs.\\
% \bu\ Exercices : feuille n° 24, ex. 7 à 10.\vspace{.4cm}\\
% 
% \noindent\textbf{Vendredi 24 mai 2020}\\
% \bu\ Devoir surveillé n° 09.\\
% \bu\ Cours : 5 - Rang ; 6 - Systèmes linéaires.\\
% \bu\ Distribution : DM n° 21 (à rendre le 6 juin).\\
% \bu\ Exercices : feuille n° 24, ex. 1 et 4 à 6.\vspace{.4cm}\\
%  
% \noindent\textbf{Jeudi 23 mai 2020}\\
% \bu\ Cours : 3 - Matrices remarquables ; 4 - Opérations élémentaires.\\
% \bu\ Exercices : feuille n° 23, ex. 14, 17 et 18.\vspace{.4cm}\\
% 
% \noindent\textbf{Mercredi 22 mai 2020}\\
% \bu\ Cours : 2 - Matrices, familles de vecteurs et applications linéaires (fin).\\
% \bu\ Exercices : feuille n° 23, ex. 11 à 13.\vspace{.4cm}\\
%  
% \noindent\textbf{Lundi 20 mai 2020}\\
% \bu\ Interrogation n° 20.\\
% \bu\ Cours : 2 - Matrices, familles de vecteurs et applications linéaires (suite).\\
% \bu\ Exercices : feuille n° 23, ex. 11 à 13.\\
% \bu\ À faire pour mercredi 22 : feuille n° 23, ex. 17.\\
% \bu\ À faire pour jeudi 23 : feuille n° 24, ex. 01 et 04.\\
% \bu\ À faire pour vendredi 24 : feuille n° 24, ex. 05 et 06.\vspace{.4cm}\\
%  
% \noindent\textbf{Vendredi 17 mai 2020}\\
% \bu\ Cours : 2 - Matrices, familles de vecteurs et applications linéaires (suite).\\
% \bu\ Exercices : feuille n° 23, ex. 08 et 16.\vspace{.4cm}\\
% 
% \noindent\textbf{Jeudi 16 mai 2020}\\
% \bu\ Cours : 2 - Matrices, familles de vecteurs et applications linéaires (suite).\\
% \bu\ Exercices : feuille n° 23, ex. 07, 10 et 15.\vspace{.4cm}\\
% 
% \noindent\textbf{Mercredi 15 mai 2020}\\
% \bu\ Distribution : DM n° 20 (à rendre le 23 mai).\\
% \bu\ Cours : \bf Chapitre XXIV \rm : Matrices ; 1 - Structure de $\mcal M_{n,p}(\K)$ ; 2 - Matrices, familles de 
% vecteurs et applications linéaires (début).\\
% \bu\ Exercices : feuille n° 23, ex. 06 et 09.\vspace{.4cm}\\
% 
% \noindent\textbf{Lundi 13 mai 2020}\\
% \bu\ Interrogation surprise n° 19.\\
% \bu\ Cours : 2 - Variables aléatoires (fin).\\
% \bu\ Exercices : feuille n° 23, ex. 01 à 05.\\
% \bu\ À faire pour mercredi 15 : feuille n° 23, ex. 06.\\
% \bu\ À faire pour jeudi 16 : feuille n° 23, ex. 07.\\
% \bu\ À faire pour vendredi 17 : feuille n° 23, ex. 08.\vspace{.4cm}\\
% 
% \noindent\textbf{Vendredi 10 mai 2020}\\
% \bu\ Cours : 2 - Variables aléatoires (suite).\\
% \bu\ Exercices : feuille n° 22, ex. 15 et 17 à 20.\vspace{.4cm}\\
% 
% \noindent\textbf{Jeudi 09 mai 2020}\\
% \bu\ Distribution : DM n° 20 (à rendre le 16 mai).\\
% \bu\ Cours : 2 - Variables aléatoires (suite).\\
% \bu\ Exercices : feuille n° 22, ex. 12 et 16.\vspace{.4cm}\\
% 
% \noindent\textbf{Lundi 06 mai 2020}\\
% \bu\ Cours : 2 - Variables aléatoires (suite).\\
% \bu\ Exercices : feuille n° 22, ex. 06 à 11, 13 et 14.\\
% \bu\ À faire pour jeudi 09 : feuille n° 22, ex. 12.\\
% \bu\ À faire pour vendredi 10 : feuille n° 22, ex. 15.\vspace{.4cm}\\
% 
% \noindent\textbf{Vendredi 03 mai 2020}\\
% \bu\ Devoir surveillé n° 08.\\
% \bu\ Cours : 2 - Variables aléatoires (début).\\
% \bu\ Exercices : feuille n° 22, ex. 02.\\
% \bu\ À faire pour lundi 06 : feuille n° 22, ex. 06 et 07.\vspace{.4cm}\\
% 
% \noindent\textbf{Jeudi 02 mai 2020}\\
% \bu\ Cours : 1 - Événements (fin).\\
% \bu\ Exercices : feuille n° 21, ex. 03 et feuille n° 22, ex. 04 et 05.\vspace{.4cm}\\
% 
% \noindent\textbf{Lundi 29 avril 2020}\\
% \bu\ Interrogation surprise n° 18.\\
% \bu\ Cours : Événements (suite).\\
% \bu\ Exercices : feuille n° 21, ex. 02, 04 et 06 à 08, et feuille n° 22, ex. 01.\\
% \bu\ À faire pour jeudi 02 : feuille n° 21, ex. 03.\\
% \bu\ À faire pour vendredi 03 : feuille n° 22, ex. 02 et 03.\vspace{.4cm}\\
% 
% \noindent\textbf{\bf Vacances de printemps}.\vspace{.4cm}\\
% 
% \noindent\textbf{Vendredi 11 avril 2020}\\
% \bu\ Cours : \bf Chapitre XXIII \rm : Probabilités : 1 - Événements (début).\\
% \bu\ Exercices : feuille n° 21, ex. 01 et 05.\vspace{.4cm}\\
% 
% \noindent\textbf{Jeudi 10 avril 2020}\\
% \bu\ Distribution : DM n° 18 (à rendre le 02 mai).\\
% \bu\ Cours : 3 - Applications linéaires en dimension finie ; 4 - Formes linéaires et hyperplans.\\
% \bu\ Exercices : feuille n° 20, ex. 20 et 27 à 28.\vspace{.4cm}\\
% 
% \noindent\textbf{Mercedi 09 avril 2020}\\
% \bu\ Cours : 2 - Sev en dimension finie (fin).\\
% \bu\ Exercices : feuille n° 20, ex. 19.\vspace{.4cm}\\
% 
% \noindent\textbf{Lundi 08 avril 2020}\\
% \bu\ Interrogations surprises n° 17 et 17bis.\\
% \bu\ Cours : 2 - Sev en dimension finie (début).\\
% \bu\ Exercices : feuille n° 20, ex. 21, 22, 25, 26 et 30.\\
% \bu\ À faire pour mercredi 10 : feuille n° 20, ex. 19.\\
% \bu\ À faire pour jeudi 11 : feuille n° 20, ex. 27 et 28.\\
% \bu\ À faire pour vendredi 12 : feuille n° 21, ex. 05.\vspace{.4cm}\\
% 
% \noindent\textbf{Jeudi 04 avril 2020}\\
% \bu\ Cours : 1 - Notion de dimension (fin).\\
% \bu\ Exercices : feuille n° 20, ex. 09, 12, 16 et 17.\vspace{.4cm}\\
% 
% \noindent\textbf{Mercedi 03 avril 2020}\\
% \bu\ Cours : 1 - Notion de dimension (début).\\
% \bu\ Exercices : feuille n° 20, ex. 08, 10 et 11.\vspace{.4cm}\\
% 
% \noindent\textbf{Lundi 01 avril 2020}\\
% \bu\ Cours : Dénombrement (fin).\\
% \bf Chapitre XXII \rm : Espaces vectoriels de dimension finie  : 1 - Notion de dimension (début).\\
% \bu\ Exercices : feuille n° 20, ex. 03 à 07.\\
% \bu\ À faire pour mercredi 03 : feuille n° 20, ex. 08.\\
% \bu\ À faire pour jeudi 04 : feuille n° 20, ex. 09.\vspace{.4cm}\\
% 
% \noindent\textbf{\bf Vendredi 29 mars 2020} \\
% \bu\ Devoir surveillé n° 07.\\
% \bu\ Cours : \bf Chapitre XXI \rm : Dénombrement (début).\\
% \bu\ À faire pour lundi 01 : feuille n° 19, ex. 12, et feuille n° 20, ex. 01 et 02.\vspace{.4cm}\\
% 
% \noindent\textbf{Jeudi 28 mars 2020}\\
% \bu\ Cours : Intégration (fin).\\
% \bu\ Exercices : feuille n° 19, ex. 11, 13 et 15.\vspace{.4cm}\\
% 
% \noindent\textbf{Mercredi 27 mars 2020} \\
% \bu\ Cours : Intégration (suite).\\
% \bu\ Exercices : feuille n° 19, ex. 06, 07 et 14.\vspace{.4cm}\\
% 
% \noindent\textbf{\bf Lundi 25 mars 2020} \\
% \bu\ Interrogation surprise n° 16.\\
% \bf Chapitre XX \rm : Intégration (début).\vspace{.4cm}\\
% \bu\ Exercices : feuille n° 19, ex. 04, 05 et 08 à 10.\\
% \bu\ À faire pour mercredi 27 : feuille n° 19, ex. 07.\\
% \bu\ À faire pour jeudi 28 : feuille n° 19, ex. 11.\\
% \bu\ À faire pour vendredi 29 : feuille n° 19, ex. 12.\vspace{.4cm}\\
% 
% \noindent\textbf{\bf Vendredi 22 mars 2020} \\
% \bu\ Cours : Applications linéaires et familles de vecteurs (fin).\\
% \bu\ Exercices : feuille n° 19, ex. 01 et 03.\vspace{.4cm}\\
% 
% \noindent\textbf{Jeudi 21 mars 2020}\\
% \bu\ Cours : Applications linéaires et familles de vecteurs (suite).\\
% \bu\ Exercices : feuille n° 18, ex. 32.\vspace{.4cm}\\
% 
% \noindent\textbf{Mercredi 20 mars 2020} \\
% \bu\ Cours : Applications linéaires et familles de vecteurs (suite).\\
% \bu\ Exercices : feuille n° 18, ex. 28, 29 et 31.\vspace{.4cm}\\
% 
% \noindent\textbf{\bf Lundi 18 mars 2020} \\
% \bu\ Interrogation surprise n° 15.\\
% \bu\ Distribution : DM n° 16 (à rendre le 28 mars).\\
% \bu\ Cours : Applications linéaires et familles de vecteurs (suite).\\
% \bu\ Exercices : feuille n° 18, ex. 20, 21, 25 à 27 et 30.\\
% \bu\ À faire pour mercredi 20 : feuille n° 18, ex. 29.\\
% \bu\ À faire pour jeudi 21 : feuille n° 18, ex. 32.\\
% \bu\ À faire pour vendredi 22 : feuille n° 19, ex. 01 et 03.\vspace{.4cm}\\
% 
% \noindent\textbf{\bf Vendredi 15 mars 2020} \\
% \bu\ Cours : Applications linéaires et familles de vecteurs (suite).\\
% \bu\ Exercices : feuille n° 18, ex. 13.\vspace{.4cm}\\
% 
% \noindent\textbf{Jeudi 14 mars 2020}\\
% \bu\ Cours : \bf Chapitre XIX \rm : Applications linéaires et familles de vecteurs (début).\\
% \bu\ Exercices : feuille n° 18, ex. 18.\vspace{.4cm}\\
% 
% \noindent\textbf{Mercredi 13 mars 2020} \\
% \bu\ Distribution : DM n° 15 (à rendre le 21 mars).\\
% \bu\ Cours : Analyse asymptotique (fin).\\
% \bf Chapitre XIX \rm : Familles de vecteurs et applications linéaires.\\ 
% (début).\\
% \bu\ Exercices : feuille n° 18, ex. 08, 09 et 16.\vspace{.4cm}\\
% 
% \noindent\textbf{\bf Lundi 11 mars 2020} \\
% \bu\ Interrogation surprise n° 14.\\
% \bu\ Cours : Analyse asymptotique (suite).\\
% \bu\ Exercices : feuille n° 17, ex. 07 et 08, et feuille n° 18, ex. 01 à 04, 06, 07, 10 et 12.\\
% \bu\ À faire pour mercredi 13 : feuille n° 18, ex. 08.\\
% \bu\ À faire pour jeudi 14 : feuille n° 18, ex. 11.\\
% \bu\ À faire pour vendredi 15 : feuille n° 18, ex. 13.\vspace{.4cm}\\
% 
% \noindent\textbf{\bf Vendredi 08 mars 2020} \\
% \bu\ Cours : Analyse asymptotique (suite).\\
% \bu\ Exercices : feuille n° 17, ex. 04 à 06.\\
% \bu\ À faire pour lundi 11 : feuille n° 17, ex. 07.\vspace{.4cm}\\
% 
% \noindent\textbf{Jeudi 07 mars 2020}\\
% \bu\ Distribution : DM n° 14 (à rendre le 14 mars).\\
% \bu\ Cours : Analyse asymptotique (suite).\\
% \bu\ Exercices : feuille n° 17, ex. 03.\vspace{.4cm}\\
% 
% \noindent\textbf{Mercredi 06 mars 2020} \\
% \bu\ Cours : Analyse asymptotique (suite).\\
% \bu\ Exercices : feuille n° 16, ex. 07 et feuille n° 17, ex. 01 et 02.\\
% \bu\ À faire pour jeudi 07 : feuille n° 17, ex. 03.\\
% \bu\ À faire pour vendredi 08 : feuille n° 17, ex. 05.\vspace{.4cm}\\
% 
% \noindent\textbf{\bf Lundi 04 mars 2020} \\
% \bu\ Interrogation surprise n° 13.\\
% \bu\ Cours : \bf Chapitre XVIII \rm : Analyse asymptotique 
% (début).\\
% \bu\ Exercices : feuille n° 16, ex. 03, 04 et 08.\vspace{.4cm}\\
% 
% \noindent\textbf{\bf Vacances d'hiver }\\
% 
% \noindent\textbf{Vendredi 15 février 2020}\\
% \bu\ Devoir surveillé n° 06.\\
% \bu\ Cours : 2 - Sev (fin) ; 3 - Espaces affines.\vspace{.4cm}\\
%  
% \noindent\textbf{Jeudi 14 février 2020}\\
% \bu\ Cours : 2 - Sev (suite).\\
% \bu\ Exercices : feuille n° 15, ex. 18 et feuille n° 16, ex. 5 et 6.\vspace{.4cm}\\
%  
% \noindent\textbf{Mercredi 13 février 2020} \\
% \bu\ Cours : 2 - Sev (suite).\\
% \bu\ Exercices : feuille n° 15, ex. 14, feuille n° 16, ex. 1 et 8 (début).\vspace{.4cm}\\
% 
\noindent\textbf{\bf Lundi 10 février 2020} \\
\bu\ Interrogation surprise n° 13.\\
\bu\ Cours : 2 - Sev (début).\\
\bu\ Exercices : feuille n° 16, ex. 1, 2, 3, 4, 7 et 8.\\
\bu\ À faire pour mercredi 12 : feuille n° 17, ex. 1.\vspace{.4cm}\\

\noindent\textbf{Vendredi 7 février 2020}\\
\bu\ Cours : \bf Chapitre XVII \rm : Espaces vectoriels : 1 - Ev.\\
\bu\ Exercices : feuille n° 15, ex. 12 et 15 à 17.\\
\bu\ À faire pour lundi 10 : feuille n° 15, ex. 18.\vspace{.4cm}\\
 
\noindent\textbf{Jeudi 6 février 2020}\\
\bu\ Distribution : DM n° 13 (à rendre le 13 février).\\
\bu\ Cours : Fractions rationnelles (fin).\\
\bu\ Exercices : feuille n° 15, ex. 1 et 13.\vspace{.4cm}\\
 
\noindent\textbf{Mercredi 5 février 2020} \\
\bu\ Cours : Fractions rationnelles (suite).\\
\bu\ Exercices : feuille n° 15, ex. 4 et 6.\vspace{.4cm}\\

\noindent\textbf{\bf Lundi 3 février 2020} \\
\bu\ Interrogation surprise n° 12.\\
\bu\ Cours :  Dérivabilité (fin) ;\\
\bf Chapitre XVI \rm : Fractions rationnelles (début).\\
\bu\ Exercices : feuille n° 15, ex. 2, 3, 5, 8, 9, 10 et 11.\\
\bu\ À faire pour mercredi 5 : feuille n° 15, ex. 4.\\
\bu\ À faire pour jeudi 6 : feuille n° 15, ex. 7.\\
\bu\ À faire pour vendredi 7 : feuille n° 15, ex. 12.\vspace{.4cm}\\
 
\noindent\textbf{Vendredi 31 janvier 2020}\\
\bu\ Cours : Dérivabilité (suite).\\
\bu\ Exercices : feuille n° 14, ex. 17 à 19.\\
\bu\ À faire pour lundi 3 : feuille n° 15, ex. 2.\vspace{.4cm}\\
 
\noindent\textbf{Jeudi 30 janvier 2020}\\
\bu\ Distribution : DM n° 12 (à rendre le 6 février).\\
\bu\ Cours : Dérivabilité (suite).\\
\bu\ Exercices : feuille n° 14, ex. 11, 12, 15 et 16.\vspace{.4cm}\\
  
\noindent\textbf{Mercredi 29 janvier 2020} \\
\bu\ Cours : \bf Chapitre XV \rm : Dérivabilité (début).\\
\bu\ Exercices : feuille n° 14, ex. 1, 2, 4, 13, 14, 8 et 9 .\vspace{.4cm}\\
  
\noindent\textbf{Lundi 27 janvier 2020} \\
\bu\ Interrogation surprise n° 11.\\
\bu\ Cours : Polynômes (fin).\\
\bu\ Exercices : feuille n° 14, ex. 3, 5, 7 et 10.\\
\bu\ À faire pour mercredi 29 : feuille n° 14, ex. 1, 2, 4, 13 et 14.\\
\bu\ À faire pour jeudi 30 : feuille n° 14, ex. 11 et 12.\\
\bu\ À faire pour vendredi 31 : feuille n° 14, ex. 17.\vspace{.4cm}\\

\noindent\textbf{Vendredi 24 janvier 2020}\\
\bu\ Cours : Polynômes (suite).\\
\bu\ Exercices : feuille n° 13, ex. 11, et feuille n° 14, ex. 6.\\
\bu\ À faire pour lundi 27 : feuille n° 14, ex. 1 et 2.\vspace{.4cm}\\

\vspace{.4cm}\\

\noindent\textbf{Jeudi 23 janvier 2020}\\
\bu\ Cours : Polynômes (suite).\\
\bu\ Exercices : feuille n° 13, ex. 6, 7, 10, 12 et 14.\vspace{.4cm}\\

\noindent\textbf{Mercredi 22 janvier 2020} \\
\bu\ Distribution : DM n° 11 (à rendre le 30 janvier).\\
\bu\ Cours : Polynômes (suite).\vspace{.4cm}\\
  
\noindent\textbf{Lundi 20 janvier 2020} \\
\bu\ Cours : Polynômes (suite).\\
\bu\ Exercices : feuille n° 13, ex. 3, 4 et 9.\\
\bu\ À faire pour mercredi 22 : feuille n° 12, ex. 6.\\
\bu\ À faire pour jeudi 23 : feuille n° 12, ex. 10.\\
\bu\ À faire pour vendredi 24 : feuille n° 12, ex. 11.\vspace{.4cm}\\ 
 
\noindent\textbf{Vendredi 17 janvier 2020}\\
\bu\ Devoir surveillé n° 5.\\
\bu\ \bf Chapitre XIV \rm : Polynômes (début).\\
\bu\ Exercices : feuille n° 12, ex. 1 et 5.\vspace{.4cm}\\

\noindent\textbf{Mercredi 15 janvier 2020}\\
\bu\ Cours : Continuité (fin).\\
\bu\ Exercices : feuille n° 12, ex. 1.\vspace{.4cm}\\

\textbf{Lundi 13 janvier 2020} \\
\bu\ Interrogation surprise n° 10.\\
\noindent\bu\ Cours : \bf Chapitre XIII \rm : Continuité (début).\\
\bu\ Exercices : feuille n° 12, ex. 2 à 7.\vspace{.4cm}\\

\noindent\textbf{Vendredi 10 janvier 2019}\\
\bu\ Cours : Limites d'une fonction (fin).\\
\bu\ Exercices : feuille n° 11, ex. 4, 6, 8 et 12.\vspace{.4cm}\\

\noindent\textbf{Mercredi 8 janvier 2020} \\
\bu\ Distribution : DM n° 10 (à rendre le 16 janvier).\\
\bu\ Cours : Limites d'une fonction (suite).\\
\bu\ Exercices : feuille n° 11, ex. 2, 10 et 11.\vspace{.4cm}\\

\noindent\textbf{Lundi 6 janvier 2020} \\
\bu\ Interrogation surprise n° 9.\\
\bu\ Cours : \bf Chapitre XII \rm : Limites d'une fonction (début).\\ 
\bu\ Exercices : feuille n° 10, ex. 8, et feuille n° 11, ex. 1, 3, 5 et 9.\\
\bu\ À faire pour mercredi 8 : feuille n° 11, ex. 10 et 11.\\
\bu\ À faire pour jeudi 9 : feuille n° 11, ex. 8.\\
\bu\ À faire pour vendredi 10 : feuille n° 11, ex. 12.\vspace{.4cm}\\ 

\noindent\textbf{\bf Vacances de Noël}.\vspace{.4cm}\\

\noindent\textbf{Vendredi 20 décembre 2019}\\
\bu\ Exercices : feuille n° 10, ex. 14, 16 et 17.\vspace{.4cm}\\

 \noindent\textbf{Jeudi 19 décembre 2019}\\
\bu\ Cours : Anneaux et corps.\\ 
\bu\ Exercices : feuille n° 10, ex. 10, 12 et 13.\vspace{.4cm}\\
  
\noindent\textbf{\bf Mercredi 18 décembre 2019}\\
\bu\ Cours : 2 - Groupes (fin).\\ 
\bu\ Exercices : feuille n° 10, ex. 1 (fin), 3 et 4.\vspace{.4cm}\\
 
\noindent\textbf{Lundi 16 décembre 2019}\\
\bu\ Interrogation surprise n° 8.\\
\bu\ Cours : 2 - Groupes (début).\\ 
\bu\ Exercices : feuille n° 10, ex. 1 (début), 2 et 5 à 7.\\
\bu\ À faire pour mercredi 18 : feuille n° 10, ex. 10.\\
\bu\ À faire pour jeudi 19 : feuille n° 10, ex. 12.\\
\bu\ À faire pour vendredi 20 : feuille n° 10, ex. 16.\vspace{.4cm}\\ 

\noindent\textbf{Vendredi 13 décembre 2019}\\
\bu\ Cours : 6 - Suites récurrentes (fin) ; 7 - Suites complexes ; 8 - Premières séries numériques.\\
\bu\ Cours : \bf Chapitre XI \rm : Groupes, anneaux, corps  : 1 - Lci .\\ 
\bu\ Exercices : feuille n° 9, ex. 18.\vspace{.4cm}\\

\noindent\textbf{Jeudi 12 décembre 2019}\\
\bu\ Cours : 6 - Suites récurrentes (début).\\
\bu\ Exercices : feuille n° 9, ex. 15 à 17.\vspace{.4cm}\\ 
 
\noindent\textbf{\bf Mercredi 11 décembre 2019}\\
\bu\ Distribution : DM n° 9 (à rendre le 19 décembre).\\
\bu\ Cours : 4 - Traduction séquentielle de certaines propriétés ; 5 - Suites particulières.\\
\bu\ Exercices : feuille n° 9, ex. 6, 13, 14 et 19.\vspace{.4cm}\\ 

\noindent\textbf{Lundi 9 décembre 2019}\\
\bu\ Cours : 3 - Résultats de convergence (fin).\\
\bu\ Exercices : feuille n° 8, ex. 11 et 12, et feuille n° 9, ex. 1, 4, 8 et 11.\\ 
\bu\ À faire pour mercredi 11 : feuille n° 9, ex. 13 et 14.\\
\bu\ À faire pour jeudi 12 : feuille n° 9, ex. 15.\\
\bu\ À faire pour vendredi 13 : feuille n° 9, ex. 18.\vspace{.4cm}\\

\noindent\textbf{Vendredi 06 décembre 2019}\\
\bu\ Devoir surveillé n° 04.\\
\bu\ Cours : 2 - Limite d'une suite réelle (fin) ; 3 - Résultats de convergence (début).\\
\bu\ Exercices : feuille n° 08, ex. 7 et 9.\\
\bu\ À faire pour lundi 9 : feuille n° 8, ex. 11 et 12.\vspace{.4cm}\\
  
\noindent\textbf{Mercredi 04 décembre 2019}\\
\bu\ Cours : 2 - Limite d'une suite réelle (début).\\
\bu\ Exercices : feuille n° 08, ex. 5 et 8.\\
\bu\ À faire pour vendredi 06 : feuille n° 8, ex. 7, 9 et 11, et feuille n° 9, ex. 1, 3 et 7.\vspace{.4cm}\\

\noindent\textbf{Lundi 02 décembre 2019}\\
\bu\ Interrogation surprise n° 07.\\
\bu\ Cours : \bf Chapitre X :\rm Suites réelles et complexes : 1 - Vocabulaire.
\\
\bu\ Exercices : feuille n° 08, ex. 1, 3, 4, 6 et 10.\vspace{.4cm}\\

\noindent\textbf{Vendredi 29 novembre 2019}\\
\bu\ Cours : 3 - Nombres premiers (fin).\vspace{.4cm}\\


\noindent\textbf{Jeudi 28 novembre 2019}\\
\bu\ Cours : 2 - PPCM ; 3 - Nombres premiers (début).\\
\bu\ Exercices : feuille n° 07, ex. 05, 08, 10 et 11.\vspace{.4cm}\\
  
\noindent\textbf{\bf Mercredi 27 novembre 2019}\\
\bu\ Distribution : DM n° 08 (à rendre le 05 décembre).\\
\bu\ Cours : 2 - PGCD (fin).\\
\bu\ Exercices : feuille n° 07, ex. 06 et 09.\vspace{.4cm}\\

\noindent\textbf{Lundi 25 novembre 2019}\\
\bu\ Interrogation surprise n° 06.\\
\bu\ Cours : 2 - PGCD (début).\\
\bu\ Exercices : feuille n° 07, ex. 03.\vspace{.4cm}\\ 

\noindent\textbf{Vendredi 22 novembre 2019}\\
\bu\ Cours : \bf Chapitre IX \rm : Arithmétique : 1 - Divisibilité.\\
\bu\ Exercices : feuille n° 07, ex. 01 et 02.\vspace{.4cm}\\ 

\noindent\textbf{Jeudi 21 novembre 2019}\\
\bu\ Distribution : DM n° 07 (à rendre le 28 novembre).\\
\bu\ Cours : 6 - La relation d'ordre naturelle sur \R.\\
\bu\ Exercices : feuille n° 07, ex. 04.\vspace{.4cm}\\ 

\noindent\textbf{\bf Mercredi 20 novembre 2019}\\
\bu\ Cours : 5 - La relation d'ordre naturelle sur \N.\\
\bu\ Exercices : feuille n° 06, ex. 06 (fin).\vspace{.4cm}\\ 

\noindent\textbf{Lundi 18 novembre 2019}\\
\bu\ Cours : 3 - Relations d'ordre ;  4 - Majorants, minorants et compagnie.\\
\bu\ Exercices : feuille n° 06, ex. 04, 05, 07, 08 et 06 (début).\\ 
\bu\ À faire pour mercredi 20 : ex. 06 à finir.\\
\bu\ À faire pour jeudi 21 : feuille n° 07, ex. 01 et 02.\\
\bu\ À faire pour vendredi 22 : feuille n° 07, ex. 03.\vspace{.4cm}\\

\noindent\textbf{Vendredi 15 novembre 2019}\\
\bu\ Devoir surveillé n° 03.\\
\bu\ Cours : \bf Chapitre VIII \rm : Relations d'ordre : 1 - Relations binaires ; 2 - Relations d'équivalence.\\
\bu\ Exercices : feuille n° 06, ex. 01 à 03.\vspace{.4cm}\\

\noindent\textbf{Jeudi 14 novembre 2019}\\
\bu\ Cours : 5 - Images directe et réciproque.\\
\bu\ Exercices : feuille n° 05, ex. 05, 08 et 09.\vspace{.4cm}\\

\noindent\textbf{Mercredi 13 novembre 2019}\\
\bu\ Cours : 4 - Injectivité, surjectivité, bijectivité.\\
\bu\ Exercices : feuille n° 05, ex. 10.\vspace{.4cm}\\

\noindent\textbf{Vendredi 08 novembre 2019}\\
\bu\ Cours : \bf Chapitre VII \rm : Notion d'application. 1 - Vocabulaire ; 2 - Restriction et prolongement ; 3 - Composition.\\
\bu\ Exercices : feuille n° 05, ex. 06.\\
\bu\ À faire pour mercredi 13 : feuille n° 05, ex. 10.\\
\bu\ À faire pour jeudi 14 : feuille n° 06, ex. 01.\\
\bu\ À faire pour vendredi 15 : feuille n° 06, ex. 02 et 03.\vspace{.4cm}\\

\noindent\textbf{Jeudi 07 novembre 2019}\\
\bu\ Distribution du DM n° 06 (à rendre le 14 novembre).\\
\bu\ Cours : Théorie des ensembles (fin).\\
\bu\ Exercices : feuille n° 04, ex. 27 et feuille n° 05, ex. 02.\vspace{.4cm}\\

\noindent\textbf{\bf Mercredi 06 novembre 2019}\\
\bu\ Cours : Théorie des ensembles (suite).\\
\bu\ Exercices : feuille n° 04, ex. 23 et 26.\\
\bu\ À faire pour jeudi 07 : feuille n° 04, ex. 27.\vspace{.4cm}\\
 
\noindent\textbf{Lundi 04 novembre 2019}\\
\bu\ Interrogation surprise n° 05.\\
\bu\ Cours : Théorie des ensembles (suite).\\
\bu\ Exercices : feuille n° 05, ex. 03.\vspace{.4cm}\\
 
\noindent\textbf{ Vacances de novembre }\vspace{.4cm}\\
 
\noindent\textbf{Vendredi 18 octobre 2019}\\
\bu\ Cours : 5 - Circuits RL et RLC.\\
\bu\ Cours : \bf Chapitre VI \rm : Théorie des ensembles (début).\\
\bu\ Exercices : feuille n° 04, ex. 21 et 24.\\
\bu\ À faire pour lundi 04 : feuille n° 04, ex. 23.\vspace{.4cm}\\

\noindent\textbf{Jeudi 17 octobre 2019}\\
\bu\ Distribution du DM n° 05 (à rendre le 07 novembre).\\
\bu\ Cours : 4 - Équations différentielles du second ordre.\\
\bu\ Exercices : feuille n° 04, ex. 10, 11, 13 et 22.\vspace{.4cm}\\

\noindent\textbf{\bf Mercredi 16 octobre 2019}\\
\bu\ Cours : 3.2 - Équations différentielles du premier ordre avec second membre.\\
\bu\ Exercices : feuille n° 04, ex. 20.\vspace{.4cm}\\
  
\noindent\textbf{Lundi 14 octobre 2019}\\
\bu\ Cours : 1 - Résultats d'analyse relatifs aux fonctions à valeurs complexes d'une variable réelle, et 
intégration (fin) ; 2 - Généralités ; 3.1 - Équations différentielles homogène du premier ordre.\\
\bu\ Exercices : feuille n° 04, ex. 12 et 14 à 17 (début).\\
\bu\ À faire pour mercredi 16 : feuille n° 04, ex. 20.\\
\bu\ À faire pour jeudi 17 : feuille n° 04, ex. 22.\\
\bu\ À faire pour vendredi 18 : feuille n° 04, ex. 26, 18 et 19.\vspace{.4cm}\\
  
\noindent\textbf{Vendredi 11 octobre 2019}\\
\bu\ Devoir surveillé n° 02.\\
 \bu\ Cours : \bf Chapitre V \rm : Équations différentielles : 1 - Résultats d'analyse relatifs aux fonctions à 
valeurs complexes d'une variable réelle, et intégration (début).\\
 \bu\ Exercices : feuille n° 04, ex. ex. 07 et 09, et début du 08.\\
 \bu\ À faire pour lundi 14 : feuille n° 04, ex. 12.\vspace{.4cm}\\
  
 \noindent\textbf{Jeudi 10 octobre 2019}\\
 \bu\ Cours : 5 - Nombres complexes et géométrie plane.\\
 \bu\ Exercices : feuille n° 04, ex. 02, 03, 05 et 06.\vspace{.4cm}\\
  
\noindent\textbf{\bf Mercredi 09 octobre 2019}\\
\bu\ Cours : 3 - Équations du second degré (fin) ; 4 - Techniques de calcul.\\
 \bu\ Exercices : feuille n° 04, ex. 01 et 04.\vspace{.4cm}\\

\noindent\textbf{\bf Lundi 07 octobre 2019}\\
\bu\ Interrogation surprise n° 04.\\ 
\bu\ Cours : 2 - Le groupe \U\ des nombres complexes de module 1 (fin) ; 3 - Équations du second 
degré (début).\\
\bu\ Exercices : feuille n° 03, ex. 02 et 05 à 12.\\
\bu\ À faire pour mercredi 09 : feuille n° 04, ex. 01 et 04.\\
\bu\ À faire pour jeudi 10 : feuille n° 04, ex. 03 et 06.\\
\bu\ À faire pour vendredi 11 : feuille n° 04, ex. 07 et 08.\vspace{.4cm}\\

\noindent\textbf{Vendredi 04 octobre 2019}\\
\bu\ Cours : 2 - Le groupe \U\ des nombres complexes de module 1 (suite).\\
\bu\ Exercices : feuille n° 02, ex. 20.\\
\bu\ À faire pour lundi 07 : feuille n° 03, ex. 01, 03 et 04.\vspace{.4cm}\\ 

\noindent\textbf{Jeudi 03 octobre 2019}\\
\bu\ Cours : 2 - Le groupe \textbf{U} des nombres complexes de module 1 (début).\\
 \bu\ Exercices : feuille n° 02, ex. 9 et 19.\vspace{.4cm}\\

\noindent\textbf{\bf Mercredi 02 octobre 2019}\\
\bu\ Distribution du DM n° 04 (à rendre le 10 octobre).\\
\bu\ Cours : \bf Chapitre IV \rm : Le corps des complexes : 1 - Construction de \C\ (début).\\
 \bu\ Exercices : feuille n° 02, ex. 14.\vspace{.4cm}\\

\noindent\textbf{\bf Lundi 30 septembre 2019}\\
 \bu\ Interrogation surprise n° 03.\\
 \bu\ Cours : \bf Chapitre III \rm : Quelques fondamentaux (fin).\\
 \bu\ Exercices : feuille n° 02, ex. 13, 15 et 16.\\
\bu\ À faire pour mercredi 02 : étudier les récurrences fausses de la fin du cours.\vspace{.4cm}\\

 \noindent\textbf{Vendredi 27 septembre 2019}\\
 \bu\ Cours : \bf Chapitre III \rm : Quelques fondamentaux (début).\\
 \bu\ Exercices : feuille n° 02, ex. 05 et 08, 10 et 11.\\
 \bu\ À faire pour lundi 30 : feuille n° 02, ex. 12.\vspace{.4cm}\\
 
 \noindent\textbf{Jeudi 26 septembre 2019}\\
 \bu\ Cours : 5 - Systèmes linéaires (début).\\
 \bu\ Exercices : feuille n° 02, ex. 02, 03 et 07.\vspace{.4cm}\\
 
 \noindent\textbf{\bf Mercredi 25 septembre 2019}\\
 \bu\ Cours : 5 - Systèmes linéaires (début).\\
 \bu\ Distribution du DM n° 03 (à rendre le 03 octobre).\\
 
 \noindent\textbf{\bf Lundi 23 septembre 2019}\\
 \bu\ Cours : 4 - Matrices.\\
 \bu\ Exercices : feuille n° 02, ex. n° 01 et 04 à 06.\\
 \bu\ À faire pour mercredi 25 : feuille n° 02, ex. 03 et 07.\\
 \bu\ À faire pour jeudi 26 : feuille n° 02, ex. 02.\\
 \bu\ À faire pour vendredi 27 : feuille n° 02, ex. 10 et 11.\vspace{.4cm}\\ 
 
 \noindent\textbf{Vendredi 21 septembre 2019}\\
 \bu\ Devoir surveillé n° 01.\\
 \bu\ Exercices : feuille n° 01, ex. 15 à 17.\vspace{.4cm}\\
  
 \noindent\textbf{Jeudi 20 septembre 2019}\\
 \bu\ Cours : 3 - Quelques formules.\\
 \bu\ Exercices : feuille n° 01, ex. 09, 12, 18 et 19.\vspace{.4cm}\\
 
 \noindent\textbf{\bf Mercredi 19 septembre 2019}\\
 \bu\ Cours : 1 - Sommes (fin) ; 2 - Produits.\vspace{.4cm}\\
 
 \noindent\textbf{\bf Lundi 16 septembre 2019}\\ 
 \bu\ Interrogation surprise n° 02.\\
 \bu\ 1 - Sommes (début).\\
 \bu\ Exercices : feuille n° 01, ex. 10, 11, 13 et 14.\\
 \bu\ À faire pour mercredi 19 : feuille n° 01, ex. 18.\\
 \bu\ À faire pour jeudi 20 : feuille n° 01, ex. 19.\vspace{.4cm}\\
 
 \noindent\textbf{Vendredi 14 septembre 2019}\\
 \bu\ Distribution du DM n° 02 (à rendre le 19 septembre).\\
 \bu\ Cours : 9 - Fonctions hyperboliques.\\
 \bu\ Exercices : feuille n° 01, ex. 7.\\
 \bu\ À faire pour lundi 10 : feuille n° 01, ex. 9.\vspace{.4cm}\\
 
 \noindent\textbf{\bf Jeudi 13 septembre 2019}\\
 \bu\ Cours : 7 - Fonctions circulaires ; 8 - Fonctions circulaires inverses.\\
 \bu\ Exercices : feuille n° 01, ex. 4, 5 et 6.\vspace{.4cm}\\
     
 \noindent\textbf{\bf Mercredi 12 septembre 2019}\\
 \bu\ Cours : 6 - Exponentielle et logarithme.\vspace{.4cm}\\
 
 \noindent\textbf{\bf Lundi 10 septembre 2019}\\
 \bu\ Cours : 5 - Puissances.\\
 \bu\ Interrogation surprise n° 01.\\
 \bu\ Exercices : feuille n° 01, ex. 8, et quelques exercices supplémentaires.\vspace{.4cm}\\
 
\noindent\textbf{Vendredi 06 septembre 2019}\\
\bu\ Cours : 3 - Réciproques ; 4 - Fonction valeur absolue.\\
\bu\ Exercices : feuille n° 01, ex. 3.\\
\bu\ À faire pour lundi 09 : feuille n° 01, ex. 4.\\
\bu\ À faire pour mercredi 11 : feuille n° 01, ex. 5.\\
\bu\ À faire pour jeudi 12 : feuille n° 01, ex. 6.\\
\bu\ À faire pour vendredi 13 : feuille n° 01, ex. 7.\vspace{.4cm}\\
 
\noindent\textbf{\bf Jeudi 05 septembre 2019}\\
\bu\ Cours : 1 - Vocabulaire (fin) ; 2 - Effet d'une transformation sur le graphe ; 3 - Composée de fonctions 
(début).\\
\bu\ Exercices : feuille n° 01, ex. 1 et 2.\vspace{.4cm}\\
     
\noindent\textbf{\bf Mercredi 04 septembre 2019}\\
\bu\ Cours : \bf Chapitre I \rm : Fonctions usuelles : 1 - Vocabulaire (début).\\
\bu\ À faire pour jeudi 05 : feuille n° 01, ex. 3.\\
\bu\ À faire pour vendredi 06 : feuille n° 01, ex. 4.\vspace{.4cm}\\
 

\noindent\textbf{\bf Mardi 03 septembre 2019}\\
Journée de rentrée ; distribution des feuilles de TD, des formulaires, des
chapitres I et II, et du DM n° 01 (à rendre le 12 septembre).\vspace{.4cm}\\


\label{end}
\end{document}


