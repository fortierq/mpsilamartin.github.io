\documentclass[12pt,a4paper]{article}

\textheight=25cm
\topmargin=-50pt
\input{/home/skanderk/.latex/intro2.sty}
\usepackage{hyperref}

\begin{document}

\begin{center}
\Large\bf PROGRAMME DE TRAVAIL À DISTANCE DE MATHÉMATIQUES\\
MPSI 2 La Martinière Monplaisir\\ Printemps 2020
\end{center}
\vspace{1cm}
\vspace{.4cm}


\noindent\textbf{Vendredi 11 juin 2020}
\begin{itemize}
\item Cours : section 4 (vidéo \href{https://youtu.be/bxy6vg6_DS0}{ ICI}).
\item Exercices : feuille n° 25, ex. 5 pour ceux qui ont du temps.\vspace{.4cm}
\end{itemize}

\noindent\textbf{Jeudi 10 juin 2020}
\begin{itemize}
\item Cours : fin de la section 3.1 et section 3.2 (vidéo \href{https://youtu.be/LjDFfUYnHZk}{ ICI} et \href{https://youtu.be/-pnLRj_3loA}{ LÀ}). À regarder très rapidement, pas fondamental du tout.
\item Exercices : feuille n° 25, ex. 6.\vspace{.4cm}
\end{itemize}
 
\noindent\textbf{Mercredi 9 juin 2020}
\begin{itemize}
\item Cours : section 3.1, 1ère et 2ème parties (vidéo \href{https://youtu.be/pv299BRWnXQ}{ ICI} et \href{https://youtu.be/LRI5gsSShV4}{ LÀ}). À regarder rapidement, pas fondamental.
\item Exercices : feuille n° 25, ex. 1 à 4.\vspace{.4cm}
\end{itemize}

\noindent\textbf{Lundi 7 juin 2020}
\begin{itemize}
\item Cours : section 1.4 (vidéo \href{https://youtu.be/iMoYlN1wqKA}{ ICI}) et section 2 (vidéo \href{https://youtu.be/22HK-I6wIm8}{ LÀ}). Dans la section 1.4, vous pouvez zapper les démonstrations. Mais vous devez connaître l'énoncé du théorème 1.4.5 et savoir calculer la signature d'une permutation. Dans la section 2 encore, les points techniques peuvent être zappés. Contentez-vous de comprendre les définitions et les énoncés des résultats.
\item Exercices : feuille n° 25, ex. 1 à 3, sans le calcul de la signature à la fin.\vspace{.4cm}
\end{itemize}

\noindent\textbf{Week-end du 5 au 6 juin 2020}
\begin{itemize}
\item Cours : revoir le cours de la semaine.
\item Exercices : feuille n° 24, ex. 20.\vspace{.4cm}
\end{itemize}

\noindent\textbf{Vendredi 4 juin 2020}
\begin{itemize}
\item Cours : \textbf{Chapitre XXV} - Déterminant. Sections 1.1 et 1.2 (vidéo \href{https://youtu.be/s-WUHnSWgLs}{ ICI}) et 1.3 (vidéo \href{https://youtu.be/UtaFPRkdxU8}{ LÀ}). Les sections 1.1 et 1.2 sont assez courtes et sans difficulté particulière, mais sont importantes pour la suite. Dans la section 7.3, il y a des démonstrations très techniques : il n'y a aucune nécessité à les connaître, elles sont hors-programme. Les seuls résultats importants sont les théorèmes 1.3.4 et 1.3.6, mais là aussi, vous pouvez sauter les démonstrations si vous voulez. Mais vous devez savoir décomposer une permutation en produits de transpositions ou en produits de cycles de supports disjoints. Le principe est simple. Regardez la seconde vidéo pour voir de quoi ça parle, et ne vous attardez que sur les exemples. Nous ferons un topo en cours pour récapituler tout ça.
\item Exercices : feuille n° 24, ex. 19.\vspace{.4cm}
\end{itemize}

\noindent\textbf{Jeudi 3 juin 2020}
\begin{itemize}
\item 
\item Cours : Section 7.2 (vidéos \href{https://youtu.be/uxTWn1yr0dE}{ ICI} et \href{https://youtu.be/AtZB0jCKsfM}{ LÀ}). Cette partie est importante et à bien travailler.
\item Exercices : feuille n° 24, ex. 17.\vspace{.4cm}
\end{itemize}
 
\noindent\textbf{Mercredi 2 juin 2020}
\begin{itemize}
\item Cours : Sections 6 (vidéo \href{https://youtu.be/eLJ-axFV7ag}{ ICI}) et 7.1 (vidéo \href{https://youtu.be/0K9zS6irk18}{ LÀ}). La section 6 ne contient rien de difficile et est à parcourir rapidement, le seul résultat important est celui du théorème 6.2.2. La section 7.1 est plus compliquée, et n'est pas proritaire si vous manquez de temps.
\item Exercices : feuille n° 24, ex. 15 et 16.\vspace{.4cm}
\end{itemize}

\noindent\textbf{Week-end du 30 mai au 1er juin 2020}
\begin{itemize}
\item Cours : revoir le cours de la semaine.
\item Exercices : se mettre au point sur les calculs de rangs, noyaux, images (il y a plein de calculs dans la feuille n° 24, commencez à piocher dans les ex. 12 à 14).\vspace{.4cm}
\end{itemize}

\noindent\textbf{\bf Remarques sur la semaine du 25 au 29 mai 2020}
\begin{itemize}
\item Le programme de colle imaginaire est celui-ci : tout depuis le début de l'année (c'est ce que je ferais s'il y avait vraiment colle).
\item Tout ce qui concerne le chapitre XXIV se trouve ICI.\vspace{.4cm}
\end{itemize}

\noindent\textbf{Vendredi 29 mai 2020}
\begin{itemize}
\item Devoir surveillé n° 9.
\item Cours : sections 5.3 et 5.4 .
\item Exercices : feuille n° 24, ex. 11.\vspace{.4cm}
\end{itemize}

\noindent\textbf{Jeudi 28 mai 2020}
\begin{itemize}
\item 
\item Cours : fin de la section 5.1, et section 5.2.
\item Exercices : feuille n° 24, ex. 10.\vspace{.4cm}
\end{itemize}
 
\noindent\textbf{Mercredi 27 mai 2020}
\begin{itemize}
\item Cours : partie 4 et partie 5, section 5.1, de 5.1.1 à 5.1.7 inclus.
\item Exercices : feuille n° 24, ex. 8.\vspace{.4cm}
\end{itemize}
 
\noindent\textbf{Lundi 25 mai 2020}
\begin{itemize}
\item Rendre le DM n° 20.
\item Interrogation surprise n° 21.
\item Cours : partie 3.
\item Exercices : feuille n° 24, ex. 7.\vspace{.4cm}
\end{itemize}

\noindent\textbf{Week-end du 21 au 24 mai 2020}
\begin{itemize}
\item Cours : revoir le cours de la semaine.
\item Faire la fiche d'entraînement n° 6 : .
\item Exercices : feuille n° 24, ex. 5.\vspace{.4cm}
\end{itemize}

\noindent\textbf{\bf Remarques sur la semaine du 18 au 20 mai 2020}
\begin{itemize}
\item Le programme de colle imaginaire est celui-ci : tout le chapitre XXIII.
\item Feuille de TD n° 24 : les indications et corrections des exercices se trouvent ICI.
\item Des remarques et corrections d'exercices sur le cours du chapitre XXIV se trouvent ICI.\vspace{.4cm}
\end{itemize}
 
\noindent\textbf{Mercredi 20 mai 2020}
\begin{itemize}
\item Cours : section 2.4.
\item Exercices : feuille n° 24, ex. 3.\vspace{.4cm}
\end{itemize}
 
\noindent\textbf{Lundi 18 mai 2020}
\begin{itemize}
\item Interrogation surprise n° 20.
\item Cours : section 2.4.
\item Exercices : feuille n° 24, ex. 1, 2 et 4.\vspace{.4cm}
\end{itemize}

\noindent\textbf{Week-end du 16 au 17 mai 2020}
\begin{itemize}
\item Cours : revoir le cours de la semaine.
\item Faire la fiche d'entraînement n° 5 : .
\item Exercices : feuille n° 23, ex. 18, et ex. 14 pour les plus courageux.\vspace{.4cm}
\end{itemize}

\noindent\textbf{\bf Remarques sur la semaine du 11 au 15 mai 2020}
\begin{itemize}
\item Le programme de colle imaginaire est celui-ci : la partie 1 du chapitre XXIII.
\item Feuille de TD n° 23 : les indications et corrections des exercices de la feuille de TD n° 23 se trouvent ICI.
\item Des remarques et corrections d'exercices sur le cours du chapitre XXIII se trouvent ICI et LÀ.\vspace{.4cm}
\end{itemize}

\noindent\textbf{Vendredi 15 mai 2020}
\begin{itemize}
\item Cours : fin de la section 2.2.
\item Exercices : feuille n° 23, ex. 16 et 17.\vspace{.4cm}
\end{itemize}

\noindent\textbf{Jeudi 14 mai 2020}
\begin{itemize}
\item Rendre le DM n° 19 (vous pouvez remplacer l'exercice 3 par l'exercice 6 de la feuille d'entraînement n° 4).
\item Cours : section 2.2, de 2.2.1 à 2.2.9 inclus.
\item Exercices : feuille n° 23, ex. 13 et 15.\vspace{.4cm}
\end{itemize}
 
\noindent\textbf{Mercredi 13 mai 2020}
\begin{itemize}
\item Cours : \textbf{Chapitre XXIV} - Matrices. Partie 1 et section 2.1 (dans la partie 1, seule la section 1.3 sera vue en cours, tout ce qui précède à déjà été étudié dans le chapitre II, et je vous laisse le relire).
\item Exercices : feuille n° 23, ex. 11 et 12.\vspace{.4cm}
\end{itemize}
 
\noindent\textbf{Lundi 11 mai 2020}
\begin{itemize}
\item Interrogation surprise n° 19.
\item Cours : section 2.7 (fin du chapitre de probas).
\item Exercices : feuille n° 23, ex. 10.\vspace{.4cm}
\end{itemize}

\noindent\textbf{Week-end du 8 au 10 mai 2020}
\begin{itemize}
\item Cours : revoir le cours de la semaine.
\item Faire la fiche d'entraînement n° 4 : .
\item Exercices : feuille n° 23, ex. 9.\vspace{.4cm}
\end{itemize}

\noindent\textbf{\bf Remarques sur la semaine du 4 au 8 mai 2020}
\begin{itemize}
\item Le programme de colle imaginaire est celui-ci : tout le chapitre XXII sur la dimension finie.
\item Feuille de TD n° 23 : les indications et corrections des exercices de la feuille de TD n° 23 se trouvent ICI.
\item Des remarques et corrections d'exercices sur le cours du chapitre XXIII se trouvent ICI et LÀ.\vspace{.4cm}
\end{itemize}
 
\noindent\textbf{Jeudi 7 mai 2020}
\begin{itemize}
\item Cours : fin de la section 2.6.
\item Exercices : feuille n° 23, ex. 6 et 8.\vspace{.4cm}
\end{itemize}
 
\noindent\textbf{Mercredi 6 mai 2020}
\begin{itemize}
\item Cours : fin de la section 2.5 et section 2.6 jusqu'à 2.6.8 inclus.
\item Exercices : feuille n° 23, ex. 4, 5 et 7.\vspace{.4cm}
\end{itemize}
 
\noindent\textbf{Lundi 4 mai 2020}
\begin{itemize}
\item Interrogation surprise n° 18.
\item Cours : section 2.5 jusqu'à 2.5.17 inclus.
\item Exercices : feuille n° 23, ex. 1 à 3.\vspace{.4cm}
\end{itemize}
 
\noindent\textbf{\bf Vacances de printemps}.\vspace{.4cm}\\
\begin{itemize}
\item Cours : revoir le cours de la semaine et finir la section 2.4 (exercice 2.4.12 corrigé ICI).
\item Exercices : feuille n° 22, ex. 16, 19 et 20 à faire.
\item Faire la fiche d'entraînement n° 3 : 
\item Le devoir facultatif n° 7 se trouve ici (en bas de la page, après les DM).
\vspace{.4cm}
\end{itemize}

\noindent\textbf{\bf Remarques sur la semaine du 13 au 17 avril 2020}
\begin{itemize}
\item Le programme de colle imaginaire est celui-ci : tout le chapitre XXI sur le dénombrement.
\item Feuille de TD n° 22 : les indications et corrections des exercices de la feuille de TD n° 22 se trouvent ICI.
\item Feuille de TD n° 23 : les indications et corrections des exercices de la feuille de TD n° 23 se trouvent ICI.
\item Des remarques et corrections d'exercices sur le cours du chapitre XXIII se trouvent ICI et LÀ.\vspace{.4cm}
\end{itemize}

\noindent\textbf{\bf Vendredi 17 avril 2020}
\begin{itemize}
\item Devoir surveillé n° 8 : énoncé donné à 13h30 (sur le site et sur Google Classroom), feuille de calculs d'1h00 à déposer sur Google Classroom de 14h30 à 14h45, puis devoir de 3h00, à déposer sur Google Classroom de 17h45 à 18h00.
\item Cours : Probabilités : 2.3 et 2.4.
\item Exercices : feuille n° 22, ex. 17 et 18.
\item Classe virtuelle à 9h50.\vspace{.4cm}
\end{itemize}

\noindent\textbf{Jeudi 16 avril 2020}
\begin{itemize}
\item DM18 à rendre.
\item Cours : Probabilités : 2.1 et 2.2.
\item Exercices : feuille n° 22, ex. 14 et 15.
\item Classe virtuelle à 14h00.\vspace{.4cm}
\end{itemize}

\noindent\textbf{Mercredi 15 avril 2020}
\begin{itemize}
\item Cours : Probabilités : fin partie 1.3, et partie 1.4.
\item Exercices : feuille n° 22, ex. 12 et 13.
\item Classe virtuelle à 14h00.\vspace{.4cm}
\end{itemize}
 
\noindent\textbf{\bf Remarques sur la semaine du 6 au 12 avril 2020}
\begin{itemize}
\item Le programme de colle imaginaire est celui-ci : tout le chapitre XX sur l'intégration.
\item Feuille de TD n° 22 : les indications et corrections des exercices de la feuille de TD n° 22 se trouvent ICI.
\item Des remarques et corrections d'exercices sur le cours des chapitres XXII et XXIII se trouvent ICI et LÀ.\vspace{.4cm}
\end{itemize}

\noindent\textbf{Week-end du 11 et 12 avril 2020}
\begin{itemize}
\item Cours : revoir le cours de la semaine, et l'exercice 1.3.12 (vidéo ICI).
\item Exercices : feuille n° 22, ex. 9 et 11.
\item Faire la fiche d'entrainement n° 2 : 
\vspace{.4cm}
\end{itemize}

\noindent\textbf{\bf Vendredi 10 avril 2020}
\begin{itemize}
\item Cours : Probabilités : sections 1.2.c, 1.2.d, 1.3.a. et 1.3.b.
\item Exercices : feuille n° 22, ex. 8 et 10.
\item Classe virtuelle à 9h50.\vspace{.4cm}
\end{itemize}

\noindent\textbf{Jeudi 9 avril 2020}
\begin{itemize}
\item DM17 à rendre.
\item Cours : \bf Chapitre XXIII \rm : Probabilités : Sections 1.1, 1.2.a et 1.2.b.
\item Exercices : feuille n° 22, ex 6 et 7.
\item Classe virtuelle à 14h00.\vspace{.4cm}
\end{itemize}

\noindent\textbf{Mercredi 8 avril 2020}
\begin{itemize}
\item Cours : Dimension finie, partie 4.
\item Exercices : feuille n° 22, ex. 3, 4 et 5.
\item Classe virtuelle à 14h00.\vspace{.4cm}
\end{itemize}

\noindent\textbf{\bf Lundi 6 avril 2020}
\begin{itemize}
\item Interrogation n° 17.
\item Cours : Dimension finie : Partie 3 (consulter la vidéo sur la partie 1 avant la classe virtuelle de 14h00).
\item Exercices : feuille n° 22, ex. 1 et 2.
\item Classe virtuelle à 14h00.\vspace{.4cm}
\end{itemize}


\noindent\textbf{\bf Remarques sur la semaine du 30 mars au 5 avril 2020}
\begin{itemize}
\item Le programme de colle imaginaire est celui-ci : tout le chapitre XX sur les familles de vecteurs et les applications linéaires.
\item Feuille de TD n° 20 : les indications et corrections des exercices de la feuille de TD n° 20 se trouvent ICI. Les exercices 19 et 20 ne sont pas à chercher. Seuls ceux qui ont le courage et le temps peuvent s'y attaquer. Leur correction se trouve sur le site de la classe.
\item Des remarques et corrections d'exercices sur le cours des chapitres XXI et XXII se trouvent ICI et LÀ.
\item Les indications et corrections des exercices de la feuille de TD n° 21 se trouvent ICI.
\item Les indications et corrections des exercices de la feuille de TD n° 22 se trouvent ICI.\vspace{.4cm}
\end{itemize}

\noindent\textbf{Week-end du 4 et 5 avril 2020}
\begin{itemize}
\item Cours : revoir le cours de la semaine.
\item Exercices : feuille n° 21, ex. 6 et 8.
\item Faire la fiche d'entraînement n° 1 : 
\vspace{.4cm}
\end{itemize}

\noindent\textbf{\bf Vendredi 3 avril 2020}
\begin{itemize}
\item Cours : Dimension finie, fin partie 2.
\item Exercices : feuille n° 21, ex. 5 et 7.
\item Classe virtuelle à 9h50.\vspace{.4cm}
\end{itemize}

\noindent\textbf{Jeudi 2 avril 2020}
\begin{itemize}
\item DM16 à rendre.
\item Cours : Dimension finie, fin de la partie 1, et partie 2, sections 2.1 et 2.2.
\item Exercices : feuille n° 21, ex.3.
\item Classe virtuelle à 14h00.\vspace{.4cm}
\end{itemize}

\noindent\textbf{Mercredi 1er avril 2020}
\begin{itemize}
\item Cours : Dimension finie, partie 1, sections 1.3 à 1.5.
\item Exercices : feuille n° 20, ex. 21 ; feuille n° 21, ex. 1 et 2.
\item Classe virtuelle à 14h00.\vspace{.4cm}
\end{itemize}

\noindent\textbf{\bf Lundi 30 mars 2020}
\begin{itemize}
\item Cours : \textbf{Chapitre XXII :} Dimension finie : Partie 1, sections 1.1 à 1.2.
\item Exercices : feuille n° 20, ex. 26 à 30.
\item Classe virtuelle à 14h00.\vspace{.4cm}
\end{itemize}

\noindent\textbf{Week-end du 28-30 mars 2020}
\begin{itemize}
\item Cours : revoir le cours de la semaine.
\item Exercices : feuille n° 20, ex. 18, 22, 23 et 24.
\item Faire l'\og interrogation \fg n° 16 : 
\vspace{.4cm}
\end{itemize}

\noindent\textbf{\bf Remarques sur la semaine du 23 au 28 mars 2020}
\begin{itemize}
\item Le programme de colle imaginaire est celui-ci : tout le chapitre d'analyse asymptotique.
\item Des remarques et corrections d'exercices sur le cours du chapitre XX se trouvent ICI.
\item Les indications et corrections des exercices de la feuille de TD n° 20 se trouvent ICI.
\item Les exercices 4, 10 et 11 ne sont pas à chercher. Seuls ceux qui ont le courage et le temps peuvent s'y attaquer. Leur correction se trouve sur le site de la classe, à l'adresse donnée au point précédent.\vspace{.4cm}
\end{itemize}

\noindent\textbf{\bf Vendredi 28 mars 2020}
\begin{itemize}
\item Cours : Dénombrement, partie 2.
\item Exercices : feuille n° 20, ex. 16 et 17.
\item Classe virtuelle à 9h50.\vspace{.4cm}
\end{itemize}

\noindent\textbf{Jeudi 26 mars 2020}
\begin{itemize}
\item Cours : \bf Chapitre XXI \rm : Dénombrement, partie 1.
\item Exercices : feuille n° 20, ex. 12 à 15 (exercices de calculs, révision des exercices d'intégration vus au début du chapitre sur les équations différentielles).
\item Classe virtuelle à 13h30.\vspace{.4cm}
\end{itemize}

\noindent\textbf{Mercredi 25 mars 2020}
\begin{itemize}
\item Cours : Intégration : fin (parties 7, 8 et 9).
\item Exercices : feuille n° 20, ex. 7, 8 et 9.
\item Classe virtuelle à 14h00.\vspace{.4cm}
\end{itemize}

\noindent\textbf{\bf Lundi 23 mars 2020}
\begin{itemize}
\item Seconde partie du DM15 à rendre.
\item Cours : Intégration, parties 3, 4, 5, et 6 (les parties 3 et 4 ne sont que des révisions, et la partie 6 n'est qu'une extension aux fonctions complexes).
\item Exercices : feuille n° 20, ex. 1, 2, 3, 5 et 6.
\item Classe virtuelle à 14h00.\vspace{.4cm}
\end{itemize}

\noindent\textbf{Week-end du 21-22 mars 2020}
\begin{itemize}
\item Cours : revoir le cours de la semaine.
\item Exercices : feuille n° 19, ex. 13 et 15.\vspace{.4cm}
\end{itemize}


\noindent\textbf{\bf Vendredi 20 mars 2020}
\begin{itemize}
\item Devoir surveillé n° 7 : énoncé donné à 13h30 (sur le site et sur Google Classroom), feuille de calculs d'1h00 à déposer sur Google Classroom de 14h30 à 14h45, puis devoir de 3h00, à déposer sur Google Classroom de 17h45 à 18h00.
\item Cours : chapitre XX, fin de la partie 2.
\item Exercices : feuille n° 19, ex. 12 et 14.
\item Chat à 10h00.\vspace{.4cm}
\end{itemize}
 
\noindent\textbf{Jeudi 19 mars 2020}
\begin{itemize}
\item Cours : \bf Chapitre XX \rm : Intégration : partie 1, et section 2.1.
\item Exercices : feuille n° 19, ex. 8, 9 et 11.
\item Chat sur la classe virtuelle de 9h50 à 11h40.\vspace{.4cm}
\end{itemize}

\noindent\textbf{Mercredi 18 mars 2020}
\begin{itemize}
\item Cours : fin du chapitre XIX (vidéo complémentaire sur les projecteurs et symétries en ligne prochainement).
\item Exercices : feuille n° 18, ex. 29, et feuille n° 19, ex. 6 et 10.
\item Chat à 15h15.\vspace{.4cm}
\end{itemize}

\noindent\textbf{\bf Lundi 16 mars 2020}
\begin{itemize}
\item Exercices du matin (TD) : feuille n° 19, ex. 1, 2, 3 et 7 (corrections mises en ligne vers 11h00). Ce sont essentiellement des questions d'applications directes du cours. Seule la question 3 de l'exercice 3 peut poser problème.
\item Interrogation à 14h00, durée : 15 min (précisions à venir).
\item Cours : chapitre XIX, toute la partie 2 (vidéo complémentaire en ligne prochainement).
\item Exercices de l'après-midi : feuille n° 19, ex. 4 et 5 (indications en ligne sur la page TD du site ; corrections mises en ligne dans l'après-midi).
\item Séance de chat à 16h00.
\vspace{.4cm}
\end{itemize}

\noindent\textbf{Week-end du 14-15 mars 2020}
\begin{itemize}
\item Cours : revoir le cours de la semaine.
\item Exercices : feuille n° 18, ex. 26 à 28 (corrections en ligne dimanche après-midi).\vspace{.4cm}
\end{itemize}


\label{end}
\end{document}


