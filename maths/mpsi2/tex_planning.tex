\documentclass[12pt,a4paper]{article}

\textheight=25cm
\topmargin=-50pt
\input{/home/skanderk/.latex/intro2.sty}

\begin{document}

\begin{center}
\Large\bf PROGRAMME DE TRAVAIL À DISTANCE DE MATHÉMATIQUES\\
MPSI 2 La Martinière Monplaisir\\ Mars 2020
\end{center}
\vspace{1cm}
\vspace{.4cm}

% 
% \noindent\textbf{\bf Jeudi 18 juin 2019} \\
% 
% 
% \noindent\textbf{\bf Mercredi 17 juin 2019} \\
% 
% \noindent\textbf{\bf Lundi 15 juin 2019} \\
% \bu\ Interrogation n° 24.\\
% \bu\ Cours : \bf Chapitre XXV \rm : Dénombrement : fin.\\
% \bu\ Exercices : feuille n° 24, ex. 20 à 22 et feuille n° 25, ex. 2, 3, 6 et 7.\vspace{.4cm}\\
% 
% \noindent\textbf{\bf Vendredi 12 juin 2019} \\
% \bu\ Exercices : feuille n° 24, ex. 17 à 19.\vspace{.4cm}\\
% 
% \noindent\textbf{\bf Jeudi 11 juin 2019} \\
% \bu\ Exercices : feuille n° 24, ex. 12 à 16.\vspace{.4cm}\\
% 
% \noindent\textbf{\bf Mercredi 10 juin 2019} \\
% \bu\ Cours : 3 - Automorphismes orthogonaux (fin).\\
% \bu\ Exercices : feuille n° 24, ex. 6 à 11.\vspace{.4cm}\\
% 
% \noindent\textbf{\bf Lundi 08 juin 2019} \\
% \bu\ Interrogation n° 23.\\
% \bu\ Cours : 3 - Automorphismes orthogonaux (suite).\\
% \bu\ Exercices : feuille n° 23, ex. 14 et feuille n° 24, ex. 1 à 5.\vspace{.4cm}\\
% 
% \noindent\textbf{Vendredi 05 juin 2019}\\ 

% 
% \noindent\textbf{ Mercredi 29 avril 2019} \\
% \bu\ Cours : 4 - Séries absolument convergentes ; 5 - Représentation décimale des réels ; 6 - 
% Compléments.\\
% \bu\ Exercices : feuille n° 20, ex. 19, et feuille n° 21, ex. 1.\vspace{.4cm}\\
%

%\bu\ Exercices : feuille n° 20, ex. 11 et 13 à 18.\vspace{.4cm}\\

% \noindent\textbf{\bf Lundi 13 juin 2020}\\
%  ; 2 - Séries à termes réels positifs ; 3 - Comparaison série - intégrale.\\
% \bu\ Exercices : feuille n° 25, ex. 14 à 22.\vspace{.4cm}\\
% 
% \noindent\textbf{Vendredi 10 juin 2020}\\
% \bu\ Devoir surveillé n° 10.\\
% \bu\ Exercices : feuille n° 25, ex. 12 et 13.\vspace{.4cm}\\

% \noindent\textbf{Vendredi 09 juin 2020}\\
% \bu\ Cours : \bf Chapitre XXV \rm : Séries : 1 - Prolégomènes.\\
% \bu\ Exercices : feuille n° 25, ex. 20 et 16.\vspace{.4cm}\\
% 
% \noindent\textbf{\bf Jeudi 08 juin 2020}\\
% \bu\ Exercices : feuille n° 25, ex. 12, 14, 15, 17, 19 et 21.\vspace{.4cm}\\
% 
% \noindent\textbf{\bf Mardi 06 juin 2020} \\
% \bu\ Interrogation n° 19.\\
% \bu\ Cours : 3 - Automorphismes orthogonaux (fin).\\
% \bu\ Exercices : feuille n° 25, ex. 9 et 10.\\
% \bu\ À faire pour jeudi 08 juin : feuille n° 25, ex. 21 et 22.\vspace{.4cm}\\
% 
% \noindent\textbf{Vendredi 02 juin 2020}\\
% \bu\ Cours : 3 - Automorphismes orthogonaux (suite).\\
% \bu\ Exercices : feuille n° 25, ex. 11 et 13.\vspace{.4cm}\\
% 
% \noindent\textbf{\bf Jeudi 01 juin 2020}\\
% \bu\ Distribution : DM n° 20 (à rendre le 08 juin).\\
% \bu\ Cours : 3 - Automorphismes orthogonaux (début).\\
% \bu\ Exercices : feuille n° 25, ex. 7 et 8.\vspace{.4cm}\\
% 
% \noindent\textbf{Mardi 30 mai 2020}\\
% \bu\ Cours : 2 - Orthogonalité (fin).\\
% \bu\ Exercices : feuille n° 25, ex. 5 et 6.\vspace{.4cm}\\
% 
% \noindent\textbf{\bf Lundi 29 mai 2020} \\
% \bu\ Interrogation n° 18.\\
% \bu\ Cours :  2 - Orthogonalité (suite).\\
% \bu\ Exercices : feuille n° 24, ex. 14 et 17, et feuille n° 25, ex. 1 à 4.\vspace{.4cm}\\
% 
% \noindent\textbf{Mardi 23 mai 2020}\\
% \bu\ Cours : \bf Chapitre XXIV \rm : Espaces vectoriels euclidiens : 1 - Produits scalaires, normes et distances 
% (fin).\\
% .\vspace{.4cm}\\
% 
% \noindent\textbf{\bf Lundi 22 mai 2020} \\
% \bu\ Interrogation n° 17.\\
% \bf Chapitre XXIV \rm : Espaces vectoriels euclidiens : 1 - Produits scalaires, normes et distances 
% (suite).\\
% \bu\ Exercices : feuille n° 24, ex. 10, 12 et 13.\vspace{.4cm}\\
% 
% \noindent\textbf{Vendredi 19 mai 2020}\\
% (début).\\
% \bu\ Exercices : feuille n° 24, ex. 1 à 5.\vspace{.4cm}\\
% 
% \noindent\textbf{\bf Jeudi 18 mai 2020} \\
% \bu\ Distribution : DM n° 19 (à rendre le 01 juin).\\
% 
% 
% \noindent\textbf{Mercredi 12 juin 2020}\\
% \bu\ Cours :  2 - Orthogonalité (début).\\
% \bu\ Exercices : feuille n° 25, ex. 8 à 14.\vspace{.4cm}\\
% 
% \noindent\textbf{Vendredi 07 juin 2020}\\
% \bf Chapitre XXVI \rm : Espaces vectoriels euclidiens : 1 - Produits scalaires, normes et distances .\\
% \bu\ Exercices : feuille n° 25, ex. 14.1).\vspace{.4cm}\\
%  
% \noindent\textbf{Jeudi 06 juin 2020}\\
% \bu\ Cours : 5 - Déterminant d'une matrice carrée.
% \bu\ Exercices : feuille n° 25, ex. 6 et 7.\vspace{.4cm}\\
% 
% \noindent\textbf{Mercredi 05 juin 2020}\\
% \bu\ Distribution : DM n° 22 (à rendre le 13 juin).\\
% \bu\ Exercices : feuille n° 25, ex. 1 à 5.\vspace{.4cm}\\
%  
% \noindent\textbf{Lundi 3 juin 2020}\\
% \bu\ Interrogation n° 21.\\
% \bu\ Cours : 2. Applications multilinéaires - 3. Déterminant d’une famille de vecteurs.\\
% \bu\ Exercices : feuille n° 24, ex. 12 à 20.\vspace{.4cm}\\
% 
% \noindent\textbf{Mercredi 29 mai 2020}\\
% \bu\ Cours : \bf Chapitre XXV \rm : Déterminants : 1 - Groupe symétrique.\\
% \bu\ Exercices : feuille n° 24, ex. 2, 3 et 11.\vspace{.4cm}\\
% 
% \noindent\textbf{Lundi 27 mai 2020}\\
% \bu\ Cours : 7 - Matrices semblables et trace ; 8 - Matrices par blocs.\\
% \bu\ Exercices : feuille n° 24, ex. 7 à 10.\vspace{.4cm}\\
% 
% \noindent\textbf{Vendredi 24 mai 2020}\\
% \bu\ Devoir surveillé n° 09.\\
% \bu\ Cours : 5 - Rang ; 6 - Systèmes linéaires.\\
% \bu\ Distribution : DM n° 21 (à rendre le 6 juin).\\
% \bu\ Exercices : feuille n° 24, ex. 1 et 4 à 6.\vspace{.4cm}\\
%  
% \noindent\textbf{Jeudi 23 mai 2020}\\
% \bu\ Cours : 3 - Matrices remarquables ; 4 - Opérations élémentaires.\\
% \bu\ Exercices : feuille n° 23, ex. 14, 17 et 18.\vspace{.4cm}\\
% 
% \noindent\textbf{Mercredi 22 mai 2020}\\
% \bu\ Cours : 2 - Matrices, familles de vecteurs et applications linéaires (fin).\\
% \bu\ Exercices : feuille n° 23, ex. 11 à 13.\vspace{.4cm}\\
%  
% \noindent\textbf{Lundi 20 mai 2020}\\
% \bu\ Interrogation n° 20.\\
% \bu\ Cours : 2 - Matrices, familles de vecteurs et applications linéaires (suite).\\
% \bu\ Exercices : feuille n° 23, ex. 11 à 13.\\
% \bu\ À faire pour mercredi 22 : feuille n° 23, ex. 17.\\
% \bu\ À faire pour jeudi 23 : feuille n° 24, ex. 01 et 04.\\
% \bu\ À faire pour vendredi 24 : feuille n° 24, ex. 05 et 06.\vspace{.4cm}\\
%  
% \noindent\textbf{Vendredi 17 mai 2020}\\
% \bu\ Cours : 2 - Matrices, familles de vecteurs et applications linéaires (suite).\\
% \bu\ Exercices : feuille n° 23, ex. 08 et 16.\vspace{.4cm}\\
% 
% \noindent\textbf{Jeudi 16 mai 2020}\\
% \bu\ Cours : 2 - Matrices, familles de vecteurs et applications linéaires (suite).\\
% \bu\ Exercices : feuille n° 23, ex. 07, 10 et 15.\vspace{.4cm}\\
% 
% \noindent\textbf{Mercredi 15 mai 2020}\\
% \bu\ Distribution : DM n° 20 (à rendre le 23 mai).\\
% \bu\ Cours : \bf Chapitre XXIV \rm : Matrices ; 1 - Structure de $\mcal M_{n,p}(\K)$ ; 2 - Matrices, familles de 
% vecteurs et applications linéaires (début).\\
% \bu\ Exercices : feuille n° 23, ex. 06 et 09.\vspace{.4cm}\\
% 
% \noindent\textbf{Lundi 13 mai 2020}\\
% \bu\ Interrogation surprise n° 19.\\
% \bu\ Cours : 2 - Variables aléatoires (fin).\\
% \bu\ Exercices : feuille n° 23, ex. 01 à 05.\\
% \bu\ À faire pour mercredi 15 : feuille n° 23, ex. 06.\\
% \bu\ À faire pour jeudi 16 : feuille n° 23, ex. 07.\\
% \bu\ À faire pour vendredi 17 : feuille n° 23, ex. 08.\vspace{.4cm}\\
% 
% \noindent\textbf{Vendredi 10 mai 2020}\\
% \bu\ Cours : 2 - Variables aléatoires (suite).\\
% \bu\ Exercices : feuille n° 22, ex. 15 et 17 à 20.\vspace{.4cm}\\
% 
% \noindent\textbf{Jeudi 09 mai 2020}\\
% \bu\ Distribution : DM n° 20 (à rendre le 16 mai).\\
% \bu\ Cours : 2 - Variables aléatoires (suite).\\
% \bu\ Exercices : feuille n° 22, ex. 12 et 16.\vspace{.4cm}\\
% 
% \noindent\textbf{Lundi 06 mai 2020}\\
% \bu\ Cours : 2 - Variables aléatoires (suite).\\
% \bu\ Exercices : feuille n° 22, ex. 06 à 11, 13 et 14.\\
% \bu\ À faire pour jeudi 09 : feuille n° 22, ex. 12.\\
% \bu\ À faire pour vendredi 10 : feuille n° 22, ex. 15.\vspace{.4cm}\\
% 
% \noindent\textbf{Vendredi 03 mai 2020}\\
% \bu\ Devoir surveillé n° 08.\\
% \bu\ Cours : 2 - Variables aléatoires (début).\\
% \bu\ Exercices : feuille n° 22, ex. 02.\\
% \bu\ À faire pour lundi 06 : feuille n° 22, ex. 06 et 07.\vspace{.4cm}\\
% 
% \noindent\textbf{Jeudi 02 mai 2020}\\
% \bu\ Cours : 1 - Événements (fin).\\
% \bu\ Exercices : feuille n° 21, ex. 03 et feuille n° 22, ex. 04 et 05.\vspace{.4cm}\\
% 
% \noindent\textbf{Lundi 29 avril 2020}\\
% \bu\ Interrogation surprise n° 18.\\
% \bu\ Cours : Événements (suite).\\
% \bu\ Exercices : feuille n° 21, ex. 02, 04 et 06 à 08, et feuille n° 22, ex. 01.\\
% \bu\ À faire pour jeudi 02 : feuille n° 21, ex. 03.\\
% \bu\ À faire pour vendredi 03 : feuille n° 22, ex. 02 et 03.\vspace{.4cm}\\
% 
% \noindent\textbf{\bf Vacances de printemps}.\vspace{.4cm}\\
% 
% \noindent\textbf{Vendredi 11 avril 2020}\\
% \bu\ Cours : \bf Chapitre XXIII \rm : Probabilités : 1 - Événements (début).\\
% \bu\ Exercices : feuille n° 21, ex. 01 et 05.\vspace{.4cm}\\
% 
% \noindent\textbf{Jeudi 10 avril 2020}\\
% \bu\ Distribution : DM n° 18 (à rendre le 02 mai).\\
% \bu\ Cours : 3 - Applications linéaires en dimension finie ; 4 - Formes linéaires et hyperplans.\\
% \bu\ Exercices : feuille n° 20, ex. 20 et 27 à 28.\vspace{.4cm}\\
% 
% \noindent\textbf{Mercedi 09 avril 2020}\\
% \bu\ Cours : 2 - Sev en dimension finie (fin).\\
% \bu\ Exercices : feuille n° 20, ex. 19.\vspace{.4cm}\\
% 
% \noindent\textbf{Lundi 08 avril 2020}\\
% \bu\ Interrogations surprises n° 17 et 17bis.\\
% \bu\ Cours : 2 - Sev en dimension finie (début).\\
% \bu\ Exercices : feuille n° 20, ex. 21, 22, 25, 26 et 30.\\
% \bu\ À faire pour mercredi 10 : feuille n° 20, ex. 19.\\
% \bu\ À faire pour jeudi 11 : feuille n° 20, ex. 27 et 28.\\
% \bu\ À faire pour vendredi 12 : feuille n° 21, ex. 05.\vspace{.4cm}\\
% 
% \noindent\textbf{Jeudi 04 avril 2020}\\
% \bu\ Cours : 1 - Notion de dimension (fin).\\
% \bu\ Exercices : feuille n° 20, ex. 09, 12, 16 et 17.\vspace{.4cm}\\
% 
% \noindent\textbf{Mercedi 03 avril 2020}\\
% \bu\ Cours : 1 - Notion de dimension (début).\\
% \bu\ Exercices : feuille n° 20, ex. 08, 10 et 11.\vspace{.4cm}\\
% 
% \noindent\textbf{Lundi 01 avril 2020}\\
% \bu\ Cours : Dénombrement (fin).\\
% \bf Chapitre XXII \rm : Espaces vectoriels de dimension finie  : 1 - Notion de dimension (début).\\
% \bu\ Exercices : feuille n° 20, ex. 03 à 07.\\
% \bu\ À faire pour mercredi 03 : feuille n° 20, ex. 08.\\
% \bu\ À faire pour jeudi 04 : feuille n° 20, ex. 09.\vspace{.4cm}\\
% 
% \noindent\textbf{\bf Vendredi 29 mars 2020} \\
% \bu\ Devoir surveillé n° 07.\\
% \bu\ Cours : \bf Chapitre XXI \rm : Dénombrement (début).\\
% \bu\ À faire pour lundi 01 : feuille n° 19, ex. 12, et feuille n° 20, ex. 01 et 02.\vspace{.4cm}\\
% 
% \noindent\textbf{Jeudi 28 mars 2020}\\
% \bu\ Cours : Intégration (fin).\\
% \bu\ Exercices : feuille n° 19, ex. 11, 13 et 15.\vspace{.4cm}\\
% 
% \noindent\textbf{Mercredi 27 mars 2020} \\
% \bu\ Cours : Intégration (suite).\\
% \bu\ Exercices : feuille n° 19, ex. 06, 07 et 14.\vspace{.4cm}\\
% 
% \noindent\textbf{\bf Lundi 25 mars 2020} \\
% \bu\ Interrogation surprise n° 16.\\
% \bf Chapitre XX \rm : Intégration (début).\vspace{.4cm}\\
% \bu\ Exercices : feuille n° 19, ex. 04, 05 et 08 à 10.\\
% \bu\ À faire pour mercredi 27 : feuille n° 19, ex. 07.\\
% \bu\ À faire pour jeudi 28 : feuille n° 19, ex. 11.\\
% \bu\ À faire pour vendredi 29 : feuille n° 19, ex. 12.\vspace{.4cm}\\
% 
% \noindent\textbf{\bf Vendredi 22 mars 2020} \\
% \bu\ Cours : Applications linéaires et familles de vecteurs (fin).\\
% \bu\ Exercices : feuille n° 19, ex. 01 et 03.\vspace{.4cm}\\
% 
% \noindent\textbf{Jeudi 21 mars 2020}\\
% \bu\ Cours : Applications linéaires et familles de vecteurs (suite).\\
% \bu\ Exercices : feuille n° 18, ex. 32.\vspace{.4cm}\\
% 
% \noindent\textbf{Mercredi 20 mars 2020} \\
% \bu\ Cours : Applications linéaires et familles de vecteurs (suite).\\
% \bu\ Exercices : feuille n° 18, ex. 28, 29 et 31.\vspace{.4cm}\\
% 
% 
% \bu\ Interrogation surprise n° 15.\\
% \bu\ Distribution : DM n° 16 (à rendre le 28 mars).\\
% \bu\ Cours : Applications linéaires et familles de vecteurs (suite).\\
% \bu\ Exercices : feuille n° 18, ex. 20, 21, 25 à 27 et 30.\\
% \bu\ À faire pour mercredi 20 : feuille n° 18, ex. 29.\\
% \bu\ À faire pour jeudi 21 : feuille n° 18, ex. 32.\\
% \bu\ À faire pour vendredi 22 : feuille n° 19, ex. 01 et 03.\vspace{.4cm}\\

\noindent\textbf{\bf Lundi 16 mars 2020} \\
\bu\ Exercices du matin (TD) : feuille n° 19, ex. 1, 2, 3 et 7 (corrections mises en ligne en fin de matinée). Ce sont essentiellement des questions d'applications directes du cours. Seule la question 3 de l'exercice 3 peut poser problème.\\
\bu\ Interrogation à 14h00, durée : 15 min (précisions à venir).\\
\bu\ Cours : chapitre 19, toute la partie 2 (vidéo complémentaire en ligne prochainement).\\
\bu\ Exercices de l'après-midi : feuille n° 19, ex. 4 et 5 (indications en ligne sur la page TD du site ; corrections mises en ligne dans l'après-midi).\\
\bu\ On essaie de faire une séance de chat à 16h00.
\vspace{.4cm}\\

\noindent\textbf{Week-end du 14-15 mars 2020}\\
\bu\ Cours : revoir le cours de la semaine.\\
\bu\ Exercices : feuille n° 18, ex. 26 à 28 (corrections en ligne dimanche après-midi).\vspace{.4cm}\\




\label{end}
\end{document}


