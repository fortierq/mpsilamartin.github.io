\documentclass[12pt,a4paper]{article}

\textheight=25cm
\topmargin=-50pt
\textwidth= 17cm
\oddsidemargin=-.5cm


\usepackage{enumerate}
\usepackage{stmaryrd} %\sslash chapitre 19 au moins (parallèle)
\usepackage{amsfonts}
\usepackage{fancybox}
\usepackage{color}
\usepackage{eurosym}
\usepackage{amssymb}
\usepackage[T1]{fontenc}
\usepackage{amsmath}
\usepackage{theorem}
\usepackage[french]{babel}
\usepackage[utf8]{inputenc}
\usepackage{latexsym}
\usepackage{amscd}
\usepackage{indentfirst}
\usepackage[dvips]{graphicx}
\usepackage{textcomp}
\usepackage{mathrsfs}

\newcommand{\G}{\ensuremath{\mcal{G}}}
\newcommand{\cG}{\ensuremath{\mcal{G}}}
\newcommand{\cL}{\ensuremath{\mcal{L}}}
\newcommand{\lin}{\ensuremath{\mcal{L}}}
\newcommand{\rmdim}{\dim}
\newcommand{\dprod}{\displaystyle\prod}
\newcommand{\dsum}{\displaystyle\sum}
\newcommand{\intt}{\displaystyle\int}
\newcommand{\ml}{\ensuremath{\mcal{L}}}
\newcommand{\E}{\ensuremath{\mathbb{E}}}
\newcommand{\I}{\ensuremath{\mathcal{I}}}
\newcommand{\N}{\ensuremath{\field{N}}}
\newcommand{\Z}{\ensuremath{\field{Z}}}
\newcommand{\D}{\ensuremath{\field{D}}}
\newcommand{\M}{\ensuremath{\mcal{M}}}
\newcommand{\Q}{\ensuremath{\field{Q}}}
\newcommand{\R}{\ensuremath{\field{R}}}
\newcommand{\Rp}{\ensuremath{\field{R}_+}}
\newcommand{\Rpe}{\ensuremath{\field{R}_+^{\ast}}}
\newcommand{\Ret}{\ensuremath{\field{R}^{\ast}}}
\newcommand{\Rd}{\ensuremath{\field{R}^2}}
\newcommand{\Rt}{\ensuremath{\field{R}^3}}
\newcommand{\C}{\ensuremath{\field{C}}}
\newcommand{\F}{\ensuremath{\field{F}}}
\newcommand{\U}{\ensuremath{\field{U}}}
\newcommand{\K}{\ensuremath{\field{K}}}
\newcommand{\limcur}{\ensuremath{\text{\cursive{l}}}}
\newcommand{\sfrak}{\ensuremath{\mathfrak{S}}}
\newcommand{\sfrakn}{\ensuremath{\mathfrak{S}_n}}
\newcommand{\bu}{\noindent\ensuremath{\bullet}}
\newcommand{\llbr}{\ensuremath{\llbracket}}
\newcommand{\rrbr}{\ensuremath{\rrbracket}}
\newcommand{\minus}{\ensuremath{\backslash}}
\newcommand{\eps}{\ensuremath{\varepsilon}}
\newcommand{\ssi}{si et seulement si}
\newcommand{\implique}{\Rightarrow}
\newcommand{\pgcd}{\text{pgcd\,}}
\newcommand{\ppcm}{\text{ppcm\,}}
\newcommand{\id}{\text{Id}}
\newcommand{\norm}[1]{\ensuremath{\Vert #1\Vert}}
\newcommand{\ch}{\mathop{\mathrm{ch}}\nolimits}
\newcommand{\sh}{\mathop{\mathrm{sh}}\nolimits}
\newcommand{\Vect}{\mathop{\mathrm{Vect}}\nolimits}
\renewcommand{\tanh}{\mathop{\mathrm{th}}\nolimits}
\renewcommand{\geq}{\geqslant}
\renewcommand{\leq}{\leqslant}
\newcommand{\tq}{\text{ tq }}
\newcommand{\chbox}{\LARGE\Checkedbox\normalsize\ }
\newcommand{\crbox}{\LARGE\Crossedbox\normalsize\ }
\newcommand{\Arcsin}{\mathop{\mathrm{Arcsin}}\nolimits}
\newcommand{\Arccos}{\mathop{\mathrm{Arccos}}\nolimits}
\newcommand{\Arctan}{\mathop{\mathrm{Arctan}}\nolimits}
\newcommand{\Argsh}{\mathop{\mathrm{Argsh}}\nolimits}
\newcommand{\Argch}{\mathop{\mathrm{Argch}}\nolimits}
\newcommand{\Argth}{\mathop{\mathrm{Argth}}\nolimits}
\newcommand{\abs}[1]{\left| #1 \right|}
\newcommand{\tend}{\ensuremath{\underset{n\to +\infty}{\longrightarrow}}}
\newcommand{\inv}{\ensuremath{^{-1}}}
\newcommand{\enx}[1]{\ensuremath{\underset{x\to #1}{=}}}
\newcommand{\simm}{\ensuremath{\underset{n\to +\infty}{\sim}}}
\newcommand{\simx}[1]{\ensuremath{\underset{x\to #1}{\sim}}}
\newcommand{\conj}[1]{\ensuremath{\overline{#1}}}






\begin{document}

\begin{center}
\Large\bf Cahier de textes de mathématiques en MPSI 1 :
\end{center}
\vspace{1cm}
\vspace{.4cm}\\

\noindent\textbf{Lundi 16 octobre 2017}\\
\bu{} Interrogation écrite n°6.\\
\bu{} Cours sur les équations différentielles (V) : parti1 terminée, partie 2 commencée. \\
\bu{} Feuille d'exercices sur les nombres complexes (4) : exercice 14 et exercices 16 à 22 corrigés. \\
\vspace{.4cm}


\noindent\textbf{Vendredi 13 octobre 2017}\\
\bu{} Cours sur les équations différentielles (V) : parties 1.2, 1.3 et 1.4.a terminées. \\
\bu{} Devoir surveillé n°2 : autour de la série harmonique.\\
\vspace{.4cm}

\noindent\textbf{Jeudi 12 octobre 2017}\\
\bu{} Cours sur les complexes (IV) : terminé. \\
\bu{} Cours sur les équations différentielles (V) : partie 1.1 terminée. \\
\bu{} Feuille d'exercices sur les nombres complexes (4) : exercices 5 et 7 à 12 corrigés. \\
\bu{} DM n°5 distribué.\\
\vspace{.4cm}

\noindent\textbf{Mardi 10 octobre 2017}\\
\bu{} Cours sur les complexes (IV) : partie 5 jusqu'à la définition 5.3.5. \\
\bu{} Feuille d'exercices sur les nombres complexes (4) : exercice 6 corrigé. \\
\vspace{.4cm}

\noindent\textbf{Lundi 9 octobre 2017}\\
\bu{} Interrogation écrite n°5.\\
\bu{} Cours sur les complexes (IV) : partie 4 terminée, partie 5 jusqu'à l'expression complexe d'une rotation (5.3). \\
\bu{} Feuille d'exercices de logique (3) : terminée. \\
\bu{} Feuille d'exercices sur les nombres complexes (4) : exercice 1 à 4 corrigés. \\
\vspace{.4cm}

\noindent\textbf{Vendredi 6 octobre 2017}\\
\bu{} Cours sur les complexes (IV) : parties 2 et 3 terminée. \\
\vspace{.4cm}

\noindent\textbf{Jeudi 5 octobre 2017}\\
\bu{} Cours sur les complexes (IV) : partie 1 terminée, partie 2 jusqu'au début de la partie 2.4. \\
\bu{} Feuille d'exercices de logique (3) : exercice 1 à 6 corrigés. \\
\bu{} DM n°4 distribué.\\
\vspace{.4cm}

\noindent\textbf{Mardi 3 octobre 2017}\\
\bu{} Cours sur les complexes (IV) : partie 1 jusqu'à la proposition 1.4.3. \\
\bu{} Feuille d'exercice sur les calculs algébriques (2) : exercice 20 corrigé. \\
\vspace{.4cm}

\noindent\textbf{Lundi 2 octobre 2017}\\
\bu{} Interrogation écrite n°4.\\
\bu{} Cours de logique (III) : terminé. \\
\bu{} Cours sur les complexes (IV) : partie 1.1 terminée. \\
\bu{} Feuille d'exercice sur les calculs algébriques (2) : exercices 15 à 19 corrigés. \\
\vspace{.4cm}

\noindent\textbf{Vendredi 29 septembre 2017}\\
\bu{} Cours de logique (III) : partie 3 terminée, partie 4 avancée jusqu'à la partie 4.4. \\
\vspace{.4cm}

\noindent\textbf{Jeudi 28 septembre 2017}\\
\bu{} Cours sur les calculs algébriques (II) : terminé. \\
\bu{} Cours de logique (III) : parties 1 et 2 terminées. \\
\bu{} Feuille d'exercice sur les calculs algébriques (2) : exercices 7 à 14 corrigés.  \\
\bu{} DM n°3 distribué\\
\vspace{.4cm}

\noindent\textbf{Mardi 26 septembre 2017}\\
\bu{} Cours sur les calculs algébriques (II) : partie 5 presque terminée. \\
\bu{} Feuille d'exercice sur les calculs algébriques (2) : exercice 7 corrigé.  \\
\vspace{.4cm}

\noindent\textbf{Lundi 25 septembre 2017}\\
\bu{} Interrogation écrite n°3.\\
\bu{} Cours sur les calculs algébriques (II) : partie 4 terminée, parties 5.1 et 5.2 terminées. \\
\bu{} Feuille d'exercice sur les calculs algébriques (2) : exercices 1 à 6 corrigés, questions 1 et 2 de l'exercice 7 corrigées.  \\
\vspace{.4cm}

\noindent\textbf{Vendredi 22 septembre 2017}\\
\bu\ Cours sur les calculs algébriques (II) : partie 4 jusqu'à la définition 4.3.6. \\
\bu\ Devoir surveillé n°1 : fonctions usuelles. \\
\vspace{.4cm}

\noindent\textbf{Jeudi 21 septembre 2017}\\
\bu\ Cours sur les calculs algébriques (II) : partie 3 terminée, partie 4 .1 terminée. \\
\bu\ Feuille d'exercice sur les fonctions usuelles (1) : terminée. \\
\vspace{.4cm}

\noindent\textbf{Mardi 19 septembre 2017}\\
\bu\ Cours sur les calculs algébriques (II) : partie 2 terminée, partie 3 jusqu'à la proposition 3.0.10. \\
\bu\ Feuille d'exercice sur les fonctions usuelles (1) : exercice 15 terminé. \\
\vspace{.4cm}

\noindent\textbf{Lundi 18 septembre 2017}\\
\bu\ Interrogation écrite n°2.\\
\bu\ Cours sur les calculs algébriques (II) : partie 1 terminée. \\
\bu\ Feuille d'exercice sur les fonctions usuelles (1) : exercices 9 à 15 corrigés. \\
\vspace{.4cm}

\noindent\textbf{Vendredi 15 septembre 2017}\\
\bu\ Cours sur les fonctions usuelles (I) : terminé. \\
\bu\ Cours sur les calculs algébriques (II) : jusqu'à la proposition 1.0.4. \\
\vspace{.4cm}

\noindent\textbf{Jeudi 14 septembre 2017}\\
\bu\ Cours sur les fonctions usuelles (I) : partie 8 terminée, sinus et cosinus hyperboliques étudiés. \\
\bu\ Feuille d'exercice sur les fonctions usuelles (1) : exercices 7 et 8 corrigés. \\
\vspace{.4cm}

\noindent\textbf{Mardi 12 septembre 2017}\\
\bu\ Cours sur les fonctions usuelles (I) : partie 6 terminée, partie 7 jusqu'à la définition 7.2.1. \\
\vspace{.4cm}

\noindent\textbf{Lundi 11 septembre 2017}\\
\bu\ Interrogation écrite n°1. \\
\bu\ Cours sur les fonctions usuelles (I) : partie 5 terminée, partie 6 jusqu'à la proposition 6.2.5. \\
\bu\ Feuille d'exercice sur les fonctions usuelles (1) : exercices 1 à 6 corrigés. \\
\vspace{.4cm}

\noindent\textbf{Vendredi 8 septembre 2017}\\
\bu\ Cours sur les fonctions usuelles (I) : parties 3 et 4 terminées, partie 5 jusqu'à la définition 5.2.1. \\
\vspace{.4cm}

\noindent\textbf{Jeudi 7 septembre 2017}\\
\bu\ Cours sur les fonctions usuelles (I) : parties 1 et 2 terminées, partie 3 jusqu'à la proposition 3.3.5. \\
\vspace{.4cm}

\noindent\textbf{Mardi 5 septembre 2017}\\
\bu\ Journée de rentrée ; distribution des feuilles de TD, des formulaires, des
chapitres I et II.  \\
\bu\ DM n° 1 distribué (à rendre le 14 septembre). \\
\bu\ Cours sur les fonctions usuelles (I) : partie 1.1. \\
\vspace{.4cm}


\label{end}
\end{document}


