\documentclass[12pt,a4paper]{article}

\textheight=25cm
\topmargin=-50pt
\textwidth= 17cm
\oddsidemargin=-.5cm


\usepackage{enumerate}
\usepackage{stmaryrd} %\sslash chapitre 19 au moins (parallèle)
\usepackage{amsfonts}
\usepackage{fancybox}
\usepackage{color}
\usepackage{eurosym}
\usepackage{amssymb}
\usepackage[T1]{fontenc}
\usepackage{amsmath}
\usepackage{theorem}
\usepackage[french]{babel}
\usepackage[utf8]{inputenc}
\usepackage{latexsym}
\usepackage{amscd}
\usepackage{indentfirst}
\usepackage[dvips]{graphicx}
\usepackage{textcomp}
\usepackage{mathrsfs}

\newcommand{\G}{\ensuremath{\mcal{G}}}
\newcommand{\cG}{\ensuremath{\mcal{G}}}
\newcommand{\cL}{\ensuremath{\mcal{L}}}
\newcommand{\lin}{\ensuremath{\mcal{L}}}
\newcommand{\rmdim}{\dim}
\newcommand{\dprod}{\displaystyle\prod}
\newcommand{\dsum}{\displaystyle\sum}
\newcommand{\intt}{\displaystyle\int}
\newcommand{\ml}{\ensuremath{\mcal{L}}}
\newcommand{\E}{\ensuremath{\mathbb{E}}}
\newcommand{\I}{\ensuremath{\mathcal{I}}}
\newcommand{\N}{\ensuremath{\field{N}}}
\newcommand{\Z}{\ensuremath{\field{Z}}}
\newcommand{\D}{\ensuremath{\field{D}}}
\newcommand{\M}{\ensuremath{\mcal{M}}}
\newcommand{\Q}{\ensuremath{\field{Q}}}
\newcommand{\R}{\ensuremath{\field{R}}}
\newcommand{\Rp}{\ensuremath{\field{R}_+}}
\newcommand{\Rpe}{\ensuremath{\field{R}_+^{\ast}}}
\newcommand{\Ret}{\ensuremath{\field{R}^{\ast}}}
\newcommand{\Rd}{\ensuremath{\field{R}^2}}
\newcommand{\Rt}{\ensuremath{\field{R}^3}}
\newcommand{\C}{\ensuremath{\field{C}}}
\newcommand{\F}{\ensuremath{\field{F}}}
\newcommand{\U}{\ensuremath{\field{U}}}
\newcommand{\K}{\ensuremath{\field{K}}}
\newcommand{\limcur}{\ensuremath{\text{\cursive{l}}}}
\newcommand{\sfrak}{\ensuremath{\mathfrak{S}}}
\newcommand{\sfrakn}{\ensuremath{\mathfrak{S}_n}}
\newcommand{\bu}{\noindent\ensuremath{\bullet}}
\newcommand{\llbr}{\ensuremath{\llbracket}}
\newcommand{\rrbr}{\ensuremath{\rrbracket}}
\newcommand{\minus}{\ensuremath{\backslash}}
\newcommand{\eps}{\ensuremath{\varepsilon}}
\newcommand{\ssi}{si et seulement si}
\newcommand{\implique}{\Rightarrow}
\newcommand{\pgcd}{\text{pgcd\,}}
\newcommand{\ppcm}{\text{ppcm\,}}
\newcommand{\id}{\text{Id}}
\newcommand{\norm}[1]{\ensuremath{\Vert #1\Vert}}
\newcommand{\ch}{\mathop{\mathrm{ch}}\nolimits}
\newcommand{\sh}{\mathop{\mathrm{sh}}\nolimits}
\newcommand{\Vect}{\mathop{\mathrm{Vect}}\nolimits}
\renewcommand{\tanh}{\mathop{\mathrm{th}}\nolimits}
\renewcommand{\geq}{\geqslant}
\renewcommand{\leq}{\leqslant}
\newcommand{\tq}{\text{ tq }}
\newcommand{\chbox}{\LARGE\Checkedbox\normalsize\ }
\newcommand{\crbox}{\LARGE\Crossedbox\normalsize\ }
\newcommand{\Arcsin}{\mathop{\mathrm{Arcsin}}\nolimits}
\newcommand{\Arccos}{\mathop{\mathrm{Arccos}}\nolimits}
\newcommand{\Arctan}{\mathop{\mathrm{Arctan}}\nolimits}
\newcommand{\Argsh}{\mathop{\mathrm{Argsh}}\nolimits}
\newcommand{\Argch}{\mathop{\mathrm{Argch}}\nolimits}
\newcommand{\Argth}{\mathop{\mathrm{Argth}}\nolimits}
\newcommand{\abs}[1]{\left| #1 \right|}
\newcommand{\tend}{\ensuremath{\underset{n\to +\infty}{\longrightarrow}}}
\newcommand{\inv}{\ensuremath{^{-1}}}
\newcommand{\enx}[1]{\ensuremath{\underset{x\to #1}{=}}}
\newcommand{\simm}{\ensuremath{\underset{n\to +\infty}{\sim}}}
\newcommand{\simx}[1]{\ensuremath{\underset{x\to #1}{\sim}}}
\newcommand{\conj}[1]{\ensuremath{\overline{#1}}}






\begin{document}

\begin{center}
\Large\bf Cahier de textes d'informatique (tronc commun) en MPSI 2 :
\end{center}
\vspace{1cm}
\vspace{.4cm}\\

\noindent\textbf{Samedi 10 juin 2017}\\
\bu\ Devoir n°4 (papier et machine) : SQL.\vspace{.4cm}\\

\noindent\textbf{Lundi 22 mai 2017}\\
\bu\ Cours n°16 (algèbre relationnelle) : terminé.\vspace{.4cm}\\

\noindent\textbf{Mercredi 3 mai 2017}\\
\bu\ Exercices de la fin du chapitre n°14 : corrigés.\vspace{.4cm}\\

\noindent\textbf{Samedi 8 avril 2017}\\
\bu\ Devoir n°3 (papier et machine) : fichiers et méthodes numériques. \vspace{.4cm}\\

\noindent\textbf{Lundi 3 avril 2017}\\
\bu\ Cours n°14 (bases de données et SQL) : terminé.\\
\bu\ À faire pour la séance suivante : tous les exercices de la fin du chapitre n°14.\\
\bu\ Pensez à installer sqlite ou SQLite Manager (addon Firefox) sur votre PC pour le prochain cours, ainsi qu'à télécharger la base de données. \vspace{.4cm}\\

\noindent\textbf{Lundi 27 mars 2017}\\
\bu\ Cours n°13 (introduction aux bases de données) : terminé.\\
\bu\ Pensez à installer sqlite ou SQLite Manager (addon Firefox) sur votre PC pour le prochain cours, ainsi qu'à télécharger la base de données. \vspace{.4cm}\\

\noindent\textbf{Lundi 20 mars 2017}\\
\bu\ Cours n°12 (systèmes linéaires) : mise en {\oe}uvre de l'algorithme du pivot de Gauss.\\
\bu\ Pensez à bien manipuler ces exemples simples chez vous avant la séance suivante. \vspace{.4cm}\\

\noindent\textbf{Lundi 13 mars 2017}\\
\bu\ Cours n°11 (résolution d'équations) : terminé. \\
\bu\ Cours n°12 (systèmes linéaires) : rappels théoriques sur l'algorithme du pivot de Gauss.\\
\bu\ Pensez à bien manipuler ces exemples simples chez vous avant la séance suivante. \vspace{.4cm}\\

\noindent\textbf{Lundi 6 mars 2017}\\
\bu\ Cours n°11 (résolution d'équations) : partie 1 terminée, partis 2.1 à 2.4 terminées. \\
\bu\ Pensez à bien manipuler ces exemples simples chez vous avant la séance suivante. \vspace{.4cm}\\

\noindent\textbf{Mercredi 15 février 2017}\\
\bu\ Devoir n°2, à faire à la maison et à rendre les 5 et 6 mars. \vspace{.4cm}\\

\noindent\textbf{Lundi 13 février 2017}\\
\bu\ Cours n°10 (équations différentielles) : terminé. \\
\bu\ Pensez à bien manipuler ces exemples simples chez vous avant la séance suivante. \vspace{.4cm}\\

\noindent\textbf{Lundi 6 février 2017}\\
\bu\ Cours n°10 (équations différentielles) : parties 11 à 15 traitées. \\
\bu\ Pour la séance suivante : traiter l'exemple de la partie 15 (ou de toute autre équation d'ordre supérieur à 1 strictement). N'oubliez pas les portraits de phase ! \vspace{.4cm}\\

\noindent\textbf{Lundi 30 janvier 2017}\\
\bu\ Cours n°10 (équations différentielles) : parties 1 à 10 traitées. \\
\bu\ Pensez à bien manipuler ces exemples simples chez vous avant la séance suivante. \vspace{.4cm}\\

\noindent\textbf{Lundi 23 janvier 2017}\\
\bu\ Cours n°9 (complexité) : terminé. \\
\bu\ À faire pour la séance suivante : exercices n°2 et 4 du chapitre 9.\\
\bu\ Pensez à bien manipuler ces exemples simples chez vous avant la séance suivante. \vspace{.4cm}\\

\noindent\textbf{Lundi 16 janvier 2017}\\
\bu\ Cours n°9 (complexité) : traité jusqu'à la partie 3.5. \\
\bu\ À faire pour la séance suivante : exercice n°1 du chapitre 9.\\
\bu\ Pensez à bien manipuler ces exemples simples chez vous avant la séance suivante. \vspace{.4cm}\\

\noindent\textbf{Lundi 9 janvier 2017}\\
\bu\ Cours n°8 (représentation des nombres) : terminé. \\
\bu\ Pensez à bien manipuler ces exemples simples chez vous avant la séance suivante. \vspace{.4cm}\\

\noindent\textbf{Lundi 12 décembre 2016 }\\
\bu\ Cours n°8 (représentation des nombres) : parties 1 et 2 terminées. \\
\bu\ Pensez à bien manipuler ces exemples simples chez vous avant la séance suivante. \vspace{.4cm}\\

\noindent\textbf{Lundi 5 décembre 2016 }\\
\bu\ Cours n°7 (fichiers) : terminé. \\
\bu\ Pour la séance suivante : écrire un script permettant de lire le fichier etudiants.csv, d'extraire les données des étudiants (ex : dans une liste), de les trier par trinôme de khôlle et de produire un fichier comme celui-ci. 
Pensez à bien utiliser les blocs with, les méthodes split et strip (et à regarder dans un éditeur de texte ainsi que dans Python la structure des objets manipulés). \\
\bu\ Pensez à bien manipuler ces exemples simples chez vous avant la séance suivante. \vspace{.4cm}\\

\noindent\textbf{Samedi 3 décembre 2016}\\
\bu\ Devoir n°1 : 1 h sur papier, 1 h sur machine.\vspace{.4cm}\\

\noindent\textbf{Lundi 28 novembre 2016 }\\
\bu\ Cours n°6 (chaînes de caractères) : terminé. \\
\bu\ Pensez à bien manipuler ces exemples simples chez vous avant la séance suivante. \vspace{.4cm}\\

\noindent\textbf{Lundi 21 novembre 2016 }\\
\bu\ Cours n°5 (tableaux) : terminé. \\
\bu\ Pour la séance suivante : faire les exercices n°3 et 4 du chapitre 5. \\
\bu\ Pensez à bien manipuler ces exemples simples chez vous avant la séance suivante. \vspace{.4cm}\\

\noindent\textbf{Lundi 14 novembre 2016 }\\
\bu\ Cours n°5 (tableaux) : parties 3.1 et 3.2 terminées. \\
\bu\ Pour la séance suivante : faire l'exercice n°1 du chapitre 5. \\
\bu\ Pensez à bien manipuler ces exemples simples chez vous avant la séance suivante. \vspace{.4cm}\\

\noindent\textbf{Lundi 7 novembre 2016 }\\
\bu\ Cours n°5 (tableaux) : parties 1 et 2 terminées. \\
\bu\ Pensez à bien manipuler ces exemples simples chez vous avant la séance suivante. \vspace{.4cm}\\

\noindent\textbf{Mardi 18 octobre 2016 }\\
\bu\ Cours n°4 : terminé. \\
\bu\ Pensez à bien manipuler ces exemples simples chez vous avant la séance suivante. \vspace{.4cm}\\

\noindent\textbf{Lundi 10 octobre 2016 }\\
\bu\ Cours n°3 : terminé. \\
\bu\ À faire pour la séance suivante : faire l'exercice n°7 du chapitre 3. \\
\bu\ Pensez à bien manipuler ces exemples simples chez vous avant la séance suivante. \vspace{.4cm}\\

\noindent\textbf{Lundi 3 octobre 2016 }\\
\bu\ Cours n°3 : partie 2 terminée (boucles for et invariants). \\
\bu\ À faire pour la séance suivante : justifier la seconde fonction de test de primalité avec un invariant. \\
\bu\ Pensez à bien manipuler ces exemples simples chez vous avant la séance suivante. \vspace{.4cm}\\

\noindent\textbf{Lundi 26 septembre 2016 }\\
\bu\ Cours n°3 : partie 1 terminée (boucles if-then-else) et partie 2 commencée (boucles for). \\
\bu\ À faire pour la séance suivante : exercice 1 du cours n°3 et écrire la fonction somme(n) calculant 1+2+...+n. \\
\bu\ Pensez à bien manipuler ces exemples simples chez vous avant la séance suivante. \vspace{.4cm}\\


\noindent\textbf{Lundi 19 septembre 2016 }\\
\bu\ Cours n°2 terminé. \\
\bu\ À faire pour la séance suivante : exercices 2 et 11 du cours n°2. \\
\bu\ Cours n°3 commencé (boucles conditionnelles). Nous reprendrons le cours suivant dessus, en reprenant l'exemple de la recherche des racines réelles du trinôme $ax^2+bx+c$. \\
\bu\ Pensez à bien manipuler ces exemples simples chez vous avant la séance suivante. \vspace{.4cm}\\

\noindent\textbf{Lundi 12 septembre 2016 }\\
\bu\ Cours n°2 : parties 2 (types simples) et 3 (types composés) terminées. \\
\bu\ Pensez à bien manipuler ces exemples simples chez vous avant la séance suivante. \vspace{.4cm}\\

\noindent\textbf{Lundi 05 septembre 2016 }\\
\bu\ Cours n°1 : terminé. Cours n°2 commencé (explication des types). \vspace{.4cm}\\

\noindent\textbf{Jeudi 01 septembre 2016 }\\
\bu\ Journée de rentrée ; distribution des cours n° 1 et 2 et du TP n° 1. 

\label{end}
\end{document}


