\documentclass[a4paper,11pt]{scrartcl} % définit la classe
\usepackage[french]{babel} % support du français
\usepackage[T1]{fontenc} % gestion des fontes
\usepackage[utf8]{inputenc} % on travaille en utf8
\usepackage{hyperref} % pour les liens hypertexte
\usepackage{amsmath} % pour les maths de base

\author{Vos professeurs}
\date{\today}
\title{Exemple de fichier \LaTeX}

\begin{document}

\maketitle{}

\section{Une partie numérotée}

On peut écrire du texte librement, le système s'occupe de le mettre en forme. 

\subsection{Une sous-partie numérotée}

On peut aussi écrire des mathématiques dans le texte : $0 = x^2 - 2x$, mais aussi dans un environnement

\begin{equation*}
	\frac{1}{2\pi} \int_0^{\pi/2} \cos(t)\mathrm{d}t = 1515. 
\end{equation*}

\section*{Une partie non numérotée}

Pour un guide complet (lien cliquable) : 

\begin{center}
	\href{http://mirrors.ctan.org/info/lshort/english/lshort.pdf}{The not so short introduction to \LaTeX$2\varepsilon$}
\end{center}
ou bien le \href{https://en.wikibooks.org/wiki/LaTeX}{wikilivre} !

\end{document}